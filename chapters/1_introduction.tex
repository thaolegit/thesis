\chapter{Introduction}
\label{ch:intro}

In recent years, there have been major advances in technology and molecular biology. Technology has been of great use in the field of biology aiding in research as well as making the content more accessible. This project focuses on the display and design of protein structure.

A protein is not a single substance; there are many different proteins in an organism or cell, and they come in every shape and size, each performing a unique and specific job \parencite{noauthor_introduction_nodate}. Proteins are considered the ``ultimate players in the processes that allow an organism to function and reproduce'' \parencite{stephenson_protein_2016}.

Proteins are made up of linear chains of amino acids, called polypeptides. Each protein is formed by one or more polypeptide chains, linked together in a specific order \parencite{noauthor_introduction_nodate}. Proteins are the fundamental components of all living cells \parencite{hutchison_protein_2013}. Proteins have a countless number of functions that are extremely important in the biology of many organisms. They form enzymes to speed up reactions by break-down, link-up, or rearranging the substrates \parencite{noauthor_introduction_nodate}. They form hormones to control specific physiological processes such as ``growth, development, metabolism and reproduction'' \parencite{noauthor_introduction_nodate}. To maintain these roles, the shape of a protein is critical. If the shape changes, the protein will lose its functionality. There are four levels of protein structure: primary, secondary, tertiary, and quaternary \parencite{noauthor_introduction_nodate}.
Knowing the structure of a protein makes understanding of the function of that protein much easier. By being able to manipulate the structure of a protein, scientists can create hypotheses about how to affect, how to control or how to modify protein to design mutations and change a protein's function. 

The year 2020 substantiates the importance of studies in molecular biology. The impact of Severe Acute Respiratory Syndrome Coronavirus-2 (SARS-CoV-2) is worldwide, as it is ``a newly emerging, highly transmissible and pathogenic coronavirus in humans that has caused the ongoing global public health emergencies and economic crises'' \parencite{mittal_covid-19_2020}. At the time of writing, the number of infections worldwide has reached millions, while the death toll is in the hundreds of thousands. In the efforts to develop a vaccine, much research has been conducted. Some research developed on the protein structure of SARS-CoV-2 has provided insight into its evolution. As Wiesława has pointed out: ``The chief characteristic of proteins that allows their diverse set of functions is their ability to bind other molecules (proteins or small-molecule substrates) specifically and tightly.'' \parencite{hutchison_protein_2013}, particular to SARS-CoV-2 are the protein spikes that the virus uses to bind with and enter human cells \parencite{wrobel_sars-cov-2_2020}. The spikes of SARS-CoV-2 are highly stable, and help to bind to human cells tightly. Therefore, analysing the structure of these spikes could provide clues about the virus’s evolution. The study of the structure of spike proteins can aid with drug discovery and vaccine design. 
Understanding the new importance of implementing IT in biological research, \textbf{this project aims to aid with protein structural study as well as to raise interest in protein design.}

Due to limitations, this project only provides the first steps in bringing the visualisation of protein into AR and creating a simple protein structure in an iOS application using ARKit framework. \textbf{The main goal of this project, however, is to visualise protein structure in AR using an iOS App and allow user interaction with the structures.} 
There are various previous studies on protein visualisation in 3D and VR, however, studies pertaining to AR are limited, especially on iOS. This project proposed the implementation of protein structures display to serve as a trial for future study and research as it may provide more a visually appealing display than simple 3D, and reduces the side effects of VR. All protein models to be displayed are retrieved from \href{https://www.rcsb.org/}{RCSB Protein Data Bank}. 

The aim of this project is to visualise protein structures in two ways. The first way is directly displaying the complex protein models from RCSB PDB, and the second way is visualising the design of a simple protein structure that user created. The second function is implemented so that this project's application is not only appealing to biologists but can also be used by anyone curious about biology. The ability to construct a protein structure as a mini-game might make it easier for users to understand more about protein structure. 

In this project, a mobile application for the iOS system was developed: \textbf{ProteinAR}. This app has two main categories: \textbf{education} and \textbf{mini-game}. 

The \textbf{education} category assumes that the users have prior knowledge of proteins. They can input the name (ID) of a protein and get the 3D visualisation of the protein structure in AR. Users can study the protein by zooming in, turning, and flipping the protein structure. Due to the complexity of protein structures, it is not easily observed even under advanced microscopes. Thus, the ability to interact with the structures in this way can be largely beneficial to researchers. Moreover, since this is a mobile app, users can interact and discuss the structure with other users at the same time, which can be considered a promising tool for study and research on proteins.

The \textbf{mini-game}, is user-friendly and accessible to those who are unfamiliar with proteins or biology in general. Users can combine the polypeptide chains (i.e. Flex Coil, Rig Coil, Helix, Sheet) to create a new protein. This might make the study of proteins sound more appealing to users. and motivate those who wish to enter the field. In this game, users are also able to interact with the polypeptide chains and protein models. 

Last but not least, ProteinAR integrates other functions to make the app more interesting, such as enabling photo-capture of the proteins, video-capture of the process, and providing users with additional information about protein.

This paper will elaborate on the background and research, the problems and solutions, the design and implementation, and the final evaluation of the project.Due to time constraints as well as the ongoing pandemic, there were some limitations to the project, which would also be mentioned in the paper. 

Finally, this paper will discuss some critical points in dealing with the fairly new AR technology, especially use of the ARKit framework concerning: the feasibility of retrieving and displaying PDB contents, the usability of the app (AR) and room for future work.

In this paper, Chapter \ref{ch:litRev} presents reviews on current AR and VR technology used in protein display from research articles with some finding conclusion. Chapter \ref{ch:methodology} introduces the technological background of software and language used in developing ProteinAR. Chapter \ref{ch:analysis2} proposes the requirement functions as solutions for the current problems while Chapter \ref{ch:design} sketches out the layout of the app based on these solutions. Chapter \ref{ch:implementation} discusses the implementations of the design with code snippets and explanations. After that, in Chapter \ref{ch:evaluation}, some forms of test were conducted to evaluate the project. Finally, Chapter \ref{ch:conclusion} presents the conclusion of the project and the direction for future work. Some full code snippets can be found in Appendix A. 

Some important technical notes about the project:

\textbf{ProteinAR} was designed in Xcode 12, written in Swift 5, on MacOS version: Catalina 10.5.5. Because there is no support for AR on MacOS, the built-in simulator is unable to display AR functions and can cause some other errors. The project was run and tested on an iPhone. The attached demo video is recorded on iPhone X, iOS version 14. Other versions of Xcode or macOS or iOS might be unable to run ProteinAR and might also generate some unwanted errors. 


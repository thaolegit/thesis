\chapter{Introduction}
\label{ch:intro}

In recent years, there have been major advances in technology and molecular biology respectively. Technology has become a great help for biologists aiding their research and make contents easier to study. This project focuses on protein structure displaying and protein structure design.

A protein is not a single substance. There are many different proteins in an organism or in a cell, and they come in every shape and size, performing a unique and specific job \parencite{noauthor_introduction_nodate}. Proteins are considered the ``ultimate players in the processes that allow an organism to function and reproduce'' \parencite{stephenson_protein_2016}.

Proteins are formed by linear chains of amino acids, called a polypeptide. Each protein is formed by one or more polypeptide chains, linked together in a specific order \parencite{noauthor_introduction_nodate}. Protein are the fundamental components of all living cells \parencite{hutchison_protein_2013}. Protein have a countless number of functions that are extremely important in the biology of many organisms. They form enzymes to speed the reactions up by break-down, link-up, or rearranging the substrates \parencite{noauthor_introduction_nodate}. They form hormones to control specific physiological processes such as ``growth, development, metabolism and reproduction'' \parencite{noauthor_introduction_nodate}. To maintain these roles, the shape of a protein is critical. If the shape changes, the protein will lose its functionality. There are four levels of protein structure: primary, secondary, tertiary, and quaternary \parencite{noauthor_introduction_nodate}.
Knowing the structure of a protein makes understanding how that protein works much easier. By being able to manipulate the structure of a protein, scientists can create hypotheses about how to affect, how to control or how to modify protein to design mutations and change a protein's function. 

The year 2020 substantiates the importance of studies in molecular biology. We all have experienced the Severe Acute Respiratory Syndrome Coronavirus-2 (SARS-CoV-2) as is it ``a newly emerging, highly transmissible and pathogenic coronavirus in humans that cause the global public health emergencies and economic crises'' \parencite{mittal_covid-19_2020}. As of the present time of this project, the number of infections worldwide has reached millions, including thousands of deaths. To find a cure, much research has been conducted. Some research developed on the protein structure of SARS-CoV-2 has provided insight into its evolution. As Wiesława has pointed out: ``The chief characteristic of proteins that allows their diverse set of functions is their ability to bind other molecules (proteins or small-molecule substrates) specifically and tightly.'' \parencite{hutchison_protein_2013}. Particular to SARS-CoV-2 are the protein spikes that. The virus uses these to bind with and enter human cells \parencite{wrobel_sars-cov-2_2020}. The spikes of SARS-CoV-2 are highly stable and thus help to bind to human cells tightly. Therefore, analysing the structure of these spikes could provide clues about the virus’s evolution. The study of the structure of spike proteins can aid with drug discovery and vaccine design. 
Understanding the new importance of implementing IT in biology research, \textbf{this project aims to aid with protein structural study and raise interest in protein design.}

Due to the shortage of time and lack in experience, this project only provides the first step into bringing the visualisation of protein into AR-display and creating a simple protein structure in an iOS application using the framework ARKit. \textbf{The main goal of this project, however, is to visualise protein structure on AR using an iOS App and let user interact with the structures.} 
There are various previous studies on protein visualisation on 3D and VR, however, studies pertaining to AR is limited, especially the AR app on iOS. This project proposed the implementation of displaying protein structures to serve as a trial for future study and research as it might make displaying more appealing than simple 3D, and also cut down on the side effects of VR. All the protein models that are to be displayed are retrieved from \href{https://www.rcsb.org/}{RCSB Protein Data Bank}. 

This project's app aims to visualise protein structures in two ways. The first way is directly displaying the complex protein models from RCSB PDB, and the second way is visualising the design of a simple protein structure. The second function is implemented so that this project's application is not only appealing to biologists but also can be used by anyone curious about biology. Being able to construct a protein structure as a mini-game might make it easier for users to understand more about protein structure. 

In this project, a mobile application for iOS system was developed: \underline{\textbf{ProteinAR}}. This app has two main categories: \textbf{education} and \textbf{mini-game}. 

The \textbf{education} category assumes that the users have previous knowledge of proteins. They can input the name of protein and get the 3D visualisation of the protein structure in AR. Users can study the protein by zooming in, turning, and flipping the protein structure. Due to the complexity of protein structures, it is not easily observed even under advanced microscopes. Thus, the ability to zoom in and all other interactions can benefit researchers. Moreover, since this is on a mobile app, users can interact and discuss the structure with other users at the same time, which can be considered a promising tool for study and research on proteins.

The \textbf{mini-game}, is user-friendly to those who are unfamiliar with proteins or biology in general. Users can add the polypeptide chains onto each other (i.e. Flex Coil, Rig Coil, Helix, Sheet) to create a protein. This might make the concept of proteins sound more appealing to users and thus, motivate the wish to study more about protein. In this game, users are also able to interact with the polypeptide chains and protein models. 

Moreover, ProteinAR integrates other functions to make the app more interesting, such as enabling photo-capture of the proteins, video-capture of the process, and providing users with addition information about protein.

This paper will elaborate on the background and research, the problems and solutions, the design and implementation, and the final evaluation of the project. In the short period of time and the given circumstance of Covid-19, there were some limitations to the project, which would also be mentioned in the paper. 

Finally, the paper will discuss some critical points in dealing with the fairly new AR technology, especially using ARKit framework on, concerning:
\begin{itemize}
\item The feasibility of retrieving and displaying PDB contents
\item  The usability of the app (AR)
\item Room for future work
\end{itemize}

Some important technical notes about the project:

\textbf{ProteinAR} was designed on Xcode 12, written in Swift 5, on MacBook OS version: Catalina 10.5.5. There is no support for AR on MacOS, thus, the built-in simulator will not be able to display the AR function and can cause some other errors. The project was run and tested on an iPhone. The attached demo video is recorded on iPhone X, iOS version 14. Other versions of Xcode or macOS or iOS might be unable to run ProteinAR and thus might generate some unwanted errors. 


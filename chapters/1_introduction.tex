\chapter{Introduction}
\label{ch:intro}

In recent years, along with the advancement of technology, there are major advances in molecular biology. Technology has become a great help for scientists and biologists aiding their research and make things easier to study. This project focuses on protein structure displaying and protein structure design.

Protein is not a single substance. There are many different proteins in an organism or in a cell that comes in every shape and size, doing a unique and specific job \parencite{noauthor_introduction_nodate}. Proteins are considered as the ``ultimate players in the processes that allow an organism to function and reproduce'' \parencite{stephenson_protein_2016}.

Proteins are formed by linear chains of amino acids. A linear chain of protein is called a polypeptide. Each protein is formed by one or more of polypeptide chains, linked together in a specific order \parencite{noauthor_introduction_nodate}. Protein are the fundamental components of all living cells \parencite{hutchison_protein_2013}. Protein has a countless number of functions in a cell or organism that are extremely important in the biology of many organisms. They form enzymes to speed the reactions up by break-down, link-up, or rearrange the substrates \parencite{noauthor_introduction_nodate}. They from hormones to control specific physiological processes such as ``growth, development, metabolism and reproduction'' \parencite{noauthor_introduction_nodate}. To maintain these roles, the shape of a protein is critical. If the shape changes, the protein will lose its functionality. There are four levels of protein structure: primary, secondary, tertiary, and quaternary \parencite{noauthor_introduction_nodate}.
Knowing the structure of a protein makes understanding how that protein works much easier. By being able to manipulate a structure of a protein, scientists can create hypotheses about how to affect, control modify them to, for example, design mutations to change functions. 

This year of 2020 has once again proven the importance of molecular biology study. As of this year, we all have experienced the Severe Acute Respiratory Syndrome Coronavirus-2 (SARS-CoV-2) as is it ``a newly emerging, highly transmissible and pathogenic coronavirus in humans that cause the global public health emergencies and economic crises'' \parencite{mittal_covid-19_2020}. The number of infections worldwide had reached millions, including thousands of deaths. To find a cure, much researches have been carried out. Some research developed on the protein structure of SARS-CoV-2 has provided some insights into its evolution. As Wiesława has pointed out: ``The chief characteristic of proteins that allows their diverse set of functions is their ability to bind other molecules (proteins or small-molecule substrates) specifically and tightly.'' \parencite{hutchison_protein_2013}. The characteristic of SARS-CoV-2 is the protein spikes that cover the surface. The virus uses this to bind with and enter human cells \parencite{wrobel_sars-cov-2_2020}. The spike of SARS-CoV-2 is very stable and thus help to bind to human cell tightly. Therefore, analysing the structure of theses spikes could provide clues about the virus’s evolution. The study of the structure of the spike protein can aid with drug discovery and vaccine design. 
Understanding the new importance of implementing IT in Biology research, this project aims to aid with protein structural study and raise interest in protein design.
Due to the shortage of time and lack in experience, this project only provides the first step into bringing the visualisation of protein into AR-display and creating simple protein structure in an iOS application using the framework ARkit. The main goal of this project, however, is to \textbf{visualise protein structure on AR using an iOS App and let user interact with the structures.} 
There are various previous studies on protein visualisation on 3D and VR, however, there has not been much on AR, especially AR app on iOS. This project proposed the implementation of displaying protein structures to serve as a trial for future study and research as it might make displaying more appealing than simple 3D and also cut down on the side effects of VR. All the protein models that are to be displayed are retrieved from \href{https://www.rcsb.org/}{RCSB Protein Data Bank}. 

The app aims to visualise protein structures in two ways. First way is to directly display the complex protein models from RCSB and the second way is visualising the design of simple protein structure. The second function is implemented so that this project's application is not only appealing to biologists and scientists but also can be used by anyone curious about biology. Being able to construct a protein structure as a mini-game might make it easier for users to understand more about protein structure. 

In this project, a mobile application for iOS system was developed: ProteinAR. ProteinAR has two main categories: \textbf{Education} and \textbf{Mini-game}. 
The \textbf{Education} category assumes that the users have already known about proteins, they can input the name of protein and get the 3D visualisation of the protein structure in AR. Users can study the protein by zooming in, turning, flipping the protein structure. Due to the complexity of protein structure, even under microscope it is not easy to look through. Thus, being able to zoom in and all other interactions will definitely benefit researchers. Moreover, since this is on a mobile app, users can interact and discuss the structure with other users at the same time, which can be considered as a promising tool for study and research on proteins.

The \textbf{Mini-game}, on the other hand, is user-friendly to users who are not familiar with proteins or biology in general. Users can add the polypeptide chains namely: Flex Coil, Rig Coil, Helix, Sheet onto each other to create a protein. This might make the concept of protein sounds more appealing to the user and thus, motivate the wish to study more about protein. In this game, users are also able to interact with the polypeptide chains and protein models. 

Besides, ProteinAR also integrates other functions to make the app more interesting such as enable taking photo of the proteins, or recording the video during the process as well as provide users some information about protein.

This paper will elaborate the background and research, the problems and solutions, design and implementation and the final evaluation of the project. In the short period of time and the given circumstance of Covid-19, there were some limitations to the project, which would also be mentioned in the paper. 

Finally, the paper will conclude some critical points in dealing with the fairly new AR technology, especially using ARKit on:
- The feasibility of retrieving and displaying PDB contents
- The usability of the app (AR)
- Room for future work

Some important technical notes about the project:
ProteinAR was designed on Xcode 12, based on Swift, on MacBook OS version: Catalina 10.5.5. There is no support for AR on MacOS, thus, the built-in simulator will not be able to display the AR function and can cause some other errors. The project was run and tested on an iPhone. The attached demo video is recorded on iPhone X, iOS version 14. Other versions of Xcode or macOS or iOS might not be able to get ProteinAR running and thus might generate some unwanted errors. 


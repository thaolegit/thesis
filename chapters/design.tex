\chapter{Project Design}
\label{ch:design}

\section{Application Design}
The structure of ProteinAR is very simple. It consists of four main screens including the landing screen. 
 

To achieved the set main goals and still create a user-friendly app, the two goals are separated to be achieved on two different views. 
From the landing view (first view), user can choose to go to the “Introduction” about the app, or go to “Education” where visualization of protein happens or “Mini-game” where new proteins can be created by joining the coils, helix and sheet in different order. 
This app was created based on single view iOS app and then the other views are added later with segues to create the connection between them.

\subsubsection{User interface}
\subsubsection{Introduction to user interface}
In order to ask the computer/smart devices to do anything, users need to communicate with them. The way users (human) can communicate with the product (software, app, website) is through interacting with the user interface (UI) of that product. The purpose of a UI is to enable users to control a computer or a device they are interacting with, by giving orders and receiving feedbacks in a chain to complete a task. 
The user interface of any software or website does not only create the first impression to the products which makes user decide instantly if they want to use the product, but it also plays a big role in keeping the users. With a complicated or not efficient UI, users would not want to keep using the product because it takes memory efforts. Therefore, an UI should be \emph{intuitive} – be kept simple where no training should be needed to operate, \emph{efficient} – functions are precise and on point, and \emph{user-friendly} (User Interface, n.d)
Currently, there are three formats of user interfaces (What is User Interface Design, n.d?):
•	 \textbf{Graphical User Interfaces (GUIs)} – interactions happens through visual representations on digital control panels such as a computer desktop, a website interface.
•	 \textbf{Voice-controlled interfaces(VUIs)} – interactions happens through voice representation such as Siri, Google Home or Alexa.
•	 \textbf{Gesture-based interfaces} – interactions happen through physical motions in 3D spaces in VR games.

\subsubsection{ProteinAR’s user interface}


\chapter{Testing and Evaluation}
\label{ch:evaluation}

In this chapter, conducted test results will be reported. These tests include: function testing, unit testing, performance testing and usability testing. The result of each tests, as well as the analysis, and limitations of the tests will be provided. Finally, an overall evaluation of the project is summarised in table \ref{tab:evaluation}.

\section{Function Testing}
\subsection{Education Screen}
When the protein ID is inputted, the model of the protein is displayed.
In Figure \ref{fig:eduscreen} on the left, the protein 6K01 is displayed on the AR screen as 6K01 is a valid protein ID. On the right, when the protein is invalid, the screen shows only 3D text informing the user that protein does not exist.
 \begin{figure}[!htp]
	\centering
	\includegraphics[scale=0.6]{images/eduscreen.png}
	\caption{Education App Screen}
	\label{fig:eduscreen}
\end{figure}

\subsection{Mini-game screen}
On the Mini-game screen, when a polypeptide chain button is pressed, the model of that chain will be displayed. Similarly, when another polypeptide chain button is pressed, the second model appears on top of the first one as shown in Figure \ref{fig:minigamescreen1}. If the "Try" button is then pressed and the combination exists, it will be displayed with text showing the names of its elements. If the combination does not exist, only text will appear as shown in Figure \ref{fig:minigamescreen2}.
 \begin{figure}[!htp]
	\centering
	\includegraphics[width=\textwidth]{images/minigamescreen1.png}
	\caption{Mini-game App Screen (1)}
	\label{fig:minigamescreen1}
\end{figure}

 \begin{figure}[!htp]
	\centering
	\includegraphics[scale=0.6]{images/minigamescreen2.png}
	\caption{Mini-game App Screen (2)}
	\label{fig:minigamescreen2}
\end{figure}

\section{Unit Testing}
By running the app, only functions that can be displayed on the screen can be tested. Functions to download the PDB files cannot be tested in this way. Therefore, in this project a unit test was built to test the download function. 
In Xcode, the \emph{XCTest} is a built-in framework for writing unit tests. \emph{XCTest} asserts that during code execution, certain conditions are satisfied and if not, the errors messages will be shown together with the test failure result. 
The full code for the unit test can be found in Appendix A. In Figure \ref{fig:testdownload}, the main part of the code is shown. In this test function, the URL is given with a valid protein ID (6K03) and the code checks if the file 6K03.pdb exists in the \emph{Document directory} after the download function completes. The green tick on the function shows that the tested function (download) works. 

 \begin{figure}[!htp]
	\centering
	\includegraphics[width=\textwidth]{images/testdownload.png}
	\caption{Unit Test - Download function}
	\label{fig:testdownload}
\end{figure}

\section{Performance Testing}
Since ProteinAR is an iPhone app, evaluating how the app performs on an iPhone is important. Xcode has a built-in debug navigator to show how the app performs on the device. In this navigator, there are reports to visualise how the application impacts the running of a simulator device. 
Figure \ref{fig:cpu} shows the impact on iPhone's CPU while the app is running. The CPU percentages fluctuates frequently, though it maintains a range between 80 and 120 percent. The testing device is iPhone X with six cores, bringing the maximum CPU capacity to 600 percent. Since the app performs many tasks at the same time (recognising the real-world's surfaces, putting layers on, downloading from the web, displaying models, etc.,), this is considered acceptable. In Figure \ref{fig:cpu}, the percentage used is shown to remain in the green zone. 
 \begin{figure}[!htp]
	\centering
	\includegraphics[width=\textwidth]{images/CPU.png}
	\caption{CPU usage}
	\label{fig:cpu}
\end{figure}

As for memory, the app has low memory consumption. In future work, when a conversion function is made and the app can display models from downloaded PDB files, memory will not be a problem as the data will be stored in \emph{Document directory}. A visualisation of memory usage is shown in Figure \ref{fig:memory}.

 \begin{figure}[!htp]
	\centering
	\includegraphics[width=\textwidth]{images/memory.png}
	\caption{Memory usage}
	\label{fig:memory}
\end{figure}

An AR app should have a frame rate of at least 30 FPS to allow the app to run smoothly and save CPU and GPU usage. In ProteinAR, the frame rate is relatively high, at 60 FPS. This creates smooth movement, but also requires substantial GPU usage (Figure \ref{fig:FPS}). This might also lead to extra energy usage, negatively affecting the energy impact of the app (Figure \ref{fig:energy2}). 

 \begin{figure}[!htp]
	\centering
	\includegraphics[width=\textwidth]{images/FPS.png}
	\caption{GPU Usage}
	\label{fig:FPS}
\end{figure}

 \begin{figure}[!htp]
	\centering
	\includegraphics[width=\textwidth]{images/energy2.png}
	\caption{Energy impact}
	\label{fig:energy2}
\end{figure}

Through observation, when the app starts running, the energy impact is already stated in the report as ``Very High". The thermal state starts at ``Fair" and in only a few minutes changes to ``Serious". Sometimes, the thermal state increases to ``Critical" if the app is left running. 
This negatively affects app performance, and drains the device’s battery faster than normal. If the device overheats, loaded models lag, and extra screens (i.e. the “Help Screen”) fail to appear when clicked due to the excessive heat generated by the UI elements.
The performance evaluation is summarised in table \ref{tab:perEvaluation}.

\begin{table}[!h]
\centering
\begin{tabular}{| m{0.15\textwidth} | m{0.15\textwidth} | m{0.25\textwidth}| m{0.25\textwidth}|}
\hline
Category & Device impact & Advantages & Disadvantages \\
\hline
\hline
CPU/GPU& High & Multi-tasking, smooth transition & Increase energy impact  \\
\hline
Memory & Normal & No extra workload on the device & N/A \\
\hline
FPS & High & Smooth display of models and AR layers & Increase CPU and GPU usage \\
\hline
Energy & Very High & N/A & Freeze the app and Drain battery \\
\hline
\end{tabular}
\caption {Performance Evaluation}
\label{tab:perEvaluation}
\end{table}

To help lower the energy impact, more testing needs to be implemented in future work. 

\section{Application Usability Testing}

Usability evaluation is an important evaluation that any system must take before product release. This helps ensure the app meets the requirements of a business and secures customer satisfaction. By doing usability testing, the app can be improved based on user feedback. 

In this project, the usability evaluation was conducted based on the post-testing usability questions. There are ten questions, adapted from the SUS (System Usability Scale) questionnaire. This means the questions are asked after users have had experienced using the app. 
However, the project ran into two problems while conducting usability testing. 

\begin{itemize}
\item Limitation of testing subjects: 

For users to remotely download and use the app, ProteinAR has to be available on the App store. After an app is submitted to the App store, it moves through a very strict review process and may not be available for a period of time. This would require more work to complete the app since an app with remaining issues would not make it through the review process. Furthermore, uploading an app on the App store categorises it as a commercial product which may go against UCC policy. Therefore, it was not possible for other users to remotely test the app. Alternatively, there are two methods for the app to be tested by other users which require in-person meeting:
	\begin{itemize}
		\item Users directly use the app on developer's device or downloads from the developer's computer. With the ongoing complications of Covid-19, this was not advisable.
		\item User can install the app by downloading the project on Github. However, this requires target users to have access to a MacOS system and an iPhone 6 or onwards with the compatible MacOS and iOS version.

Since the second method was difficult to conduct, all the tests were carried out in the first method, Due to the previous mentioned difficulty of Covid-19, the number of users who conducted the test was limited to five people.

	\end{itemize}
\item Lack of objectivity in questionnaire result:

For all tests, the users and developers were in the same room. This removed the anonymity of the test introducing bias to the results.

\end{itemize}

The tests were conducted despite the above limitations, and the questionnaire results are shown in Figure \ref{fig:survey}. Prior to the test, users were provided protein names to start with. Since there are limited buttons on the screen, it did not take much time for users to learn how to navigate and use the app. The element of using Augmented Reality impressed users as it is new and considered ``exciting" and ``cool", which was a major factor in increasing user motivation for continued use. However, the app is about proteins, which might not be the subject of choice for most users. In effect, users indicated that they would not recommend the app to others, suggesting that the entertainment element does not meet with user expectations. 

 \begin{figure}[!htp]
	\centering
	\includegraphics[width=\textwidth]{images/survey.png}
	\caption{System Usability Scale (SUS) Questionnaire and Feedbacks}
	\label{fig:survey}
\end{figure}

The extra functions linking the app to RCSB website or PDB 101 were not intentionally chosen. Users indicated that reading information on another website required undesired effort. This could be replaced by simpler but educational screens or websites. The AR triggered excitement in using the app, and therefore, should be the main focus for future development. The mini-game would be more interesting if the new protein created could be linked to daily life such as ``found in human skin, found in sheep wool, etc.". 

\section{Overall Evaluation}
Based on the tests conducted, the overall project evaluation is summarised in table \ref{tab:evaluation}.

\begin{table}[!h]
\begin{tabular}{| m{0.15\textwidth} | m{0.4\textwidth} | m{0.25\textwidth}| m{0.15\textwidth}|}
\hline
GOALS & IMPLEMENTATION & ACHIEVED RESULT & EVALUATION\\
\hline
\textbf{Download Protein from}

\textbf{RCSB PDB} & 
\begin{itemize}
	\item Download using \emph{downloadTask()} using constructed URL
	\item Save files to Document directory
	\item Assign attributes to Core Data
\end{itemize} &
Based on the result from Unit Testing, these functions work without errors.&
\textbf{Completed}\\
\hline

\textbf{Visualise Protein models} 

\textbf{on AR} & 
\begin{itemize}
	\item Convert PDB file to Collada files
	\item Visualise by loading 3D models on AR screen using function \emph{displayProtein} (Figure: \ref{fig:displayreal})
\end{itemize} &
\begin{itemize}
	\item Conversion function was not completed
	\item Can visualise pre-downloaded sample models
\end{itemize}&
\textbf{Partially}

\textbf{completed} \\
\hline

\textbf{Create}

\textbf{new Protein} 

\textbf{on AR} & 
\begin{itemize}
	\item Display individual polypeptide chains with function \emph{addProtein()} (Figure: \ref{fig:addProteinfunc})
	\item Display combinations of polypeptide chains according to user input using function \emph{createProtein} (Figure: \ref{fig:createProtein}) 
\end{itemize} &
Displaying individual polypeptide chains and clearing them to display the combination creates a smooth transition and interactive experience for the user.&
\textbf{Completed}\\
\hline

\textbf{Interact with} 

\textbf{Protein models} &
 Use UIGestureRecognizer for 
 \begin{itemize} 
	 \item Pinch Gesture (allow zooming in and out) (Figure:\ref{fig:pinch})
	 \item Rotation Gesture (allow rotating models) (see the Appendix A)
	 \item Pan Gesture (allow moving models) (see the Appendix A)
\end{itemize} &
Based on the user testing questionnaire, the use of interactions is satisfactory.&
 \textbf{Completed}\\
 \hline
 
 \textbf{Appealing UI} &
 \begin{itemize}
 	\item Use two different screens for two different purposes (education and mini-game)
	\item Easy to navigate
	\item Analogous colour theme
	\item Interactive elements (buttons, display, model-interactions, create new proteins) (Chapter: \ref{ch:design})
\end{itemize} &
Based on the user testing questionnaire, the UI is easy to use and appealing. &
\textbf{Completed}\\
\hline

\textbf{Good}

\textbf{performance}&
 \begin{itemize} 
	 \item CPU Usage: High
	 \item GPU Usage: High
	 \item Memory Usage: Normal
	  \item Energy Impact: Very High
\end{itemize} &
The thermal state reaches “Serious”  after a short period of use, which is not ideal for the app. Further tests are needed to solve this problem. &
\textbf{Partially}

\textbf{completed}\\
\hline
\end{tabular}
\caption{Overall Evaluation}
\label{tab:evaluation}
\end{table}
Although much future work is desired for the completion of the app, ProteinAR succeeded in creating the first step to bringing protein visualisation to AR. Without the need to print materials, or use extra devices such as VR goggles, ProteinAR can be developed to become a useful tool for students, teachers, and scientists in the study and research of proteins.



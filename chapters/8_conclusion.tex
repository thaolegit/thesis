\chapter{Conclusion}
\label{ch:conclusion}

\section{Conclusion}
\section{Future Work}
Given the limitations of the project, there are plenty of rooms for improvement with ProteinAR. Moreover, Augmented Reality technology is relatively new and Apple has been acquiring new companies specialising in AR to update the ARKit framework rapidly, thus, possibilities for the development of the app in the future are promising.

First and foremost, the \textbf{protein real-time visualisation} is not completed because it is missing a function to convert PDB file type to Collada file type. With this function completed, ProteinAR could become useful in biology class as it displays any protein's structure in real time. This should be the main focus of the future work.

Secondly, the \textbf {new protein creation} can be improved in the following directions:
\begin{itemize}
	\item Currently, the combinations of polypeptide chains are pre-loaded in a folder. In the future, if this too can be generate in real time, using server such as I-TASSER, there would be more combinations that are created, and thus, not only tertiary but quaternary structures can also be generated.
	\item The newly created protein, if not new, should also be linked to some information such as its parameter, its function, etc. This can be achieved using POST and GET REQUEST to the available source of information.
	\item The newly created protein should be exportable into a PDB file. This way, it could be used more for research purposes.
\end{itemize}

Thirdly, more \textbf{interactive elements} can be added. ProteinAR can integrate Machine Learning and CoreML to control ARKit. There are many ways to implement this. One of the popular way is to combine image classification and AR to create new experience by experimenting with hand gesture recognition. Photos of hand poses can be taken then used in training models such as TensorFloe, Keras, Custom Vision. Xcode also has its training interface. The models can be trained so that users can use hand poses to control the protein models. Moreover, the interactions between users and the protein models could become more realistic if the connection points in the models can be bended and twisted. Currently, ProteinAR only allow user to pinch, rotate and pan the models.
 
Last but not least, \textbf{more functions} can be integrated into the app. The menu function is now only link the app to RCSB home page, with no specific information. These functions can be more customised to be suitable with the purpose of usage. 
With the development of AR technology, ProteinAR can be improved much more and might become a useful tool for biologists and biology students. 


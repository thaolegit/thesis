\chapter{Analysis: Review \& Research on the field and existing products}
\label{ch:litRev}

\section{Introduction to Protein}

As mentioned, proteins are “the most important macromolecules in all living organisms” \parencite{rashid_protein_nodate}. Sequences of amino acids which bind into linear chains create proteins. These chains have a specific folded three-dimensional (3D) shape, which enables the protein to perform a certain task \parencite{rashid_protein_nodate}.The shape of the protein defines its tasks, thus, knowing the protein structure is very important. There are four different levels of protein structures: Primary Structure, Secondary Structure, Tertiary Structure, and Quaternary Structure. A sequence of amino acids in a chain forms a \emph{Primary structure}. These chains, then, would fold into three different shapes (Helix, Coil or Sheet) where the alpha helix, the beta sheet, and the random coils are positioned, which is called the \emph{secondary structure}. The combination of these chains of helix, coil and sheet (polypeptide chains) forms a 3D structure – the \emph{tertiary structure} of a protein. The \emph{quaternary structure} is a large assembly of multiple polypeptide chains (Figure \ref{fig:proteinstr}).
 \begin{figure}[!htp]
	\centering
	\includegraphics[scale=0.5]{images/proteinstr.png}
	\caption{Orders of protein structure - source \href{https://www.khanacademy.org/science/biology/macromolecules/proteins-and-amino-acids/a/orders-of-protein-structure}{Khan Academy}\parencite{noauthor_introduction_nodate}}
	\label{fig:proteinstr}
\end{figure}

To understand a protein’s function, understanding the structure of the protein is necessary. In the same way, designing a protein from the structure will help to design its function. 
In ProteinAR, users will get to design protein structure by combining different protein secondary structures of helices, coils, and sheets to form tertiary structures. 


\section{Existing solutions to protein visualisation}
 “Proteins are three-dimensional (3D) objects” \parencite{ratamero_touching_2018}. The key to understand protein functions is to understanding protein structure. Computer models for protein have become very popular for a long time. Many projects were developed to make 3D viewing of protein possible such as {\footnotesize PYMOL, CHIMERA, VMD, ISOLDE,} etc. 
 
 \subsection{Protein visualisation in mobile applications}

There are numerous mobile applications in which proteins are visualised in 3D. The RCSB Protein Data Bank (the single worldwide repository of protein data) also provides a \href{https://www.ncbi.nlm.nih.gov/pmc/articles/PMC4271143/}{mobile app} allowing data access and visualisation. Basically, the protein can be downloaded directly from the PDB from RCSB and displayed in 3D. This app is based on the open-source molecular viewer \href{https://play.google.com/store/apps/details?id=jp.sfjp.webglmol.NDKmol&hl=en}{NDKmol}. However, NDKmol can only be used on Android and not iOS. \href{https://www.imedicalapps.com/2013/08/jmol-molecular-visualization-app/}{Jmol}l is another Android app that connects to the RCSB PDB, visualising the protein in 3D once the protein name is inputted.
There are some molecule viewers that can run on iOS devices. Unfortunately, most of them are no longer in use or experienced technical difficulty, thus, are removed from the Apple App Store. \href{https://www.molsoft.com/iMolview.html}{iMolview} can still be used, however, the interface is not very user friendly. 


\subsection{Protein Visualisation in VR}
\subsubsection{The advancement of implementing VR in Protein Display}
Visualising proteins on computer in 3D has been a great step, however, it lacks thethe immersive salience of 3D presence, and leads to limitation in analysing protein structure. Virtual Reality (VR) provides a wide field of view on an immersive display and a better perception of the protein structure by head-tracking. Furthermore, VR enables users to have the freedom of hand controllers for simple manipulation and interaction with the protein instead of the conventional manipulation on 2D using a trackpad, mouse and keyboard \parencite{goddard_molecular_2018}. This makes VR entrance into the world of protein visualising/molecular biology more than welcomed. 
HMDs\footnote{Head Mounted Display} are commonly used because they are accessible, increasingly more common, and are affordable. VR games have become popular, thus the tools for programming software that are compatible with HMD are effective and cheaper. Projects such as {\footnotesize REALITYCONVERT}, {\footnotesize AUTODESK}, {\footnotesize MOLECULE VIEWER} are well developed, providing good resource for further development on protein display in VR \parencite{ratamero_touching_2018}. {\footnotesize UNITY} is largely used with the combination of HMDs such as {\footnotesize OCULUS RIFT} and {\footnotesize HTC VIVE} to display and manipulate proteins \parencite{ratamero_touching_2018}.

\begin{figure}[!htp]
	\centering
	\includegraphics[scale=0.6]{images/OculusRift.png}
	\caption{Oculus Rift (HMD) and Kinect v2 sensor placement used during Molecular Rift development}
	\label{fig:OculusRift}
\end{figure}


There have been many advanced projects of implementing VR in molecular biology. In particular,  {\footnotesize MOLECULAR RIFT} is an open source tool that creates a virtual reality environment steered with hand movements, incorporates {\footnotesize OCULUS RIFT} as the display to create the virtual setting \parencite{norrby_molecular_2015}. The combination of a virtual reality experience with natural acts such as hand movements creates a much better experience for the users than merely experiencing the 3D \parencite{norrby_molecular_2015}.

Other research shows that the technology in displaying Protein in VR is advanced, however, tools that are designed to be installed on desktop systems are often tedious \parencite{xu_vrmol_2019}.  The configurations might be different with systems and therefore cause errors. Sharing between system is also difficult. With the help of Web Graphics Library (WebGL), web-based applications such as {\footnotesize JMOL}, {\footnotesize ASTERVIEWER} are more straightforward as VR experiences can be directly accessed with common web browsers. However, there are many limitations for these web-based applications because they only support a few file types and cannot perform complex tasks for analytical purpose \parencite{xu_vrmol_2019}. A few solutions were proposed for an integrative cloud-based system that can directly access databases and uses VR technology to visualise and analyse macromolecular structures, such as {\footnotesize VRMOL}. This might be the new direction for protein visualising in VR.

\subsubsection{The limitations of using VR in Displaying Protein}

Even though the VR implementation in displaying protein has come far, limitations are inevitable.
First, the limitations in the associated hardware/software may lead to an unsuccessful application of VR, which leads to the inaccuracy and impreciseness in the results of using the application. With the increasing development of VR techniques and the gaining popularity of VR games, software and hardware to be integrated with VR are becoming more compatible, but not without limitations. They are still costly and need to be increased in fidelity \parencite{liu_using_2018}.
The second point concerns the unnatural feeling of using VR. Even though VR offers a realistic view, the users must wear goggles which are not transparent and thus block the vision of the real world. Furthermore, the head movements are unnatural because users will have to try to move their heads in order to see contents. New HMDs are better because they are much lighter but mostly VR devices are still quite bulky and are relatively difficult to use. 
Thirdly, most VR users claim to have motion sickness. This happens because of the disparity between what the body and the eyes of a user experience at the same time. The actual physical actions and the actions that are carried out in VR might be different and this cause motion sickness to the users. Due to this, VR can only be used for a limited amount of time.


\subsection{Protein Visualisation in AR}

Similar to Virtual Reality, Augmented Reality (AR) generates realism by displaying 3D models in a real-world context. However, unlike VR where the whole vision of the users is taken away and replaced by another completely different scene, AR’s defining characteristic is that it added a layer onto the vision. While VR creates an immersive experience for users by shutting out the real physical world, AR maintains the realism of the world, allowing users to see whatever they are seeing plus more. With AR, the users have free movements while projecting images. Commonly speaking, there are two well-known types of AR technology implementation. The first one is implementation on AR smart-glasses such as the Microsoft HoloLens, Google Glass, Apple Glass. Contrast to VR goggles, AR smart-glasses look similar to sunglasses or normal glasses, thus, causing no discomfort to the users. 
The second type of implementations are on AR apps such as Pokemon Go. In this type of implementation, smartphone cameras are used to track the surrounding environment as well as adding a layer on top of the screen to show external information.

As AR gains popularity, more projects are underway, but this is limited as it is an extension of VR, and it is still very new. Some studies show that AR being used in science teaching such as displaying molecular biology in AR has yielded in good results for students, as it takes less imagination and makes things easier to understand \parencite{cai_case_2014}. However, there are not many AR apps available to support education, specifically in visualising molecules. 

As mentioned, there are not many projects concerning the visualisation of molecules on AR. Unlike VR, where there are various numbers of HDMs incorporated software and app for protein visualisation, on AR, apps are more commonly used. There are only a few apps that can be found. BiochemAR is one of those. Once such app, BiochemAR, was released in 2019 and is available on both App store (for iOS) and Google Play (for android). According to the developers, the idea of the app is to create a simple, easy-to-use teaching tools for both teachers and students in the classroom \parencite{sung_biochemar_2020}. The main function is to display protein in AR by scanning a QR code, thus the design is relatively basic. When a QR code is scanned, the app will use the smart devices’ built-in camera to bring the protein structure into life through VR as shown in Figure \ref{fig:bioChemAR}.
\begin{figure}[!htbp]
	\centering
	\includegraphics[scale=0.5]{images/bioChemAR.png}
	\caption{BiochemAR app screen shot}
	\label{fig:bioChemAR}
\end{figure}



As the main purpose is to make things simple and easy to use for teachers and students, there is no other function or interactions between users and the protein. Proteins are simply visualised and users can move the phone around to look at the protein in different angles and size. 

Having the same idea, another app called  AR Assited Visualisation
was developed in 2020 to visualise proteins. These proteins are not written under QR code form but instead printed out on paper as in Figure \ref{fig:arvisualisation}. 

\begin{figure}[!htbp]
	\centering
	\includegraphics[scale=0.8]{images/arvisualisation.png}
	\caption{AR Assisted Visualisation App \parencite{eriksen_visualizing_2020}}
	\label{fig:arvisualisation}
\end{figure}

Similar with BioChemAR, AR Assisted Visualisation only display protein structure in 3D, without any interacting elements. 

\subsection{Finding summary}
In a recent research conducted by American Chemical Society and Division of Chemical Education, it seems that when undergraduate students created their own AR protein visualisation, they were enthusiastic when performing this function, thus, their learning was enhanced when the AR module was inserted to their upper level biochemistry class \parencite{argu_fast_2020}. With the trend of online learning, the application of AR offers a promising curriculum for biochemistry.

Integrating protein visualisation on mobile apps is a good solution because of its availability. Most students have access to a smart phone and it is handy  to bring around as it is not bulky nor need specific customisation. 

The AR apps on protein visualisation are relatively new (released in 2019 and 2020). Thus, there is not much user interactions and functions to it. To use the aforementioned apps, a certain document with information of the protein, whether it be a figure of a protein or a QR code, has to printed in order to get the AR visualisation. Moreover, the proteins can be viewed but cannot be interacted with in any way. Furthermore, these apps are one-side oriented as users can only view proteins but cannot create them. 

ProteinAR’s purpose is to not only let users directly view the shape of protein in AR, or interact with the protein by gesture touch on the screen, but also allow users to design and create their own proteins.
The majority of mobile apps to visualise protein are only in 3D, and mostly on Android. Therefore, the open-source API for protein visualisation directly from the PDB files are limited. This project will have to start from little availability in pre-developed techniques.

This project creates a base for ProteinAR. With further future work, it can be applied to be used in teaching to make lessons more interesting and understandable for students as well as motivate students to do higher level in Biochemistry. 





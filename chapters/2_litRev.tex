\chapter{Literature review \& Existing products research}
\label{ch:litRev}

This chapter goes through some literature on Augmented Reality (AR) and the application of AR in education, with the specific mention in the biological field. Then, a brief explanation of protein structure will be provided, follow by research on the existing solutions of protein structure display in mobile application, Virtual Reality (VR) and AR with some personal insights. Finally, the finding summary is presented.

\section{Augmented Reality and application in education}
\subsection{Augmented Reality introduction}
Augmented Reality (AR) is a technology that involves “the overlay of computer graphics on the real world" \parencite{silva_introduction_2003}. AR allows users to look "at the real world and increases it with additional information generated by a computer" \parencite{chamba-eras_augmented_2017}. AR acts as a bridge, connecting physical and virtual objects, combining two worlds into one, and enables interaction between them by adding information to the real-world in real-time \parencite{chamba-eras_augmented_2017}.In AR, the user is able to stay in touch with both contexts of the real world and the virtual world.

On the other hand, even though often being mistaken for another, Virtual Reality (VR) is used to define the technology that allows the computer to generate 3D environments that users can enter and interact  \parencite{silva_introduction_2003}. The complete scenarios are generated by computers, creating the sensation of being physically in the generated scene for the user. In VR, users lose the context of the real-world but instead, are just aware of synthetic realism.

AR is similar to VR in the way that virtual objects are generated by the computer. However, while the goal of VR is to create an immersive experience for the user by shutting down the real physical world and replace it with a completely synthetic environment, the goal of AR is to enable the user to stay in touch with the real-world, while being able to interact with virtual objects \parencite{chamba-eras_augmented_2017}. Because of this, the defining characteristic of AR is that it adds layers to the real-world vision. Using AR, users have more freedom of movements while projecting images. The main components of AR are scene generator, tracking system, and display. The scene generator is responsible for rendering the scene for binding real-world scenes and virtual objects. The tracking system is important because the objects in the real and virtual worlds must be “properly aligned with respect to each other, or the illusion that the two worlds coexist will be compromised" \parencite{silva_introduction_2003}. As for the display, currently, there two well-known types of display for AR implementation. The first one is the implementation of AR smart-glasses such as the Microsoft HoloLens, Google Glass, Apple Glass. In contrast to VR goggles, AR smart-glasses look similar to sunglasses or normal glasses, thus, causing no discomfort to the users. The second type of implementations is on AR apps such as Pokemon Go. In this type of implementation, smartphone cameras are used to track the surrounding environment and digital models are “superimposed into the real-world" \parencite{moro_effectiveness_2017}.

Using an AR-implemented-app does not only have the advantage of allowing the user to interact with both the real and virtual elements of their surrounding environment but also in this way, no extra equipment is required, AR app has become more and more common, especially in the field of education \parencite{moro_effectiveness_2017}.


\subsection{Augmented Reality in Education}
There have been multiple articles promoting learning through AR. The study from the University of Cologne discussed the benefits of AR in educational environments with a conclusion that applications applying successful use of AR have been improving learning, especially in language education, mechanical skills, and spatial abilities training \parencite{diegmann_benefits_2015}.The study from the University of Girona analysed 32 studies from journals about the AR trend in education. The finding results were that the number of published studies about AR in education has progressively increased year by year, while the fields of education which had the most AR applications are Science, Humanities, and Arts, in which AR has been effective for better learning performance, learning motivation, student engagement and positive attitudes because the advantage of AR is allowing interactions and collaboration \parencite{bacca_augmented_2015}. The study by the University of La Laguna also concluded in the higher performance of students studying using AR applications as they can do it in their own time \parencite{martin-gutierrez_augmented_2015}.There were also some specific proposals on how to integrate AR in learning. Brown and Gabbard proposed using AR to personalised learning for every student \parencite{brown_interactive_2015} while Huang and his team came up with a learning model based on AR, in which educational resources are discovered on the internet and be translated to AR for an educational environment to improve students’ emotions and experiences \parencite{huang_animating_2016}.

Molecules studying can be confusing. In 2019, a study conducted using AR for leaning atoms and molecules reaction by students was conducted. In this study, female students, who do not show much interest in science and technology were the target of research \parencite{ewais_usability_2019}. The study found that using AR technology to visualise the molecules did motivate these students to study, as it helps them understand the structure much better. Another study conducted using an augmented reality web application for high school education in biomolecular life science also discussed the fact that it is difficult to understand the spatial relationship of a protein structure. “Proteins and protein interactions are too small to be seen by far, even under advance microscopes" \parencite{nickels_proteinscanar_2012-1}. However, by using this AR web-based app, students show much more interest and understanding of the field \parencite{nickels_proteinscanar_2012-1}. Another study from Georgia Gwimmet College in 2018 about the developing of an AR app that transforms 2D molecular representations into interactive 3D structures that user can manipulate also showed that students who have used augmented reality models found it convenient and faster than the traditional mode, and they also prefer it more as they have control over molecular manipulations \parencite{behmke_augmented_2018}.

These studies showed above showed how important it is for AR to be brought into molecular biology study.

\section{Protein Structure}

In the field of biology, researchers give prioritised attention to the shape of a protein. Proteins are “the most important macromolecules in all living organisms” \parencite{rashid_protein_nodate}. Sequences of amino acids that bind into linear chains create proteins. These chains have a specific folded three-dimensional (3D) shape, which enables the protein to perform a certain task \parencite{rashid_protein_nodate}. The shape of the protein defines its tasks, thus, knowing the protein structure is very important. This task is defined by the shape of the protein, which makes the understanding of protein structure of great importance. There are four different levels of protein structures: Primary Structure, Secondary Structure, Tertiary Structure, and Quaternary Structure. A sequence of amino acids in a chain forms a \emph{Primary structure}. These chains, then, would fold into three different shapes (Helix, Coil or Sheet) where the alpha-helix, the beta-sheet, and the random-coils are positioned, which is called the \emph{secondary structure}. The combination of these chains of helix, coil and sheet (polypeptide chains) forms a 3D structure – the \emph{tertiary structure} of a protein \parencite{rashid_protein_nodate}. The \emph{quaternary structure} is a large assembly of multiple polypeptide chains (Figure \ref{fig:proteinstr}).
 \begin{figure}[!htp]
	\centering
	\includegraphics[scale=0.5]{images/proteinstr.png}
	\caption{Orders of protein structure - source \href{https://www.khanacademy.org/science/biology/macromolecules/proteins-and-amino-acids/a/orders-of-protein-structure}{Khan Academy}\parencite{noauthor_introduction_nodate}}
	\label{fig:proteinstr}
\end{figure}

Given the importance of protein structure, designing proteins is extremely useful as this can help to change its functions, or create new functions. In ProteinAR, users will get to design protein structure by combining different protein secondary structures of helices, coils, and sheets to form tertiary structures.

\section{Existing solutions to protein visualisation}
 “Proteins are three-dimensional (3D) objects” \parencite{ratamero_touching_2018}. Computer models for protein have become very popular for a long time. Many projects were developed to make 3D viewing of protein possible such as {\footnotesize PYMOL, CHIMERA, VMD, ISOLDE,} etc. 
 
 \subsection{Protein visualisation in mobile applications}

There are numerous mobile applications in which proteins are visualised in 3D. The RCSB Protein Data Bank (the single worldwide repository of protein data) also provides a \href{https://www.ncbi.nlm.nih.gov/pmc/articles/PMC4271143/}{mobile app} allowing data access and visualisation. The protein can be downloaded directly from the PDB from RCSB and displayed in 3D. This app is based on the open-source molecular viewer \href{https://play.google.com/store/apps/details?id=jp.sfjp.webglmol.NDKmol&hl=en}{NDKmol}. However, NDKmol can only be used on Android and not iOS. \href{https://www.imedicalapps.com/2013/08/jmol-molecular-visualization-app/}{Jmol}l is another Android app that connects to the RCSB PDB, and visualises proteins in 3D.
There are some molecule viewers available on iOS. Unfortunately, most of them are no longer in use or experienced technical difficulty, and have therefore been removed from the Apple App Store. \href{https://www.molsoft.com/iMolview.html}{iMolview} is still available, however, the interface is not very user friendly. 


\subsection{Protein Visualisation in VR}
\subsubsection{The advancement of implementing VR in Protein Display}
Visualisation of proteins on the computer has been a great step, however, it lacks the immersive salience of 3D presence, and leads to limitations in the analysis of protein structure. Virtual Reality (VR) provides a wide field of view on an immersive display and a better perception of the protein structure by head-tracking. Furthermore, VR enables users to have the freedom of hand controllers for simple manipulation and interaction with the protein instead of the conventional manipulation on 2D using a trackpad, mouse, and keyboard \parencite{goddard_molecular_2018}. This makes VR's entrance into the world of molecular biology and protein visualisation more than welcome. 
HMDs\footnote{Head Mounted Display} are commonly used because they are accessible, becoming increasingly more available, and are affordable. VR games have become popular, thus the tools for programming software that are compatible with HMD are effective and cheap. Projects such as {\footnotesize REALITYCONVERT}, {\footnotesize AUTODESK}, {\footnotesize MOLECULE VIEWER} are well developed, providing good resource for further development on protein display in VR \parencite{ratamero_touching_2018}. {\footnotesize UNITY} is largely used with the combination of HMDs such as {\footnotesize OCULUS RIFT} and {\footnotesize HTC VIVE} to display and manipulate proteins \parencite{ratamero_touching_2018}.

\begin{figure}[!htp]
	\centering
	\includegraphics[scale=0.6]{images/OculusRift.png}
	\caption{Oculus Rift (HMD) and Kinect v2 sensor placement used during Molecular Rift development}
	\label{fig:OculusRift}
\end{figure}


There have been many advanced VR projects in molecular biology. In particular,  {\footnotesize MOLECULAR RIFT} is an open source tool that creates a virtual reality environment steered with hand movements, and incorporates {\footnotesize OCULUS RIFT} as the display to create the virtual setting \parencite{norrby_molecular_2015}. The combination of a virtual reality experience with natural acts such as hand movement creates a much better experience for the users than merely experiencing the 3D \parencite{norrby_molecular_2015}.

Though research shows that the technology in displaying Protein in VR is advanced, the tools that need to be installed on desktop systems are often tedious \parencite{xu_vrmol_2019}. The configurations might be different for different systems and therefore cause compatibility issues. Sharing between system is also difficult. With the help of Web Graphics Library (WebGL), web-based applications such as {\footnotesize JMOL}, {\footnotesize ASTERVIEWER} are more straightforward as VR experiences can be directly accessed with common web browsers. However, there are many limitations for these web-based applications because they only support a few file types and cannot perform complex tasks for analytical purpose \parencite{xu_vrmol_2019}. A few solutions were proposed for an integrative cloud-based system that can directly access databases and uses VR technology to visualise and analyse macromolecular structures, such as {\footnotesize VRMOL}. This might be the new direction for protein visualisation in VR.

\subsubsection{The limitations of using VR in Displaying Protein}

Even though VR implementation of protein display has come far, limitations are inevitable. First, the limitations in the associated hardware/software may lead to an unsuccessful application of VR, which leads to the inaccuracy and impreciseness in the results of using the application. With the increasing development of VR techniques and the growing popularity of VR games, software and hardware are becoming more integrated with VR and therefore more compatible, however, they are still costly and need to be increased in fidelity \parencite{liu_using_2018}.
The second point concerns the unnatural feeling of using VR. Even though VR offers a realistic view, the users must wear goggles that are not transparent and thus block the vision of the real world. Furthermore, the head movements are unnatural because users will have to try to move their heads to see contents. New HMDs are better because they are much lighter but mostly VR devices are still quite bulky and are relatively difficult to use. Thirdly, most VR users claim to have motion sickness. This happens because of the disparity between what the body and the eyes of a user experiences at the same time. The actual physical actions and the actions that are carried out in VR might be different and this causes motion sickness to the users. Due to this, VR can only be used for a limited amount of time.


\subsection{Protein Visualisation in AR}

As AR gains popularity, more projects are underway, but this is limited as it is an extension of VR, and it is still very new. Some studies show that AR being used in science teaching such as displaying molecular biology in AR has yielded in good results for students, as it takes less imagination and makes things easier to understand \parencite{cai_case_2014}.However, there are not many AR apps available to support visualising molecules. 

As mentioned, there are not many projects concerning the visualisation of molecules on AR. Unlike VR, where there are various numbers of HDMs incorporated software and app for protein visualisation, on AR, apps are more commonly used. There are only a few apps that can be found. BiochemAR is one of those. One such app, BiochemAR, was released in 2019 and is available on both the App Store (for iOS) and Google Play (for android). According to the developers, the idea of the app is to create a simple, easy-to-use teaching tool for both teachers and students in the classroom\parencite{sung_biochemar_2020}. The main function is to display protein in AR by scanning a QR code, thus the design is relatively basic. When a QR code is scanned, the app will use the smart devices’ built-in camera to bring the protein structure into life through VR as shown in Figure \ref{fig:bioChemAR}.
\begin{figure}[!htbp]
	\centering
	\includegraphics[scale=0.5]{images/bioChemAR.png}
	\caption{BiochemAR app screen shot}
	\label{fig:bioChemAR}
\end{figure}

As the main purpose is to make things simple and easy to use for teachers and students, there is no other function or interaction between users and the protein. Proteins are simply visualised and users can move the phone around to look at the protein in different angles and
sizes. 

Having the same idea, another app called AR Assisted Visualisation was developed in 2020 to visualise proteins. These proteins are not written under QR code form but instead printed out on paper as in Figure \ref{fig:arvisualisation}. 

\begin{figure}[!htbp]
	\centering
	\includegraphics[scale=0.8]{images/arvisualisation.png}
	\caption{AR Assisted Visualisation App \parencite{eriksen_visualizing_2020}}
	\label{fig:arvisualisation}
\end{figure}

Similar with BioChemAR, AR Assisted Visualisation only display protein structure in 3D, without any interacting elements. 

\subsection{Personal insights}
Integrating protein visualisation on mobile apps is a good solution because of its availability. Most students have access to a smartphone and it is handy to bring around as it is not bulky nor need specific customisation. However, the AR apps on protein visualisation are relatively
new (released in 2019 and 2020). Thus, there are not many user interactions and functions to it. To use the aforementioned apps, a certain document with information of the protein, whether it is a figure of a protein or a QR code, has to be printed in order to get the AR visualisation.
Moreover, the proteins can be viewed but cannot be manipulated in any way. Furthermore, these apps are one-side oriented as users can only view proteins but cannot create new ones.

\section{Finding summary}
With the advantage of allowing the user to stay in touch with the surrounding environment and not requiring any extra equipment, the use of AR in mobile applications is certainly gaining popularity. In education, the use of AR technology motivates students and establish better performance. In molecular biology, according to many studies, students are much more interested and have a greater understanding when AR technology is used in teaching. In a recent research, it showed that when undergraduate students created their own AR-protein, they were enthusiastic when performing this function, thus, their learning was enhanced when the AR module was inserted to their upper-level biochemistry class \parencite{argu_fast_2020}. With the trend of online learning, the application of AR offers a promising curriculum for biochemistry. Not only with education, but the integration of AR in molecular biology can also benefit researchers as it makes small interactions which are invisible under microscope visible. Currently, many applications are being developed to visualise protein structures, however, there are still limitations. Nevertheless, in the near future, AR technology will become much more commonly used in molecular biology. 

ProteinAR's purpose is to not only let users directly view the shape of a protein in AR, or interact with the protein by gesture touch on the screen, but also allow users to design and create their proteins. The majority of mobile apps to visualise protein are only in 3D, and mostly on Android. Therefore, the open-source API for protein visualisation directly from the PDB files are limited. This project will have to start with little availability in pre-developed techniques.

Based on the above findings, this project aims to provide a valuable tool for the field of biology with the experimental app ProteinAR. With further development, it can be applied as an education tool, and for researchers in need of accessible and accurate protein models.






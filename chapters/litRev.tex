\chapter{Analysis: Review \& Research on the field and existing products}
\label{ch:litRev}

\section{Introduction to Protein}

As mentioned, proteins are “the most important macromolecules in all living organisms” (Rashid, 2015). Sequences of amino acids that bounds into linear chains create proteins. These chains have a specific folded three-dimensional (3D) shape, which enables the protein to perform a certain task (Rashid, 2015). The shape of the protein defines the tasks of it, thus, knowing the protein structure is very important. There are four different levels of protein structures: Primary Structure, Secondary Structure, Teritiary Structure, and Quaternary Structure. A sequence of amino acids in a chain form a \emph{primary structure}. These chains, then, would fold into three different shapes of Helix, Coil or Sheet where the alpha helix, the beta sheet, and the random coils are positioned, which is called the \emph{secondary structure}. The combinations of these formed chains of helix, coil and sheet (a polypeptide chain) would form a 3D structure – the \emph{tertiary structure} of a protein. The \emph{quaternary structure} is a large assembly of multiple polypeptide chains.
 \begin{figure}[!htp]
	\centering
	\includegraphics[scale=0.5]{images/proteinstr.png}
	\caption{Orders of protein structure - source \href{https://www.khanacademy.org/science/biology/macromolecules/proteins-and-amino-acids/a/orders-of-protein-structure}{Khan Academy}}
	\label{fig:proteinstr}
\end{figure}

To understand a protein’s function, understanding the structure of the protein is necessary. In the same way, designing a protein from the structure will help to design its function. 
In this app, users will get to design protein structure by combining different protein secondary structures of helices, coils, and sheets to form tertiary structures. 


\section{Existing Solutions to Protein visualisation}
 “Proteins are three-dimensional (3D) objects” (Ratamero et al., 2018). The key to understanding protein functions is to understanding protein structure. Computer models for protein has become very popular for a long time. Many projects were developed to make 3D viewing of protein possible such as PYMOL, CHIMERA, VMD, ISOLDE, etc. 
 
 \subsection{Protein Visualisation in Mobile Applications}

There are numerous of mobile applications in which protein are visualised in 3D. The RCSB Protein Data Bank (the single worldwide repository of protein data) also provide a \href{https://www.ncbi.nlm.nih.gov/pmc/articles/PMC4271143/}{mobile app}to provide data access and visualization. Basically, the protein can be downloaded directly from the PDB from RCSB and displayed in 3D. This app is based on the open-source molecular viewer \href{https://play.google.com/store/apps/details?id=jp.sfjp.webglmol.NDKmol&hl=en}{NDKmol}. However, NDKmol can only be used on Android and not iOS. \href{https://www.imedicalapps.com/2013/08/jmol-molecular-visualization-app/}{Jmol}l is another Andriod app that connects to the RCSB PDB, visualising the protein in 3D once the protein name is typed in.
There are some molecule Viewers that can run on iOS devices. Unfortunately, most of them are no longer in used or was having troubled, thus, being removed from the Apple App Store. \href{https://www.molsoft.com/iMolview.html}{iMolview} can still be used, however, the interface is not very user friendly. 


\subsection{Protein Visualisation in VR}
\subsubsection{The advancement of implementing VR in Protein Display}
Visualizing protein on computer in 3D has been a great step, however, it still lacked the immersion and a true feeling of 3D presence, leads to limitation in analyzing protein structure. Virtual Reality (VR) provides a wide field of view on an immersive display and a better perception of the protein structure by head-tracking. Furthermore, VR enables user to have the freedom of hand controllers for simple manipulation and interaction with the protein instead of the conventional manipulation on 2D using trackpad, mouse and keyboard (Goddard et al., 2018). This makes VR entrance into the world of protein visualizing/molecular biology more than welcomed. 
HMDs\footnote{Head Mounted Display} are commonly used because they are easy to use and becoming more and more common and affordable. VR games have become popular, thus the tools for programming software that are compatible with HMD are better and cheaper. Project such as {\footnotesize REALITYCONVERT}, {\footnotesize AUTODESK}, {\footnotesize MOLECULE VIEWER} are well developed, providing good resource for further development on protein display in VR (Ratamero et al., 2018). {\footnotesize UNITY} is largely used with the combination of HMDs such as {\footnotesize OCULUS RIFT} and {\footnotesize HTC VIVE} to display and manipulate protein (Ratamero et al., 2018).

\begin{figure}[!htp]
	\centering
	\includegraphics[scale=0.6]{images/OculusRift.png}
	\caption{Oculus Rift (HMD) and Kinect v2 sensor placement used during Molecular Rift development}
	\label{fig:OculusRift}
\end{figure}


There have been many advanced projects of implementing VR in molecular biology. The {\footnotesize MOLECULAR RIFT} – an open source tool that creates a virtual reality environment steered with hand movements, incorporate {\footnotesize OCULOS RIFT} as the display to create the virtual setting (Norrby et al., 2015). The combination of a virtual reality experience with natural acts such as hand movements creates a much better experience for the users than just experiencing the 3D (Norrby et al., 2015).

Other research shows that the technology in displaying Protein in VR is advanced, however, tools that are designed to be installed on desktop systems are often tedious (Xu, 2019).  The configurations might be different with systems and therefore causing errors and sharing between system is difficult. With the help of Web Graphics Library (WebGL), web-based applications such as {\footnotesize JMOL} , {\footnotesize ASTERVIEWER} are more straightforward as VR experiences can be directly accessed with common web browsers. However, there are many limitations for these web-based applications because it would only support a few file types and cannot perform complex tasks for analytical purpose (Xu et al., 2019). A few solution of an integrative cloud-based system that can directly access databases and uses VR technology to visualise and analyse macromolecular structures were proposed, such as {\footnotesize VRMOL}. This might be the new direction for protein visualising in VR.

\subsubsection{The limitations of using VR in Displaying Protein}

Even though the VR implementation in displaying protein has come far and will still go further in the future, there are still some inevitable limitations. 
First, the limitations in the associated hardware, software may lead to an unsuccessful application of VR, which leads to the inaccuracy and imprecise in the results of using the application. With the increasing development of VR techniques and the popularity that VR games are gaining, software and hardware to be integrated with VR are becoming more compatible, but not without limitations. They are still costly and need to be increased in fidelity \parencite{liu_using_2018}
Secondly, it is the unnatural feeling of using VR. Even though VR offers a realistic view, the users have to be wearing goggles which are not transparent and thus, blocking the vision of the real world. Furthermore, the head movements are unnatural because users will have to try to move their heads in order to see things. New HMDs are better because they are much lighter but mostly VR devices are still quite bulky and not that easy to use yet. 
Thirdly, most VR users claim to have motion sickness. This happens because of the difference the bodies and the eyes experience at the same times. The actual physical actions and the actions that are carried out in VR might be different and therefore, causing motion sickness to the users. Due to this problem, when using VR, users cannot use it for a long time. 


\subsection{Protein Visualisation in AR}

Similar with Virtual Reality, Augmented Reality (AR) generate the realism by displaying the 3D models in a real-world context. However, unlike VR where the whole vision of the users is taken away and replaced by another completely different scene, AR’s defined characteristic is that it added a layer onto the vision. While VR creates an immersive experience for users by shutting out the real physical world, AR maintains the realism of the world, allowing users to see whatever they are seeing plus more. With AR, the users have free movements while projecting images. Commonly speaking, there are the most two well-known types of AR technology implementation. The first one is implemented on AR smart-glasses such as the Microsoft HoloLens, Google Glass, Apple Glass. Contrast to VR goggles, AR glasses looks just like sunglasses or even normal glasses, thus, causing no bulky feelings to the users. 
The second type of implementations are on AR apps such as Pokemon Go. In this type of implementation, camera’s phones are used to track the surroundings environment as well as adding a layer on top of the screen to show external information.

As AR is gaining popularity, more projects are being done but not much as it is an extension of VR, it is still very new. Some studies show that AR being used in science such as molecular displayed has yielded in good results for students, as it takes less imagination and makes things more easy to understand (Cai et al., 2014). However, there are not many AR apps available to support education, specifically in visualizing molecules. 

As mentioned, there are not many projects done on visualization molecules on AR. Unlike VR, where there are a various number of HDMs incorporated software and app for protein visualization, on AR, apps are more commonly used. There are only a few apps that can be found. BiochemAR is one of those. BiochemAR was released in 2019 and are available on both App store (for iOS) and Google Play (for android). According to the developers, Sung and her team, the idea of the app is to create a simple, easy-to-use teaching tools for both teachers and students in the class room (Sung et al., 2020). The main function is to display protein in AR by scanning a QR code, thus the design is very basic. When a QR code is scanned, the app will use the phone/ tablet/ smart devices’ built-in camera to bring the protein structure into life through VR as shown in Figure \ref{fig:bioChemAR}.
\begin{figure}[!htbp]
	\centering
	\includegraphics[scale=0.5]{images/bioChemAR.png}
	\caption{BiochemAR app landing screen}
	\label{fig:bioChemAR}
\end{figure}





As the main purpose is to make things simple and easy to use for teachers and students, there is no other functions as well as interactions between users and the protein. Those are just visualized and users can move the phone around to look at the protein in different angles and size. 

Having the same idea, another app called {\footnotesize AR ASSISTED VISUALISATION}
was developed in 2020 to visualize protein. These proteins are not written under QR code form but instead printed out on paper as in Figure \ref{fig:arvisualisation}. 

\begin{figure}[!htbp]
	\centering
	\includegraphics[scale=0.8]{images/arvisualisation.png}
	\caption{AR Assisted Visualisation App (Eriksen et al., 2020)}
	\label{fig:arvisualisation}
\end{figure}

Similar with BioChemAR, AR Assisted Visualisation only display protein structure in 3D, without any interacting elements. 

\subsection{Finding summary}
In a recent research on American Chemical Society and Division of Chemical Education, it seems that when undergraduate students get to create their own AR protein visualisation, they were enthusiastic in doing so and thus, their learning was enhanced when the AR module was inserted to their upper level biochemistry class (Arguello \& Dempski, 2020). With the trend of online learning, the application of AR would promise a better curriculum for biochemistry.

Integrating protein visualisation on Mobile Apps is a good solution because of its availability. Most students have access to a mobile phone and it is handy  to bring around as it is not bulky or need specific customisation. 

The AR apps on protein visualisation are still new and young (released in 2019 and 2020). Thus, there is not much user interactions and functions to it. To use these apps mentioned in this thesis, a certain document with information of the protein, whether it be a figure of a protein or a QR code has to printed in order to get the AR visualisation. Moreover, the proteins can be viewed but cannot be interacted with in anyway. Furthermore, these apps are one-side oriented as users can only view protein but cannot create any. 

This project’s application: ProteinAR’s purpose is to not only let users directly view the shape of protein in AR, interact with the protein by gesture touch on the screen but also allow users to design and create their own proteins. 
The majority of mobile apps to visualise protein are only in 3D, and most of the times on Android. Therefore, the open-source API for protein visualisation directly from the PDB files are limited. This project will have to start from little availability in pre-developed techniques.

With further work being put into, ProteinAR can be applied to be used in teaching to make lessons more interesting and understandable for students as well as motivate students to do higher level in Biochemistry. 





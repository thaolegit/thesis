% !TEX TS–program = pdflatexmk

\documentclass[MSCIM]{mscim}
\usepackage{graphicx}
\usepackage[english]{babel}
\usepackage{csquotes}
\usepackage[style=apa,doi=false,isbn=false,url=true,eprint=false]{biblatex}
\usepackage{hyperref}
\usepackage[utf8]{inputenc}
\usepackage{float}
\usepackage{array}
\usepackage{tabularx}
\usepackage{appendix}
%\usepackage{apacite}

\addbibresource{refs.bib}

\begin{document}
\DeclareGraphicsExtensions{.pdf,.png,.gif,.jpg}

\title{\textbf{ProteinAR} \\an interactive ios application\\for protein visualisation and design in augmented reality}

\author{Thao Phuong Le}
\principaladviser{Sabin Tabirca}

\beforeabstract

\prefacesection{Abstract}
Proteins are complex molecules with various functions that are critical to any living organism which comes in all different shapes and sizes. The shape of a protein defines its function, therefore, the study of protein structure is of great importance since it can help scientists to control or modify the protein to change its function and help with various disease treatments. 

In this project, ProteinAR - an iOS application was developed with the goal of aiding the field of biology by visualising protein structures and enabling interactions in Augmented Reality. While past projects looked to non-interactive digital environments for protein visualisation, with ProteinAR, users can observe on their smartphone screens simple to complex protein structures by zooming in, rotating or moving them. Users are also able to create new interactive proteins. ProteinAR was written in Swift on Xcode, a native language and IDE developed by Apple for iOS apps, and used the integrated framework ARKit for Augmented Reality functions. When a user inputs a protein ID, ProteinAR downloads protein data and displays it in real-time using RCSB PDB as the main source of protein data. Product evaluation concludes that, with further development, ProteinAR demonstrates potential for the application of augmented reality technology in protein visualisation and design, particularly for researchers, educators, and students of biology.

\afterabstract

\prefacesection{Acknowledgements}


 
 
\afterpreface

\chapter{Introduction}
\label{ch:intro}

In recent years, there have been major advances in technology and molecular biology respectively. Technology has become a great help for biologists aiding their research and make contents easier to study. This project focuses on protein structure displaying and protein structure design.

A protein is not a single substance. There are many different proteins in an organism or in a cell, and they come in every shape and size, performing a unique and specific job \parencite{noauthor_introduction_nodate}. Proteins are considered the ``ultimate players in the processes that allow an organism to function and reproduce'' \parencite{stephenson_protein_2016}.

Proteins are formed by linear chains of amino acids, called a polypeptide. Each protein is formed by one or more polypeptide chains, linked together in a specific order \parencite{noauthor_introduction_nodate}. Protein are the fundamental components of all living cells \parencite{hutchison_protein_2013}. Protein have a countless number of functions that are extremely important in the biology of many organisms. They form enzymes to speed the reactions up by break-down, link-up, or rearranging the substrates \parencite{noauthor_introduction_nodate}. They form hormones to control specific physiological processes such as ``growth, development, metabolism and reproduction'' \parencite{noauthor_introduction_nodate}. To maintain these roles, the shape of a protein is critical. If the shape changes, the protein will lose its functionality. There are four levels of protein structure: primary, secondary, tertiary, and quaternary \parencite{noauthor_introduction_nodate}.
Knowing the structure of a protein makes understanding how that protein works much easier. By being able to manipulate the structure of a protein, scientists can create hypotheses about how to affect, how to control or how to modify protein to design mutations and change a protein's function. 

The year 2020 substantiates the importance of studies in molecular biology. We all have experienced the Severe Acute Respiratory Syndrome Coronavirus-2 (SARS-CoV-2) as is it ``a newly emerging, highly transmissible and pathogenic coronavirus in humans that cause the global public health emergencies and economic crises'' \parencite{mittal_covid-19_2020}. As of the present time of this project, the number of infections worldwide has reached millions, including thousands of deaths. To find a cure, much research has been conducted. Some research developed on the protein structure of SARS-CoV-2 has provided insight into its evolution. As Wiesława has pointed out: ``The chief characteristic of proteins that allows their diverse set of functions is their ability to bind other molecules (proteins or small-molecule substrates) specifically and tightly.'' \parencite{hutchison_protein_2013}. Particular to SARS-CoV-2 are the protein spikes that. The virus uses these to bind with and enter human cells \parencite{wrobel_sars-cov-2_2020}. The spikes of SARS-CoV-2 are highly stable and thus help to bind to human cells tightly. Therefore, analysing the structure of these spikes could provide clues about the virus’s evolution. The study of the structure of spike proteins can aid with drug discovery and vaccine design. 
Understanding the new importance of implementing IT in biology research, \textbf{this project aims to aid with protein structural study and raise interest in protein design.}

Due to the shortage of time and lack in experience, this project only provides the first step into bringing the visualisation of protein into AR-display and creating a simple protein structure in an iOS application using the framework ARKit. \textbf{The main goal of this project, however, is to visualise protein structure on AR using an iOS App and let user interact with the structures.} 
There are various previous studies on protein visualisation on 3D and VR, however, studies pertaining to AR is limited, especially the AR app on iOS. This project proposed the implementation of displaying protein structures to serve as a trial for future study and research as it might make displaying more appealing than simple 3D, and also cut down on the side effects of VR. All the protein models that are to be displayed are retrieved from \href{https://www.rcsb.org/}{RCSB Protein Data Bank}. 

This project's app aims to visualise protein structures in two ways. The first way is directly displaying the complex protein models from RCSB PDB, and the second way is visualising the design of a simple protein structure. The second function is implemented so that this project's application is not only appealing to biologists but also can be used by anyone curious about biology. Being able to construct a protein structure as a mini-game might make it easier for users to understand more about protein structure. 

In this project, a mobile application for iOS system was developed: \underline{\textbf{ProteinAR}}. This app has two main categories: \textbf{education} and \textbf{mini-game}. 

The \textbf{education} category assumes that the users have previous knowledge of proteins. They can input the name of protein and get the 3D visualisation of the protein structure in AR. Users can study the protein by zooming in, turning, and flipping the protein structure. Due to the complexity of protein structures, it is not easily observed even under advanced microscopes. Thus, the ability to zoom in and all other interactions can benefit researchers. Moreover, since this is on a mobile app, users can interact and discuss the structure with other users at the same time, which can be considered a promising tool for study and research on proteins.

The \textbf{mini-game}, is user-friendly to those who are unfamiliar with proteins or biology in general. Users can add the polypeptide chains onto each other (i.e. Flex Coil, Rig Coil, Helix, Sheet) to create a protein. This might make the concept of proteins sound more appealing to users and thus, motivate the wish to study more about protein. In this game, users are also able to interact with the polypeptide chains and protein models. 

Moreover, ProteinAR integrates other functions to make the app more interesting, such as enabling photo-capture of the proteins, video-capture of the process, and providing users with addition information about protein.

This paper will elaborate on the background and research, the problems and solutions, the design and implementation, and the final evaluation of the project. In the short period of time and the given circumstance of Covid-19, there were some limitations to the project, which would also be mentioned in the paper. 

Finally, the paper will discuss some critical points in dealing with the fairly new AR technology, especially using ARKit framework on, concerning:
\begin{itemize}
\item The feasibility of retrieving and displaying PDB contents
\item  The usability of the app (AR)
\item Room for future work
\end{itemize}

Some important technical notes about the project:

\textbf{ProteinAR} was designed on Xcode 12, written in Swift 5, on MacBook OS version: Catalina 10.5.5. There is no support for AR on MacOS, thus, the built-in simulator will not be able to display the AR function and can cause some other errors. The project was run and tested on an iPhone. The attached demo video is recorded on iPhone X, iOS version 14. Other versions of Xcode or macOS or iOS might be unable to run ProteinAR and thus might generate some unwanted errors. 


\chapter{Analysis: Review \& Research on the field and existing products}
\label{ch:litRev}

\section{Introduction to Protein}

As mentioned, proteins are “the most important macromolecules in all living organisms” \parencite{rashid_protein_nodate}. Sequences of amino acids which bind into linear chains create proteins. These chains have a specific folded three-dimensional (3D) shape, which enables the protein to perform a certain task \parencite{rashid_protein_nodate}.The shape of the protein defines its tasks, thus, knowing the protein structure is very important. There are four different levels of protein structures: Primary Structure, Secondary Structure, Tertiary Structure, and Quaternary Structure. A sequence of amino acids in a chain forms a \emph{Primary structure}. These chains, then, would fold into three different shapes (Helix, Coil or Sheet) where the alpha helix, the beta sheet, and the random coils are positioned, which is called the \emph{secondary structure}. The combination of these chains of helix, coil and sheet (polypeptide chains) forms a 3D structure – the \emph{tertiary structure} of a protein. The \emph{quaternary structure} is a large assembly of multiple polypeptide chains (Figure \ref{fig:proteinstr}).
 \begin{figure}[!htp]
	\centering
	\includegraphics[scale=0.5]{images/proteinstr.png}
	\caption{Orders of protein structure - source \href{https://www.khanacademy.org/science/biology/macromolecules/proteins-and-amino-acids/a/orders-of-protein-structure}{Khan Academy}\parencite{noauthor_introduction_nodate}}
	\label{fig:proteinstr}
\end{figure}

To understand a protein’s function, understanding the structure of the protein is necessary. In the same way, designing a protein from the structure will help to design its function. 
In ProteinAR, users will get to design protein structure by combining different protein secondary structures of helices, coils, and sheets to form tertiary structures. 


\section{Existing solutions to protein visualisation}
 “Proteins are three-dimensional (3D) objects” \parencite{ratamero_touching_2018}. The key to understand protein functions is to understanding protein structure. Computer models for protein have become very popular for a long time. Many projects were developed to make 3D viewing of protein possible such as {\footnotesize PYMOL, CHIMERA, VMD, ISOLDE,} etc. 
 
 \subsection{Protein visualisation in mobile applications}

There are numerous mobile applications in which proteins are visualised in 3D. The RCSB Protein Data Bank (the single worldwide repository of protein data) also provides a \href{https://www.ncbi.nlm.nih.gov/pmc/articles/PMC4271143/}{mobile app} allowing data access and visualisation. Basically, the protein can be downloaded directly from the PDB from RCSB and displayed in 3D. This app is based on the open-source molecular viewer \href{https://play.google.com/store/apps/details?id=jp.sfjp.webglmol.NDKmol&hl=en}{NDKmol}. However, NDKmol can only be used on Android and not iOS. \href{https://www.imedicalapps.com/2013/08/jmol-molecular-visualization-app/}{Jmol}l is another Android app that connects to the RCSB PDB, visualising the protein in 3D once the protein name is inputted.
There are some molecule viewers that can run on iOS devices. Unfortunately, most of them are no longer in use or experienced technical difficulty, thus, are removed from the Apple App Store. \href{https://www.molsoft.com/iMolview.html}{iMolview} can still be used, however, the interface is not very user friendly. 


\subsection{Protein Visualisation in VR}
\subsubsection{The advancement of implementing VR in Protein Display}
Visualising proteins on computer in 3D has been a great step, however, it lacks thethe immersive salience of 3D presence, and leads to limitation in analysing protein structure. Virtual Reality (VR) provides a wide field of view on an immersive display and a better perception of the protein structure by head-tracking. Furthermore, VR enables users to have the freedom of hand controllers for simple manipulation and interaction with the protein instead of the conventional manipulation on 2D using a trackpad, mouse and keyboard \parencite{goddard_molecular_2018}. This makes VR entrance into the world of protein visualising/molecular biology more than welcomed. 
HMDs\footnote{Head Mounted Display} are commonly used because they are accessible, increasingly more common, and are affordable. VR games have become popular, thus the tools for programming software that are compatible with HMD are effective and cheaper. Projects such as {\footnotesize REALITYCONVERT}, {\footnotesize AUTODESK}, {\footnotesize MOLECULE VIEWER} are well developed, providing good resource for further development on protein display in VR \parencite{ratamero_touching_2018}. {\footnotesize UNITY} is largely used with the combination of HMDs such as {\footnotesize OCULUS RIFT} and {\footnotesize HTC VIVE} to display and manipulate proteins \parencite{ratamero_touching_2018}.

\begin{figure}[!htp]
	\centering
	\includegraphics[scale=0.6]{images/OculusRift.png}
	\caption{Oculus Rift (HMD) and Kinect v2 sensor placement used during Molecular Rift development}
	\label{fig:OculusRift}
\end{figure}


There have been many advanced projects of implementing VR in molecular biology. In particular,  {\footnotesize MOLECULAR RIFT} is an open source tool that creates a virtual reality environment steered with hand movements, incorporates {\footnotesize OCULUS RIFT} as the display to create the virtual setting \parencite{norrby_molecular_2015}. The combination of a virtual reality experience with natural acts such as hand movements creates a much better experience for the users than merely experiencing the 3D \parencite{norrby_molecular_2015}.

Other research shows that the technology in displaying Protein in VR is advanced, however, tools that are designed to be installed on desktop systems are often tedious \parencite{xu_vrmol_2019}.  The configurations might be different with systems and therefore cause errors. Sharing between system is also difficult. With the help of Web Graphics Library (WebGL), web-based applications such as {\footnotesize JMOL}, {\footnotesize ASTERVIEWER} are more straightforward as VR experiences can be directly accessed with common web browsers. However, there are many limitations for these web-based applications because they only support a few file types and cannot perform complex tasks for analytical purpose \parencite{xu_vrmol_2019}. A few solutions were proposed for an integrative cloud-based system that can directly access databases and uses VR technology to visualise and analyse macromolecular structures, such as {\footnotesize VRMOL}. This might be the new direction for protein visualising in VR.

\subsubsection{The limitations of using VR in Displaying Protein}

Even though the VR implementation in displaying protein has come far, limitations are inevitable.
First, the limitations in the associated hardware/software may lead to an unsuccessful application of VR, which leads to the inaccuracy and impreciseness in the results of using the application. With the increasing development of VR techniques and the gaining popularity of VR games, software and hardware to be integrated with VR are becoming more compatible, but not without limitations. They are still costly and need to be increased in fidelity \parencite{liu_using_2018}.
The second point concerns the unnatural feeling of using VR. Even though VR offers a realistic view, the users must wear goggles which are not transparent and thus block the vision of the real world. Furthermore, the head movements are unnatural because users will have to try to move their heads in order to see contents. New HMDs are better because they are much lighter but mostly VR devices are still quite bulky and are relatively difficult to use. 
Thirdly, most VR users claim to have motion sickness. This happens because of the disparity between what the body and the eyes of a user experience at the same time. The actual physical actions and the actions that are carried out in VR might be different and this cause motion sickness to the users. Due to this, VR can only be used for a limited amount of time.


\subsection{Protein Visualisation in AR}

Similar to Virtual Reality, Augmented Reality (AR) generates realism by displaying 3D models in a real-world context. However, unlike VR where the whole vision of the users is taken away and replaced by another completely different scene, AR’s defining characteristic is that it added a layer onto the vision. While VR creates an immersive experience for users by shutting out the real physical world, AR maintains the realism of the world, allowing users to see whatever they are seeing plus more. With AR, the users have free movements while projecting images. Commonly speaking, there are two well-known types of AR technology implementation. The first one is implementation on AR smart-glasses such as the Microsoft HoloLens, Google Glass, Apple Glass. Contrast to VR goggles, AR smart-glasses look similar to sunglasses or normal glasses, thus, causing no discomfort to the users. 
The second type of implementations are on AR apps such as Pokemon Go. In this type of implementation, smartphone cameras are used to track the surrounding environment as well as adding a layer on top of the screen to show external information.

As AR gains popularity, more projects are underway, but this is limited as it is an extension of VR, and it is still very new. Some studies show that AR being used in science teaching such as displaying molecular biology in AR has yielded in good results for students, as it takes less imagination and makes things easier to understand \parencite{cai_case_2014}. However, there are not many AR apps available to support education, specifically in visualising molecules. 

As mentioned, there are not many projects concerning the visualisation of molecules on AR. Unlike VR, where there are various numbers of HDMs incorporated software and app for protein visualisation, on AR, apps are more commonly used. There are only a few apps that can be found. BiochemAR is one of those. Once such app, BiochemAR, was released in 2019 and is available on both App store (for iOS) and Google Play (for android). According to the developers, the idea of the app is to create a simple, easy-to-use teaching tools for both teachers and students in the classroom \parencite{sung_biochemar_2020}. The main function is to display protein in AR by scanning a QR code, thus the design is relatively basic. When a QR code is scanned, the app will use the smart devices’ built-in camera to bring the protein structure into life through VR as shown in Figure \ref{fig:bioChemAR}.
\begin{figure}[!htbp]
	\centering
	\includegraphics[scale=0.5]{images/bioChemAR.png}
	\caption{BiochemAR app screen shot}
	\label{fig:bioChemAR}
\end{figure}



As the main purpose is to make things simple and easy to use for teachers and students, there is no other function or interactions between users and the protein. Proteins are simply visualised and users can move the phone around to look at the protein in different angles and size. 

Having the same idea, another app called  AR Assited Visualisation
was developed in 2020 to visualise proteins. These proteins are not written under QR code form but instead printed out on paper as in Figure \ref{fig:arvisualisation}. 

\begin{figure}[!htbp]
	\centering
	\includegraphics[scale=0.8]{images/arvisualisation.png}
	\caption{AR Assisted Visualisation App \parencite{eriksen_visualizing_2020}}
	\label{fig:arvisualisation}
\end{figure}

Similar with BioChemAR, AR Assisted Visualisation only display protein structure in 3D, without any interacting elements. 

\subsection{Finding summary}
In a recent research conducted by American Chemical Society and Division of Chemical Education, it seems that when undergraduate students created their own AR protein visualisation, they were enthusiastic when performing this function, thus, their learning was enhanced when the AR module was inserted to their upper level biochemistry class \parencite{argu_fast_2020}. With the trend of online learning, the application of AR offers a promising curriculum for biochemistry.

Integrating protein visualisation on mobile apps is a good solution because of its availability. Most students have access to a smart phone and it is handy  to bring around as it is not bulky nor need specific customisation. 

The AR apps on protein visualisation are relatively new (released in 2019 and 2020). Thus, there is not much user interactions and functions to it. To use the aforementioned apps, a certain document with information of the protein, whether it be a figure of a protein or a QR code, has to printed in order to get the AR visualisation. Moreover, the proteins can be viewed but cannot be interacted with in any way. Furthermore, these apps are one-side oriented as users can only view proteins but cannot create them. 

ProteinAR’s purpose is to not only let users directly view the shape of protein in AR, or interact with the protein by gesture touch on the screen, but also allow users to design and create their own proteins.
The majority of mobile apps to visualise protein are only in 3D, and mostly on Android. Therefore, the open-source API for protein visualisation directly from the PDB files are limited. This project will have to start from little availability in pre-developed techniques.

This project creates a base for ProteinAR. With further future work, it can be applied to be used in teaching to make lessons more interesting and understandable for students as well as motivate students to do higher level in Biochemistry. 





\chapter{Methodology}
\label{ch:methodology}

This chapter introduces the software, language, and framework that were used to develop ProteinAR.

\section{Software}
	\subsection{Xcode}
Xcode is an integrated development environment (IDE) for MacOS. It was first released in 2003, and enables developers to create apps for Apple platforms. Xcode supports sources codes for various programming languages including C, C++, Objective-C, Swift, etc. Xcode has a built-in \emph{Interface Builder} to construct graphical interfaces. 
During the project's development process, Xcode has had a few version updates. The latest update was Xcode version 12. 
		\subsubsection{Advantages of using Xcode}
ProteinAR is written in Swift, a native language for iOS apps, released by Apple. Since Xcode is the native IDE of Apple, the compatibility is ideal, making the app and tests run faster and less error prone. Xcode is a highly intuitive IDE with a main storyboard interface, visualising the designs elements of an app as well as various built-in functions to customise the design, from background colours to framing and a built-in library for easy adding, and changing elements such as icons, pictures, text labels, and more \parencite{noauthor_xcode_nodate}.

		\subsubsection{Disadvantages of using Xcode}
ProteinAR uses the built-in ARKit framework. As this requires camera accessibility, tests cannot be run on the built-in iPhone simulators but instead, a real iPhone device. This creates a significant disadvantage due to persistent iOS updates which can cause compatibility issues between iOS and Xcode. Moreover, in some updates, the supporting packages change, causing compatibility issues that need to resolved. 

	\subsection{UCSF Chimera}
UCSF Chimera (or Chimera) is developed by the University of California. This program allows interactive visualisation of protein data. Once a PDB file is downloaded, Chimera can open the files in a 3D form and allow users to export the files in various types such as \emph{.dae}, \emph{.x3d}, or \emph{.obj}.
	
	\section{Swift}
Swift is a powerful programming language for Apple platforms. Apple released Swift in 2014, taking ideas from various other languages (Rust, Haskell, Ruby, Python, C, etc.,), but it bares most similarities to Objective-C  \parencite{noauthor_swift_nodate}. 
	\subsubsection{Advantages of using Swift}
Swift has been considered one of the most loved programming languages on Stack Overflow for many years as it is highly interactive, with concise and expressive syntax which runs fast. There are several improvements compared to other languages: there is no need for semi-colons, UTF-8 based encoding is used, strings are formatted in unicode, etc. It is also designed for safety as by default, Swift objects can never be \emph{nil}. As a successor to C and Objective-C, Swift includes low-level primitives such as types, flow control and operators as well as object-oriented features such as classes, protocols, and generics \parencite{noauthor_swift_nodate}. Overall, Swift is a simple and to-the-point coding language.
	\subsubsection{Disadvantages of using Swift}
As mentioned above, there were a few version updates of Xcode during the programming process. Swift being a relatively new language is in a state of regular fluctuation meaning changes to sytax and package names are relatively common. An additional difficulty in coding in such a young language is that problems might be too new to have a solution, which is the most challenging aspect of Swift.

\section{ARKit API}
The technology to develop Augmented Reality was ready for mobile devices, however, algorithms for detecting objects in real world and displaying virtual objects are highly complex. This is why Apple released ARKit in 2017 as a software framework, making developing an AR iOS app significantly easier. It is an API that supplies numerous and powerful features to handle the process of building Augmented Reality apps and games for iOS devices. 

Apple has been acquiring many AR companies, thus, the ARKit is built on all of these acquisitions. One of the major ones was the German company Metaio, which IKEA initially used to let customers display IKEA furniture in their own homes. Ferrari also used Metaio’s technology to allow customer changing colours of cars in showrooms, and view a car’s internal features. In 2017, Apple acquired SensoMotoric Instrument, a company specialized in eye tracking technology to use in AR. Other companies that specialized in other parts of AR technology are being acquired by Apple throughout the year. By doing this, the features of ARKit on iOS devices are frequently newly added and updated. ARKit is continuing to grow, making the creation of AR apps easier than ever \parencite{wang_beginning_2018}.

\subsection{Basic understanding of the ARKit}
There are three layers that work simultaneously in ARKit \parencite{noauthor_introduction_nodate-1} as shown in Figure \ref{fig:3Layers}.
\begin{figure}[!htp]
	\centering
	\includegraphics[width=\textwidth]{images/3Layers.jpeg}
	\caption{Three Layers to ARKit}
	\label{fig:3Layers}
\end{figure}

\textbf{Tracking} is the key function of ARKit. Without ARKit, it would be very complex for developers to write algorithms to track a device’s position, location, and orientation in the real world.
\textbf{Scene Understanding} is the layer that allows ARKit to analyse the environment presented by the camera’s view to adjust and provide information in order to place a virtual object in it. 
\textbf{Rendering} is the process where ARKIt handles the 3D models to put them in a scene such as SceneKit, Metal, RealityKit.

\subsection{Language and System Requirements for ARKit}
As previously mentioned, since augmented reality requires access to a high resolution display and camera, ARKit apps can only run on the following iOS devices: 
\begin{itemize}
	\item iPhone SE, iPhone 6s and later
	\item iPad 2017 and later
	\item all iPad Pro models
\end{itemize}
To develop an iOS app, Xcode is the best IDE to use as it also has the built-in simulator program to mimic different iPhone and iPad models. However, with ARKit integrated, the app cannot be tested on the simulators but instead has to be tested on a real iOS device from the list above via USB connection.
Both Swift and Objective-C can be used to create an ARKit app. This project chose Swift as the language for its ease of use and speed.
The ARKit framework allows developers to be able to focus on the features of the app rather than on the AR required technologies such as detecting, displaying and tracking virtual object in the real world. 


\section{RCSB Protein Data Bank}
ProteinAR downloads PDB files directly from \href{https://www.rcsb.org/}{RCSB}. PDB (Protein Data Bank) file format provides a standard representation for macromolecular structure data. These are obtained from X-ray diffraction and NMR studies \parencite{noauthor_rcsb_nodate}.
RCSB was the first open access digital data resource for Protein Data Bank. It provides access to 3D structural data for all biological molecules. RCSB is a global archive where PDB data are available for free \parencite{noauthor_rcsb_nodate}. The data acquired on RCSB are data submitted by biologists and biochemists around the world. On the \href{https://www.rcsb.org/}{website}, users can search for any protein name and the 3D structure will be displayed and allow interaction. Information about the protein will also be displayed, and PDB files can be simply downloaded.
During the process of making ProteinAR, some other sources for protein data were used including \href{ https://zhanglab.ccmb.med.umich.edu/I-TASSER/}I-TASSER and \href{https://web.expasy.org/protparam/} {ProtParam}. 
ProtParam is a basic website with only string type of data, allowing the GET method to get information from the server to the app easily. However, the PDB files containing 3D structural information of the protein were not available, so it was used as a test to discover if the POST and GET method functioned correctly in the app.
I-TASSER predicts protein structure and function after users enter the sequence of amino acids. Similar to RCSB, I-TASSER allows free downloading of PDB files where the structure of protein is already created in 3D and can be opened using UCSF Chimera. The cons of using I-TASSER is that the data cannot be downloaded in real-time because users need to enter their emails into the server and receive the PDB files a few hours later. 
As the goal of ProteinAR is to visualise protein structures and display them instantly, RCSB was chosen for the database as it fits said goal.  

\section{Summary}
ProteinAR is an iOS app. It was written in Swift 5 using Xcode, using ARKit API. The dataset in which protein files are downloaded from is directly connected to RCSB PDB. Other sources of protein data websites were used, such as \href{https://web.expasy.org/protparam/} {Protein Parameter}, and \href{https://zhanglab.ccmb.med.umich.edu/I-TASSER/}{Protein Structure Function and Prediction I-TASSER Server}, in order to test the application during development.
The app only display full functionality on an iOS device, not a built-in simulator due to the requirement to use the camera to achieve the AR function.

\chapter{Analysis: Goals and Functional Requirements for Solutions}
\label{ch:analysis2}

This chapter identifies the goals of the project to make it predominant to the existing solutions to protein visualisation on AR application. Then, the functional requirements as well as the non-functional requirements to help achieve these goals are discussed. 

\section{Project Goals}
ProteinAR is an iOS application that visualises the three-dimensional structure of proteins. The project was set with three main goals:

(1) Provide an \textbf{educational experience}: enables download and visualisation of user-specified protein structure with data from RCSB PDB. As in Chapter \ref{ch:litRev}, there are a few existing apps that use AR to visualise protein. However, these apps need to scan a code or an image to display the protein, which creates a significant limitation as the protein needed to be pre-rendered. The apps that allow direct protein structure viewing in 3D by entering proteins names also are available but not in AR. 
Thus, the \emph{\underline{first goal}} of this project is to make it possible for the app to connect to RCBS PDB server, download the protein model, and display it on AR after the user types in the name of the protein. 

(2) Provide an \textbf{entertaining experience}: enables the creation of new proteins by the combining of polypeptide chains. As the existing apps on protein visualisation are more focused on simply displaying the protein, this project is set on bringing entertaining elements into the app by the addition of a mini-game function in which users can create new proteins. 
The \emph{\underline{second goal}} of this project is to enable users to create new proteins from the combination of coils, helices, and sheets. For this project, because of the biological complexity of the quaternary structure protein, the new protein created will be in tertiary form. 

(3) Provide an \textbf{interactive experience}: enables interactions between the user and the protein models or polypeptide chains displayed on the screen. By touching the models, users are able to scale the proteins, move the proteins around, and rotate the proteins. The findings from Chapter \ref{ch:litRev} shows that little effort has been put into interactive elements in existing products. The main function of the products is showing the protein. Therefore, the  \emph{\underline{third goal}} of this project is to enable interaction with the 3D models in AR.

\section{Functional requirements for solutions}
\subsection{Educational purpose: Visualising Protein from RCSB PDB server}
There are a few problems must first be addressed in order to visualise the proteins. 
Firstly, the app needs to be able to send requests to the RCSB PDB server. Secondly, the app needs to be able to download the files from the server. Thirdly, the app needs to be able to track the location of the downloaded files. Finally, the app should be able to open the files and display them as an AR layer on the screen. 
	\subsubsection{Send request and download the files}
As mentioned, the app needs to be able to send information (user input) to the server and retrieve the files. Based on this approach, the first attempt was to use the \emph{POST} and \emph{GET} method. This can be achieved by using \emph{HTTP Request} in Swift. 
\emph{HTTP POST Request} allows the app to post information to the destination URL where the specified embedded method is \emph{POST}. This is achieved by first accessing the website, then inspecting its element to find the action method as well as the parameters needed for in this method. 

Similarly, \emph{HTTP GET Request }allows the app to get information from the destination URL where the method is specified as \emph{GET}. The approach is the same as with \emph{POST}; usually the parameters can be found by inspecting the source code of the website, often under \emph{form action}.

To test the function, \href{https://web.expasy.org/protparam/}{ProtParam} was used as the website only consists of string type data. The URL for both \emph{POST} and \emph{GET} are the same and the methods are in the form action. However, since there is no PDB files on ProtParam, RCSB PDB has to be the data source. On RCSB PDB, the methods of \emph{POST} and \emph{GET} do not exist in the \emph{form action} function. The PDB files are directly downloaded by a separate URL in which the only variable part (parameter) is the name of the protein. Understanding this, ProteinAR uses \emph{URLSesssion} and \emph{downloadTask()}. \emph{URLSession} makes network transfers easy and \emph{downloadTask()} fetches the contents of a specified URL, saves it to a local file and calls a completion handle. The \emph{URLSession} tracks the storing place of the download task while it happens. This will be explained more in Chapter \ref{ch:implementation}.
	
	\subsubsection{Display the file}
When the 	files are downloaded, they are saved in the \emph{.pdb} format. In Swift, when a file is downloaded, it is downloaded to a temporary location, after which it can can be moved to the \emph{Document Directory}. ProteinAR specifies the format of the download by saving it as ``proteinName.pdb". 
The solution to visualise the \emph{.pdb} is to convert it to a \emph{.dae} files and then load it on the \emph{SCN}Scene as a scene. 
In order to load the file, one solution is to use move all downloaded items into the project folder using \emph{moveItem}. However, this affects app performance as the entire content of the project is loaded every time the app is run. The solution that was used in this app is to keep all downloaded files in the \emph{Document Directory}. The file path and file name will be specified so that whenever a model is needed, it will be identified using these attributes. 
For loading the file, there needs to be a converter which converts the downloaded \emph{.pdb} file to \emph{.dae} file. This converter will automatically convert any downloaded \emph{.pdb} file in the \emph{Document Directory} into \emph{.dae} so that it can be loaded as a \emph{SCNScene} in the app. 
Unfortunately, due to technical difficulties as well as time constraints, a converter could not be made and remains the largest avenue for future work on this app. Therefore, to demonstrate the app's functionalities, a sample folder of existing "protein.dae" files is imported. 

 	
\subsection{Entertainment purpose: Create new proteins from combination}
\subsubsection{Add polypeptide chains to screen}
The app needs to be able to display user-selected individual polypeptide chain. There are four types of polypeptide chains: Flex Coil, Rig Coil, Helix, and Sheet. Each polypeptide chain is input into the project as a \emph{.dae} model. In order for these models to be loaded on ARKit, they must be converted into \emph{.scn} files. Each model consists of different nodes: the model, lighting, camera, etc. By using the pre-defined function of \emph{SCNScene}, the 3D models can be loaded into the AR view. By passing  the name of each models as a parameter, only one function is needed to add each of the four different polypeptide chains using four different buttons. 

If a model is loaded on screen and the location is not specified, the model may render off-screen. 
An additional problem can occur when two of the same polypeptide types are loaded: in such an occurrence, the two models will appear in the exact same location with the exact same orientation appearing as if there is in fact only one model on the screen. To solve this problem, the app randomises the orientations of the models every time a new model is added to screen by using the pre-defined function of \emph{eulerAngles} to specify the \emph{SCNVector3} with random x, y, and z.

\subsubsection{Combining polypeptide chains}
After adding individual polypeptide chains to the screen, ProteinAR must be able to combine these chains into proteins. If such a combination exists, then the resulting protein should be displayed. For this to happen, successful combinations of these chains are pre-loaded into the apps in a “Combinations” folder. 
In the code, an empty string array for the protein name is created. Every time a user adds a polypeptide chain to the screen, the name of the protein is appended to the array. After a user clicks the “Try” button to combine the polypeptide chains, the names in the array are concatenated using the \emph{array.joined()} function. The name of the models in the “Combinations” folder have a naming convention so that when the array are joined, the name it generated matches with the name of the models in the “Combinations” folder. See Chapter \ref{ch:implementation} for further details. 

\subsection{Interactive purpose: Interacting with the models}
After the polypeptide chains or the protein models are loaded onto the screen, users should be able to interact with the models by using the touchscreen. To make this happen, \emph{UIGestureRecognizer} was used. There are three types of \emph{Gesture Recognizer} used in ProteinAR:

\begin{table}[h!]
\centering
\begin{tabularx}{\textwidth} {
  | >{\raggedright\arraybackslash}X 
  | >{\raggedright\arraybackslash}X 
  | >{\raggedright\arraybackslash}X | }
\hline
UI Gesture & Gesture Description & Function in the app \\
\hline
\hline
Pinch Gesture & “A two-fingers gesture that moves the two fingertips closer or farther apart” \parencite{wang_beginning_2018}. & Allows users to scale (zoom in, zoom out) on the model. \\
\hline
Rotation Gesture & “A two-fingers gesture that moves the two fingertips in a circular motion” \parencite{wang_beginning_2018}. & Allows users to rotate the models in any angle. \\
\hline
Pan Gesture & “Press a finger on the screen and then slide it across the screen” \parencite{wang_beginning_2018}. & Allows users to move the models on the screen. \\
\hline
\end{tabularx}
\caption {Interacting Gestures in ProteinAR}
\label{tab:gesture}
\end{table}


\section{Non-functional requirements}
\subsection{Core Data}
Core Data is a popular framework provided by Apple to manage the model layer object in an application. 
Core data can automate solutions to common tasks associated with object life cycle and object graph management, including persistence \parencite{noauthor_core_nodate}. In this app, in order to manage the downloaded PDB files, Core Data is used in which Protein is defined as an \emph{Entity}, having two attributes \emph{name} and \emph{location}, stored as \emph{String}. After the file is downloaded and stored in \emph{Document Directory}, a new \emph{Protein Entity} is saved into Core Data, with two attribute values of \emph{name} and \emph{location}. When a model is called, it will use the specified ``proteinName" name attribute and ``filePath.dae" location attribute to open the file in the app. 
Because there are only two attributes in a \emph{Protein Entity}, it can easily be replaced by using String value directly in the app to call the files. However, taking the future work into consideration, where more attributes can be added to a \emph{Protein Entity} such as molecular weights of protein, number of amino acids in a protein, etc., using Core Data can help managing and displaying all these values more effectively.

\subsection{Constraint}
Although it is not mandatory, the app should be able to run on different iOS devices without problem. As the screen size of different iOS devices are different, if the app was designed on the view of iPhone 11 but run on iPhone 6, the buttons might be off screen or other elements might move around, making it impossible to navigate through the app. This is why constraints are important in developing an iOS app. ProteinAR does not have many elements on the screen at the same time, however, the \emph{Auto Layout} was chosen as the solution for the constraints. Using \emph{Auto Layout}, every new view that is a layer on top of a view is made into a \emph{childView} attaching to the \emph{parentView} which makes it easy for the anchor to be pinched to the \emph{parentView}. \emph{NSLayoutConstraint} was used to keep the elements in place. 

\subsection{Protein combination models and Polypeptide chains models management}
The combinations of polypeptide chains are kept in a “Combination” folder and has a naming convention that makes it easy to find each model. It is the combination of the names of the polypeptides, which makes it possible for the array to be combined into the new names. 

When users tap on the individual polypeptide buttons, the models are displayed distinctively without having to tap on the “Try” button. When the “Try” button is tapped, the models on screen combine. In order to do this, the function to add polypeptide chain is created separatedly. 

\section{Summary}
 There are three main goals the project was set to achieve. The first goal is to allow download and visualisation of protein structures from RCSB. The second goal is to allow new protein structures creation and the third goal is to enable interactions with the structures. For this to happen, a number of functional requirements are needed. These include functions to download the files, display the files, loading models as individual and combination, as well as gesture recognitions. Besides, non-functional requirements are also mentioned such as the use of core data, constraints and models management. 

\chapter{Project Design}
\label{ch:design}

This chapter elaborates the design of ProteinAR by starting with the skeleton of the app, followed by the solutions design for each and every functions. The overall structure of the app is illustrated by an UML diagram. After that, the UI design is explained. A brief description on the design of core data is presented in the end of the chapter.

\section{Application Skeleton Design}
The structure of ProteinAR is fairly simple. It consists of four main screens including the landing screen as shown in Figure \ref{fig:appskeleton}. 
\begin{figure}[!htp]
	\centering
	\includegraphics[width=\textwidth]{images/appskeleton.png}
	\caption{The skeleton of the app}
	\label{fig:appskeleton}
\end{figure}

On the landing view (first view) (Figure \ref{fig:firstview}), there are three buttons, leading to the three other views of the apps. 
\begin{figure}[!htp]
	\centering
	\includegraphics[scale=0.6]{images/firstscreenview.png}
	\caption{The First Screen View – Landing view after launch screen}
	\label{fig:firstview}
\end{figure}

(1) By tapping on \emph{Introduction}, the segue will bring up the introduction view. Instead of using multiple screens connecting from the introduction, there are three sub-screens added by using \emph{page control} on the \emph{Introduction Screen View} to give information about the app as shown in Figure \ref{fig:introviewxib}.
\begin{figure}[!htp]
	\centering
	\includegraphics[width=0.5\textwidth]{images/introviewxib.png}
	\caption{Introduction View Controller and Page Controller}
	\label{fig:introviewxib}
\end{figure}
Using \emph{Page Control} maintains coherency for the same content, while at the same time keep less words per screen, making it more appealing to users. Users can click on the \emph{GET STARTED} button to go back to the first view to explore the options or simply drag the screen down and away.

(2) Tapping on the \emph{Education} will bring users to the Education View Controller as shown in Figure \ref{fig:eduview}.
\begin{figure}[!htp]
	\centering
	\includegraphics[scale=0.6]{images/eduview.png}
	\caption{Education View Controller}
	\label{fig:eduview}
\end{figure}
Since the main function on this screen is for the user to input the protein’s name and get the pdb file back,  the design is kept simple with a \emph{textfield} and a \emph{GET button}. To make the app more appealing, there are four more buttons with four more minor actions on top of the screen. These actions are \emph{Menu, Screen Record, Screen Capture} and \emph{Exit} as shown in Figure \ref{fig:topbuttons}. These functions will be explained in section 5.2.   
\begin{figure}[!htp]
	\centering
	\includegraphics[scale=0.8]{images/topbuttons.png}
	\caption{Four buttons on top of Education View Controller and Game View Controller}
	\label{fig:topbuttons}
\end{figure}

(3) Tapping on \emph{Mini-game} will bring up the \emph{Game View Controller} (Figure \ref{fig:gameview}).
In this view, the user can create new proteins by combining the coils, helix and sheet in different orders simply by adding each polypeptide onto the screen by tapping on them, and then pressing \emph{Try}. Similar to Education View Controller, the four buttons on top of the screen are kept. 
\begin{figure}[!htp]
	\centering
	\includegraphics[scale=0.6]{images/gameview.png}
	\caption{Game View Controller}
	\label{fig:gameview}
\end{figure}


\section{Solution Design}
\begin{figure}[!htp]
	\centering
	\includegraphics[width=\textwidth]{images/uml.png}
	\caption{Application Solution Design}
	\label{fig:uml}
\end{figure}

Figure \ref{fig:uml} shows the overall structure of the app. In this Figure, the four frames represents the four screens of the app. Different colours are used to indicate different components of the app.
\begin{itemize}
\item \textbf{Purple} represents the buttons displayed on the app's screen.
\item \textbf{Yellow} represents user action (swipe, press, tap, long press, type).
\item \textbf{Green} represents the options displayed on the screen after an action was taken.
\item \textbf{White} represents the actions that the app executes after a button was pressed or an option was chosen.
\item \textbf{Pink} represents the destination screen or URL the app opens after a button was pressed or an option was chosen. 
\end{itemize}
The app starts from "First View Controller" and depends on the chosen buttons, the suitable "View Controller" will be in display. From any "View Controller", the user can always go back to "First View Controller" by pressing on the "Exit" button. The details of each function design will be elaborate in the following part of this section.

\subsection{Utility buttons}
The utility buttons are the same on both Education View Controller and Game View Controller. This creates coherency throughout the app. However, the downside is that all functions and buttons have to be duplicated on the two views, causing heavier memory load for the app. 
\begin{itemize}
	\item \emph{Menu} is the function that gives users extra options. The extra options on Education View Controller and Game View Controller are slightly different. On Education View Controller, when the user press \emph{Menu}, an \emph{Alert Service} is used, where the options rise up from the bottom of the screen, giving the user four options: \emph{More About Protein}, \emph{Help}, and \emph{PDB 101} and \emph{Cancel}. While \emph{More About Protein} get users directly to the \href{http://rcsb.org}{homepage of RCSB PDB} and \emph{PDB 101} links to the \href{http://pdb101.rcsb.org}{PDB 101 page} on the RCSB website, the \emph{Help} options bring a small pop-up screen layer on top of the AR scene. This pop-up screen contains some guidelines on how to use and navigate around the \emph{Education View} and \emph{Game View} respectively. 
Since this is more akin to a demo-app, the options only directly open link to the RCSB website, however, in future development, more in-depth options can be integrated to create a more scientific experience for the user. 
	\item \emph{Record} is the function to record the AR screen and then save the recorded video to the phone’s \emph{camera roll} if the users choose to do so. In interacting with a protein or creating a new one, users might want to record the process as there might be interesting and new findings for future study. When users long press on the \emph{Record} button, the recording process will start. By doing so, there will be a pop-up on screen asking for permission to save the recorded file to the \emph{camera roll}. The recording can be stopped simply by tapping on the record button. The app will then bring up a \emph{Preview screen}, allowing users to watch the recorded video before deciding to save the video or not. The two actions of \emph{long-pressing} and \emph{tapping} are enabled using the \emph{UIGestureRecognizer} of the ARKit. 
	\item \emph{Camera} is the function to capture the AR screen and save the photo to the phone’s \emph{Camera Roll}. When the user taps the \emph{Camera} button, the screen will be captured and saved immediately. The UIButton flashes colours to indicate that the shot has been taken.  Even though iOS already has the screen-capture function, by using that, all the buttons on the screen will also be captured, which is not desirable. With this \emph{Camera} function, users can save a photo of just the protein they want. 

\end{itemize}

\subsection{Getting pdb files from RCSB Server and storing it in CoreData}
This is one of the critical functions of the project. It requires the app to be able to download the PDB files, save it and then display it on the screen. 
In order to achieve the download function, there were much trial and error as mentioned in Chapter \ref{ch:analysis2} of using the \emph{HTTP Request POST} and \emph{HTTP Request GET} method. In the process of making the task possible, Alamofire was also considered as an option. Alamofire is a Swift-based HTTP networking library for iOS which simplifies a number of common networking tasks. However, after a few tries, the conclusion was that it was not necessary since the main task of the function is just to download a PDB file. This can be achieved using \emph{URLSession} with \emph{downloadTask()}. The downloaded destination is pre-defined to the internal \emph{Document Directory}. 
To save the downloaded file’s information to CoreData, firstly, a CoreData model was created with an \emph{Entity} Protein. This \emph{Entity} has two attributes of \emph{name} and \emph{location}. Both attribute types are of \emph{String} type. If the download process is successful (the file exists, the connection was stable, etc.), at the same time of downloading, a new \emph{NSObject} is created with the two attributes. These will be saved as an \emph{Entity} in the CoreData database. 
\emph{NSFetchRequestResult} is used to fetch the data in CoreData database back to the app. This process is visualised in Figure \ref{fig:coredata}.

\begin{figure}[!htp]
	\centering
	\includegraphics[scale=0.7]{images/CoreData.png}
	\caption{Process of downloading, saving and displaying downloaded protein model}
	\label{fig:coredata}
\end{figure}


\subsection{Visualisation of protein models from pDB files}
In order to visualise protein models from downloaded pDB files, a converter to convert file type \emph{.pdb} to file type \emph{.dae} must be made. The ideal design is as shown in Figure \ref{fig:coredata}. UCSF Chimera was used in the process of converting, however, it is only compatible with MacOS, not iOS and therefore could not be implemented into the app.

\subsection{Combining polypeptide chains into a protein}
Each polypeptide chain is designed to be referred to as a value in an array. Every time a user presses on a polypeptide chain’s button, that model of protein is displayed, while at the same time, that model’s name is added as a value in the array. After these actions, if the \emph{Clear} button is pressed, not only are the models on the screen deleted but also the values in the array are emptied. On the other hand, if \emph{Try} is pressed, all the values in the array will combined into a new name. First, the screen will be cleared and then the new name protein model will be displayed. Together with this, a text of “Congratulations, you have created a new protein made of …” will also be displayed if the combination is valid. If the combination is invalid, no models will be displayed. The errors will be caught and, on the console, “This model does not exist” will be printed. On the user's end, a 3D text of “Sorry, this combination cannot be made” will appear on screen. The simplification of the design can be found in Figure \ref{fig:array}.
\begin{figure}[!htp]
	\centering
	\includegraphics[width=\textwidth]{images/array.png}
	\caption{Create a new protein name from existing ones}
	\label{fig:array}
\end{figure}


\section{User Interface (UI) Design}
\subsection{Introduction to user interface}
In order to appropriate the tools of computers and smart devices, users need to communicate with them. The way users can communicate with the product (software, app, website) is through interacting with the user interface (UI) of that product. The purpose of a UI is to enable users to control a computer or a device they are interacting with, by giving commands and receiving feedback in a chain to complete a task. 
The user interface of any computer-based product does not only create first impressions which convinces users to continue to use that product, it also plays an important role in maintaining the interest of the users. With a complicated or inefficient UI, users would not want to keep using the product because it requires too much cognitive effort. Therefore, a UI should be \emph{intuitive} – be kept simple where no training should be needed to operate, and be \emph{efficient} – functions are precise, on point, and \emph{user-friendly} \parencite{noauthor_what_nodate}.
Currently, there are three formats of user interfaces \parencite{noauthor_what_nodate}:
\begin{itemize}
\item \textbf{Graphical User Interfaces (GUIs)} – interactions happen through visual representations on digital control panels such as a computer desktop, or a website interface.
\item \textbf{Voice-controlled interfaces(VUIs)} – interactions happen through voice representation such as Siri, Google Home or Alexa.
\item \textbf{Gesture-based interfaces} – interactions happen through physical motions in 3D spaces, such as in VR games.
\end{itemize}
The UI of ProteinAR is categorized as a GUI since users interact with the device through the visual representations of functions on a phone screen. 

\subsection{ProteinAR’s user interface design}
\subsubsection{Logo design}
The logo for the app was designed simple with just a letter P. This was created in GIMP and then exported to various sizes to maintain the resolution in different views (refer to Figure \ref{fig:appicon}). Other designs with symbols or words were considered but sticking to the “simplicity is the best” approach, the logo ended up with only one simple letter and two colours, making it easy to remember for users. This is not a new approach. Simple logo design with only one letter can be found amongst popular apps such as Facebook app or Google app. 
\begin{figure}[!htp]
	\centering
	\includegraphics[scale=0.65]{images/appicon.png}
	\caption{App’s logo in different sizes.}
	\label{fig:appicon}
\end{figure}

\subsubsection{Colour scheme}
The logo, the flash screen, the buttons and popup view elements in the app all follow the same colour scheme. In ProteinAR, an analogous colour scheme was chosen as shown in Figure \ref{fig:UIFinal}. 
\begin{figure}[hbt!]
	\centering
	\includegraphics[width=\textwidth]{images/UIFinal.png}
	\caption{UI design of ProteinAR}
	\label{fig:UIFinal}
\end{figure}

This is one of the traditional colour palettes which is the combination of related colours that are placed next to each other on the colour wheel. Analogous is known to be one of the most-used colour pallets because they are harmonious and pleasing to the eyes. ProteinAR uses two colours from the Pinks and Mauves colour sections as shown in Figure \ref{fig:colorwheel}. 
\begin{figure}[hbt!]
	\centering
	\includegraphics[scale=0.6]{images/colorwheel.png}
	\caption{Analogous Colour Scheme}
	\label{fig:colorwheel}
\end{figure}


\subsubsection{Button design}
As for the utility buttons of the Education and Mini-game screen, the main colour scheme is maintained. As the first three buttons generate actions, they are in the same mauves colour and the exit button is in pink, which creates the slight distinction of the function. The two main function buttons of \emph{Try} and \emph{Clear} also follow the main colour scheme. 

In the utility buttons, button icons are used instead of button labels. These icons are familiar to mobile app users, this makes the design more concise and easier to navigate. 

(1) Menu button: there are many styles of menu buttons as shown in Figure \ref{fig:menubuttons}. Each menu buttons generates a type of menu display. For example, the \emph{hamburger icon} opens a navigation drawer to more actions; the \emph{kebab icon}, commonly seen on Android operating system, normally opens a smaller inline menu. In this project, the chosen icon for the \emph{Menu} button is the \emph{Veggie burger} style as it is common for this style to be placed on the top left of the screen, and it symbolises generating more actions but less actions than a \emph{hamburger icon}.
\begin{figure}[hbt!]
	\centering
	\includegraphics[width=\textwidth]{images/menubuttons.png}
	\caption{Different styles of menu buttons – source: \href{https://ux.stackexchange.com/questions/115468/what-the-difference-between-the-2-menu-icons-3-dots-kebab-and-3-lines-hambur}{ux.stackexchange}}
	\label{fig:menubuttons}
\end{figure}

(2) Record and Camera button: The record button accepts two types of actions: long press and tap. Long press generates the action of recording the screen and tap ends it. The long press action also changes the colour of the button to red, which is commonly associated with recording. Tap brings it back to its original colour, which symbolises the end of the recording action. As for the camera button, the colour only flashes, implying the act of picture taking has been done. 

As mentioned, the buttons in ProteinAR mainly follow the colour scheme of Pinks-Mauves. However, the four buttons to add polypeptide chains are the exceptions. These four buttons use images as the buttons and to increase usability for users, they are labelled with their names. The designs of the buttons are inherited from Tianshu Xu’s 2019 Master project \parencite{xu_interactive_2019} and the label colours were chosen to be matched with the colours of the polypeptides. Since this is a mini-game, the colourful elements are chosen for visual appeal and clarity. 
\begin{figure}[hbt!]
	\centering
	\includegraphics[scale=0.7]{images/polybuttons.png}
	\caption{Polypeptide chains button}
	\label{fig:polybuttons}
\end{figure}

\section{Core Data Design}
In ProteinAR, Core Data is simply designed with only one \emph{Entity} called Protein. This \emph{Entity} has two \emph{attributes}: \emph{name} and \emph{location} of type \emph{String}. In the current design of the app, Core Data only stores two information in a Protein \emph{Entity}: the name of downloaded protein and its file path in the \emph{Document Directory}. When the data is called, information stored in the attribute \emph{location} of Core Data simply act as a \emph{String} value to concatenate with the file's name and extension to fetch the file that are stored in the \emph{Document Directory}. 
Figure \ref{fig:datamodel} shows the \emph{Entity} and \emph{attributes} in the data model in Core Data.
 \begin{figure}[!htp]
	\centering
	\includegraphics[width=\textwidth]{images/datamodel.png}
	\caption{Core Data Entity and attributes}
	\label{fig:datamodel}
\end{figure}

The details on implementation of Core Data can be found in Figure \ref{fig:displayreal} with explanation in Chapter \ref{ch:implementation}. As explain in Chapter \ref{ch:analysis2}, with the current implementations of ProteinAR, integrating Core Data is unnecessary. Document path can be directly called into the displaying function using \emph{File Manager}. The reason Core Data was used is for future work, when more protein data may need to be stored. 





\chapter{Project Implementation}
\label{ch:implementation}

\section{Download and Visualise Protein Models}
Due to its complexity, the process to download and visualise protein models will be explained in five steps.
\subsection{Step 1. Set up Core Data (Figure \ref{fig:datamodel} and Figure \ref{fig:subclass})}
First, Core Data is set up by adding a new \emph{Data model} from the \emph{Core Data} section. It is important to add-on the \emph{App delegate} if \emph{Core Data} was added in later in the project because Xcode will not automatically generate those delegate and the database will not be set up. In this app, the database has only one entity with the name of \emph{Protein} which has two \emph{attributes} of \emph{name} and \emph{location}, defined in type \emph{String} as in Figure \ref{fig:datamodel}. After that, \emph{NSManagedObjectSubclass} were created where Protein is defined as a public class in \emph{NSManagedObject} and so as function \emph{fetchRequest}.  In these subclass, the attributes of \emph{name} and \emph{location} are also declared as public variables (refer to Figure \ref{fig:subclass})
 \begin{figure}[!htp]
	\centering
	\includegraphics[width=\textwidth]{images/datamodel.png}
	\caption{Core Data Entity and attributes}
	\label{fig:datamodel}
\end{figure}

 \begin{figure}[!htp]
	\centering
	\includegraphics[width=\textwidth]{images/subclass.png}
	\caption{NSManagedObject subclass}
	\label{fig:subclass}
\end{figure}

\subsection{Step 2. Download from RCSB PDB using downloadTask (Figure \ref{fig:download1} and Figure \ref{fig:download2})}
\emph{Download} is the critical function in this process and since it is a long one, the code snippet will show the code in two parts with modification for easy explanation. The full functioning code with alternative options can be found in Appendix A or in the attached source code folder. 

In the code shown in Figure \ref{fig:download1}, the URL to the source file is created. After observing how files are downloaded from RCSB, an URL pattern was found. Instead of using the \emph{GET} method in \emph{action form}, RCSB allows downloading the PDB files directly from an URL. The structure of these URL are the same for all of the different PDB files, starting with the same domain. The file name is the only part that needs to be changed, and it is the protein's name. With this logic, the URL to the source file was constructed using the parameter as the user-input-text to change the file name accordingly. 
 \begin{figure}[!htp]
	\centering
	\includegraphics[width=\textwidth]{images/download1.png}
	\caption{Download function part 1}
	\label{fig:download1}
\end{figure}

The code snippet in Figure \ref{fig:download2} shows how the code uses \emph{URLSession} and \emph{downloadTask()} to generate the download. \emph{URLSession} provides API for downloading and uploading data to the specified URLs. This API helps performing background downloads. In this code, \emph{default} type of \emph{URLSession} is used instead of \emph{shared} because it allows more freedom of configuration. \emph{URLSessionConfiguration} defines the behaviour policies when the app downloads data from the server. There are a few types of \emph{URL Session Tasks}. In this app, \emph{download task} is used as it retrieves data in the form of a file and supports background downloads. 
The status code of \emph{HTTPURLResponse} is important to be known because if the file cannot be downloaded, the problems could be addressed as there could be different reasons that triggers the unsuccessful downloads: the file does not exist (status code 404), the connection to the server was interrupted (status code 500) or something else is wrong with the code. 
When the code performs its download task, the file is store in a temporary location, as called in the code \emph{temporaryURL}. The file will be moved to an absolute path in the \emph{Document Directory} by using the \emph{File Manager} to remove and move item. In order to keep track of the downloaded files and create an absolute path, the name of the files are pre-defined by \emph{destinationURL} (refer to Figure \ref{fig:download2}). 


 \begin{figure}[!htp]
	\centering
	\includegraphics[width=\textwidth]{images/download2.png}
	\caption{Download function part 2}
	\label{fig:download2}
\end{figure}


The alternative way is to move the downloaded file into the main app bundle as shown in Figure \ref{fig:movetobundle}. The directory of the main app's bundle is created and the file can be moved by the same \emph{moveItem()} method. In the source code, this alternative way is disable. The reason for this was previously explained: if all the downloaded files are saved into the main app's folder, the app will have to load them all every time it is loaded, making the app heavy and slow. 
 \begin{figure}[!htp]
	\centering
	\includegraphics[width=\textwidth]{images/movetobundle.png}
	\caption{Alternative: Move downloaded files to main app's folder}
	\label{fig:movetobundle}
\end{figure}


\subsection{Step 3. Assign downloaded files to Core Data (Figure \ref{fig:savefunc})}
Firstly, the \emph{context} is declared by \emph{persistentContainer} and the \emph{proteinManagedObject} is declared by starting with \emph{nil}. 
Then, the function to save \emph{proteinManagedObject} \emph{context} is called inside of the \emph{do} action in the download function (Line 433 -Figure \ref{fig:download2}). The function to save context is displayed in Figure \ref{fig:savefunc}. 
When the download is successful, the attributes of \emph{proteinManagedObject.name} and \emph{proteinManagedObject.location} are saved into the Protein \emph{Entity} as \emph{String}. 
Since \emph{proteinManageObject} is a global variable, it can be accessed anywhere in the code. 
 \begin{figure}[!htp]
	\centering
	\includegraphics[width=\textwidth]{images/savefunc.png}
	\caption{Save and Assign downloaded file to attributes in Core Data}
	\label{fig:savefunc}
\end{figure}

\subsection{Step 4. Convert PDB file to Collada file}
The process would be completed with a script converting \emph{.pdb} file to \emph{.dae} file type because ARScene only allow loading Collada models as its AR Scene. 
A few solutions were used to solve this problem. One of those is to borrow the converting from PDB to Collada script from UCSF Chimera. Since Chimera was written in Python, its scripts could be run in Swift because Python has a C interface API. However, the problem with this was Chimera is not compatible with iOS in the first place, thus, this solution could not be used. 

OpenBabel was another a solution that was carried out. Unfortunately, OpenBabel is also just compatible to Android and MacOS, not iOS and since, was not able to be implemented. 

RCSB PDB published an article on the releasing of their mobile version in 2015 which can help visualise the PDB file on both iOS and Android, however, as of 2020, it was no longer available on the App Store.

Thus, this remains an unsolved problem of this project.

\subsection{Step 5. Fetch and Visualise PDB files (Figure \ref{fig:displayreal})}

The data which are saved into the Core Data would be fetched using \emph{NSFetchRequestResult} and can be display easily using the attributes assigned into Core Data. If the PDB to Collada converter script can be made, the function could just be as easy as shown in Figure \ref{fig:displayreal}
 \begin{figure}[!htp]
	\centering
	\includegraphics[width=\textwidth]{images/displayreal.png}
	\caption{Function to display protein after being converted into Collada models}
	\label{fig:displayreal}
\end{figure}

Due to the unsolved problem of PDB to Collada converter, the app demo video shows some pre-downloaded protein models to give a complete image of how the app would be done if given more time in the future work.

\section{Create new Protein Models}
\subsection{Step 1. Import and name models (Figure \ref{fig:combinations})}
Since ProteinAR uses ARKit and SceneKit to load the models on, for importing the models, first, a new file of \emph{Scene Catalogue} must be made. In this project, the folder is named “Combinations”. The provided combination model type was in \emph{.dae}, however, to enable smooth loading for SceneKit, the files are converted into \emph{.scn} type. These are named after the polypeptide chains names and their order in creating the combinations. See Figure \ref{fig:combinations} for more details of some examples.
 \begin{figure}[!htp]
	\centering
	\includegraphics[scale=0.7]{images/combinations.png}
	\caption{Combinations of polypeptide chains stored in a folder}
	\label{fig:combinations}
\end{figure}
This naming convention makes it easy to pass as arguments and load models in the following functions. To make the models more appealing, Phong shading is used and the colour of each and every model are randomly picked. 

\subsection{Step 2. Add polypeptide function (Figure \ref{fig:addProteinfunc}, \ref{fig:polypeptidefunc})}
Figure \ref{fig:addProteinfunc} shows the function to add the polypeptides on the screen. In this function, the argument is pre-defined as \emph{String} type and has the name of \emph{name}. In \emph{ARKit}, \emph{SCNScene} is used to load the 3D models. Since all the models have the same format (inside of “Combinations.scnassets” folder and has the “.scn” extension), the models will easily be called by passing the names of the protein as arguments each time a Polypeptide button is pressed, as shown in Figure \ref{fig:polypeptidefunc}. 
The function also add a camera node to the screen at the position of (0, 0, 0) and the position of the models are also fixed by \emph{SCNVector3} to ensure that the models will always appear in front of the camera. One of the problems that users might encounter with using AR technology is that the space is infinity and so the models might be loaded in places that cannot be seen. Thus, it is more user-friendly to make sure the position of the models loaded are visible. The function also uses \emph{eulerAngles} with a random \emph{SCNVector3} to make sure that every time new models are loaded, they are at the same position, but with different rotation so they do not lay on top of each other perfectly, ensuring users that they have already added the same models more than one time. 
\begin{figure}[!htp]
	\centering
	\includegraphics[width=\textwidth]{images/addProteinfunc.png}
	\caption{Function to add protein to the screen}
	\label{fig:addProteinfunc}
\end{figure}

\begin{figure}[!htp]
	\centering
	\includegraphics[width=\textwidth]{images/polypeptidefunc.png}
	\caption{Actions happen when users press on each Polypeptide Button}
	\label{fig:polypeptidefunc}
\end{figure}


\subsection{Step 3. Create new protein (Figure \ref{fig:clearScreen}, \ref{fig:createProtein})}
To simplify the process, the combinations of protein are not completely new but instead, loaded from the “Combinations” folder and displayed. To create a smooth transition and generate the feeling of joining the polypeptide chains, the process has three small steps. 

\subsubsection{Clear everything off the screen}
Clearing the screen off will make the transition to a new model feels more approved. 
\begin{figure}[!htp]
	\centering
	\includegraphics[width=\textwidth]{images/clearScreen.png}
	\caption{Function to clear models off the screen}
	\label{fig:clearScreen}
\end{figure}

As the models are added on the screen as nodes (model node, camera node, light node), simply removing all the node from \emph{ParentNode()} would enable clearing off the screen.

\subsubsection{Load a combination model}
In Figure \ref{fig:polypeptidefunc}, it is shown that every time a button is pressed, besides loading a model onto the screen, it does something else. An empty array variable was declared in the beginning and every time a button is pressed, a value is added to the array. For example, when \emph{fCoil Button} is pressed, “fCoil” is added to the array. These values in the array will then be joined using \emph{array.joined()} in the function to create the combination name as shown in Figure \ref{fig:createProtein}. Without any separator, these joined values will become exactly like the names of models in the “Combinations” folder, which enable the code to run and load the models from there. 
\begin{figure}[!htp]
	\centering
	\includegraphics[width=\textwidth]{images/createProtein.png}
	\caption{Function to create a new protein}
	\label{fig:createProtein}
\end{figure}

\subsubsection{Display 3D text}
These functions are called inside of function \emph{createProtein}. If the combination that user created is valid, together with the model, the text will be loaded with “Congratulations”, and followed by the names of the polypeptides in order of input. If the combination that user created is invalid, no model would be loaded and instead, only the 3D text will appear with “Sorry”! The combination of (\emph{user-pressed buttons}) cannot be made. See Appendix A for this function. 

\section{Interactive elements}
\subsection{The three gestures to interact with Protein Models (Figure \ref{fig:pinch})}
ARKit is a very powerful framework as it cuts off a lot of coding work to enable gesture interaction. To enable gesture interactions, the first step is to drag and drop the gesture on \emph{Main storyboard} from the built-in library. The three gestures used in ProteinAR are \emph{Pinch Gesture} , \emph{Rotation Gesture} and \emph{Pan Gesture}. The gestures are initiated by \emph{.state: .change}. As in ProteinAR, the goal of interaction is the full screen, the area of gesture is set to be \emph{SCNView} and \emph{hitTest} is used to run the gesture. 

In the function to generate \emph{Pinch Gesture} as shown in figure \ref{fig:pinch}, \emph{SCNVector3} is used with changeable (x, y, z) set in float. By using the two fingertips, users can zoom in and zoom out on the models. 
\begin{figure}[!htp]
	\centering
	\includegraphics[width=\textwidth]{images/pinch.png}
	\caption{Pinch Gesture function}
	\label{fig:pinch}
\end{figure}
The other two functions of rotation and pan gesture are similar to pinch gesture and thus, will not be displayed in code here. The codes can be found in the Appendix A . 

\subsection{Other interactive elements (Figure \ref{fig:taplong}, \ref{fig:objcfunc})}
Although it is not a compulsory requirement of the app, more interactive display will make the UI more appealing, thus, some other gestures and touches are added in the app. This might not seem very obvious but it improves the user experience.
\subsubsection{Gestures Recognizer for button}
For the \emph{Record} button, the two gestures of \emph{Tap} and \emph{Long press} were added. There might be other methods to do this, however, creating two objective-C functions was the simplest solution. For this to work, the button should not be connected to the code as an action, but an outlet. Then, in the \emph{viewDidLoad()}, \emph{GestureRecognizer} can be added to the outlet as shown in Figure \ref{fig:taplong}. These \emph{GestureRecognizer} needs handling, which will be handled in objective-C’s function (refer to Figure \ref{fig:objcfunc})
\begin{figure}[!htp]
	\centering
	\includegraphics[width=\textwidth]{images/taplong.png}
	\caption{Add Gesture Recognizer to Button outlet}
	\label{fig:taplong}
\end{figure}
\begin{figure}[!htp]
	\centering
	\includegraphics[width=\textwidth]{images/objcfunc.png}
	\caption{Objective-C functions to handle Gesture Recognizer}
	\label{fig:objcfunc}
\end{figure}

\subsubsection{Dismiss subview and keyboard (Figure \ref{fig:dismiss})}
When user finishes reading the guidelines on \emph{Help Screen View} or finishing input in the \emph{textFied}, the sub-screen and the keyboard should be dismissed. For the \emph{Help Screen View}, the solution was to use \emph{UITouch}. This is set so that if users touches every place that is not the \emph{Help Screen View}, the view will be hidden. 
For the keyboard, usually it can just be set with \emph{textFieldShouldEndEditting} after specified \emph{TextFieldDelegate} in the class. However, since in ProteinAR, the whole screen are covered in other \emph{UIGesture}, this did not work. The solution was to set the \emph{Return} key to \emph{Done} key in the \emph{viewDidLoad} and then use the function of \emph{textFieldShouldReturn} to dismiss the keyboard, as shown in Figure \ref{fig:dismiss}.

\begin{figure}[!htp]
	\centering
	\includegraphics[width=\textwidth]{images/dismiss.png}
	\caption{Others interactive elements}
	\label{fig:dismiss}
\end{figure}


\chapter{Testing and Evaluation}
\label{ch:evaluation}

In this chapter, the conducted test results will be reported. These tests include function testing, unit testing, performance testing, and usability testing. The result of each test, as well as the analysis, and limitations of the tests will be provided. Finally, an overall evaluation of the project is summarised in table \ref{tab:evaluation}.

\section{Function Testing}
\subsection{Education Screen}
When the protein ID is inputted, the model of the protein is displayed.
In Figure \ref{fig:eduscreen} on the left, the protein 6K01 is displayed on the AR screen as 6K01 is a valid protein ID. On the right, when the protein is invalid, the screen shows only 3D text informing the user that protein does not exist.
 \begin{figure}[!htp]
	\centering
	\includegraphics[scale=0.6]{images/eduscreen.png}
	\caption{Education App Screen}
	\label{fig:eduscreen}
\end{figure}

\subsection{Mini-game screen}
On the Mini-game screen, when a polypeptide chain button is pressed, the model of that chain will be displayed. Similarly, when another polypeptide chain button is pressed, the second model appears on top of the first one as shown in Figure \ref{fig:minigamescreen1}. If the "Try" button is then pressed and the combination exists, it will be displayed with text showing the names of its elements. If the combination does not exist, only text will appear as shown in Figure \ref{fig:minigamescreen2}.
 \begin{figure}[!htp]
	\centering
	\includegraphics[width=\textwidth]{images/minigamescreen1.png}
	\caption{Mini-game App Screen (1)}
	\label{fig:minigamescreen1}
\end{figure}

 \begin{figure}[!htp]
	\centering
	\includegraphics[scale=0.6]{images/minigamescreen2.png}
	\caption{Mini-game App Screen (2)}
	\label{fig:minigamescreen2}
\end{figure}

\section{Unit Testing}
By running the app, only functions that can be displayed on the screen can be tested. Functions to download the PDB files cannot be tested in this way. Therefore, in this project a unit test was built to test the download function. 
In Xcode, the \emph{XCTest} is a built-in framework for writing unit tests. \emph{XCTest} asserts that during code execution, certain conditions are satisfied and if not, the errors messages will be shown together with the test failure result. 
The full code for the unit test can be found in Appendix A. In Figure \ref{fig:testdownload}, the main part of the code is shown. In this test function, the URL is given with a valid protein ID (6K03) and the code checks if the file 6K03.pdb exists in the \emph{Document directory} after the download function completes. The green tick on the function shows that the tested function (download) works. 

 \begin{figure}[!htp]
	\centering
	\includegraphics[width=\textwidth]{images/testdownload.png}
	\caption{Unit Test - Download function}
	\label{fig:testdownload}
\end{figure}

\section{Performance Testing}
Since ProteinAR is an iPhone app, evaluating how the app performs on an iPhone is important. Xcode has a built-in debug navigator to show how the app performs on the device. In this navigator, there are reports to visualise how the application impacts the running of a simulator device. 
Figure \ref{fig:cpu} shows the impact on iPhone's CPU while the app is running. The CPU percentages fluctuates frequently, though it maintains a range between 80 and 120 percent. The testing device is iPhone X with six cores, bringing the maximum CPU capacity to 600 percent. Since the app performs many tasks at the same time (recognising the real-world's surfaces, putting layers on, downloading from the web, displaying models, etc.,), this is considered acceptable. In Figure \ref{fig:cpu}, the percentage used is shown to remain in the green zone. 
 \begin{figure}[!htp]
	\centering
	\includegraphics[width=\textwidth]{images/CPU.png}
	\caption{CPU usage}
	\label{fig:cpu}
\end{figure}

As for memory, the app has low memory consumption. In future work, when a conversion function is made and the app can display models from downloaded PDB files, memory will not be a problem as the data will be stored in \emph{Document directory}. A visualisation of memory usage is shown in Figure \ref{fig:memory}.

 \begin{figure}[!htp]
	\centering
	\includegraphics[width=\textwidth]{images/memory.png}
	\caption{Memory usage}
	\label{fig:memory}
\end{figure}

An AR app should have a frame rate of at least 30 FPS to allow the app to run smoothly and save CPU and GPU usage. In ProteinAR, the frame rate is relatively high, at 60 FPS. This creates smooth movement, but also requires substantial GPU usage (Figure \ref{fig:FPS}). This might also lead to extra energy usage, negatively affecting the energy impact of the app (Figure \ref{fig:energy2}). 

 \begin{figure}[!htp]
	\centering
	\includegraphics[width=\textwidth]{images/FPS.png}
	\caption{GPU Usage}
	\label{fig:FPS}
\end{figure}

 \begin{figure}[!htp]
	\centering
	\includegraphics[width=\textwidth]{images/energy2.png}
	\caption{Energy impact}
	\label{fig:energy2}
\end{figure}

Through observation, when the app starts running, the energy impact is already stated in the report as ``Very High". The thermal state starts at ``Fair" and in only a few minutes changes to ``Serious". Sometimes, the thermal state increases to ``Critical" if the app is left running. 
This negatively affects app performance, and drains the device’s battery faster than normal. If the device overheats, loaded models lag, and extra screens (i.e. the “Help Screen”) fail to appear when clicked due to the excessive heat generated by the UI elements.
The performance evaluation is summarised in table \ref{tab:perEvaluation}.

\begin{table}[!h]
\centering
\begin{tabular}{| m{0.15\textwidth} | m{0.15\textwidth} | m{0.25\textwidth}| m{0.25\textwidth}|}
\hline
Category & Device impact & Advantages & Disadvantages \\
\hline
\hline
CPU/GPU& High & Multi-tasking, smooth transition & Increase energy impact  \\
\hline
Memory & Normal & No extra workload on the device & N/A \\
\hline
FPS & High & Smooth display of models and AR layers & Increase CPU and GPU usage \\
\hline
Energy & Very High & N/A & Freeze the app and Drain battery \\
\hline
\end{tabular}
\caption {Performance Evaluation}
\label{tab:perEvaluation}
\end{table}

To help lower the energy impact, more testing needs to be implemented in future work. 

\section{Application Usability Testing}

Usability evaluation is an important evaluation that any system must take before product release. This helps ensure the app meets the requirements of a business and secures customer satisfaction. By doing usability testing, the app can be improved based on user feedback. 

In this project, the usability evaluation was conducted based on the post-testing usability questions. There are ten questions, adapted from the SUS (System Usability Scale) questionnaire. This means the questions are asked after users have had experienced using the app. 
However, the project ran into two problems while conducting usability testing. 

\begin{itemize}
\item Limitation of testing subjects: 

For users to remotely download and use the app, ProteinAR has to be available on the App store. After an app is submitted to the App store, it moves through a very strict review process and may not be available for a period of time. This would require more work to complete the app since an app with remaining issues would not make it through the review process. Furthermore, uploading an app on the App store categorises it as a commercial product which may go against UCC policy. Therefore, it was not possible for other users to remotely test the app. Alternatively, there are two methods for the app to be tested by other users which require in-person meeting:
	\begin{itemize}
		\item Users directly use the app on developer's device or downloads from the developer's computer. With the ongoing complications of Covid-19, this was not advisable.
		\item User can install the app by downloading the project on Github. However, this requires target users to have access to a MacOS system and an iPhone 6 or onwards with the compatible MacOS and iOS version.

Since the second method was difficult to conduct, all the tests were carried out in the first method, Due to the previous mentioned difficulty of Covid-19, the number of users who conducted the test was limited to five people.

	\end{itemize}
\item Lack of objectivity in questionnaire result:

For all tests, the users and developers were in the same room. This removed the anonymity of the test introducing bias to the results.

\end{itemize}

The tests were conducted despite the above limitations, and the questionnaire results are shown in Figure \ref{fig:survey}. Prior to the test, users were provided protein names to start with. Since there are limited buttons on the screen, it did not take much time for users to learn how to navigate and use the app. The element of using Augmented Reality impressed users as it is new and considered ``exciting" and ``cool", which was a major factor in increasing user motivation for continued use. However, the app is about proteins, which might not be the subject of choice for most users. In effect, users indicated that they would not recommend the app to others, suggesting that the entertainment element does not meet with user expectations. 

 \begin{figure}[!htp]
	\centering
	\includegraphics[width=\textwidth]{images/survey.png}
	\caption{System Usability Scale (SUS) Questionnaire and Feedbacks}
	\label{fig:survey}
\end{figure}

The extra functions linking the app to RCSB website or PDB 101 were not intentionally chosen. Users indicated that reading information on another website required undesired effort. This could be replaced by simpler but educational screens or websites. The AR triggered excitement in using the app, and therefore, should be the main focus for future development. The mini-game would be more interesting if the new protein created could be linked to daily life such as ``found in human skin, found in sheep wool, etc.". 

\section{Overall Evaluation}
Based on the tests conducted, the overall project evaluation is summarised in table \ref{tab:evaluation}.

\begin{table}[!h]
\begin{tabular}{| m{0.15\textwidth} | m{0.4\textwidth} | m{0.25\textwidth}| m{0.15\textwidth}|}
\hline
GOALS & IMPLEMENTATION & ACHIEVED RESULT & EVALUATION\\
\hline
\textbf{Download Protein from}

\textbf{RCSB PDB} & 
\begin{itemize}
	\item Download using \emph{downloadTask()} using constructed URL
	\item Save files to Document directory
	\item Assign attributes to Core Data
\end{itemize} &
Based on the result from Unit Testing, these functions work without errors.&
\textbf{Completed}\\
\hline

\textbf{Visualise Protein models} 

\textbf{on AR} & 
\begin{itemize}
	\item Convert PDB file to Collada files
	\item Visualise by loading 3D models on AR screen using function \emph{displayProtein} (Figure: \ref{fig:displayreal})
\end{itemize} &
\begin{itemize}
	\item Conversion function was not completed
	\item Can visualise pre-downloaded sample models
\end{itemize}&
\textbf{Partially}

\textbf{completed} \\
\hline

\textbf{Create}

\textbf{new Protein} 

\textbf{on AR} & 
\begin{itemize}
	\item Display individual polypeptide chains with function \emph{addProtein()} (Figure: \ref{fig:addProteinfunc})
	\item Display combinations of polypeptide chains according to user input using function \emph{createProtein} (Figure: \ref{fig:createProtein}) 
\end{itemize} &
Displaying individual polypeptide chains and clearing them to display the combination creates a smooth transition and interactive experience for the user.&
\textbf{Completed}\\
\hline

\textbf{Interact with} 

\textbf{Protein models} &
 Use UIGestureRecognizer for 
 \begin{itemize} 
	 \item Pinch Gesture (allow zooming in and out) (Figure:\ref{fig:pinch})
	 \item Rotation Gesture (allow rotating models) (see the Appendix A)
	 \item Pan Gesture (allow moving models) (see the Appendix A)
\end{itemize} &
Based on the user testing questionnaire, the use of interactions is satisfactory.&
 \textbf{Completed}\\
 \hline
 
 \textbf{Appealing UI} &
 \begin{itemize}
 	\item Use two different screens for two different purposes (education and mini-game)
	\item Easy to navigate
	\item Analogous colour theme
	\item Interactive elements (buttons, display, model-interactions, create new proteins) (Chapter: \ref{ch:design})
\end{itemize} &
Based on the user testing questionnaire, the UI is easy to use and appealing. &
\textbf{Completed}\\
\hline

\textbf{Good}

\textbf{performance}&
 \begin{itemize} 
	 \item CPU Usage: High
	 \item GPU Usage: High
	 \item Memory Usage: Normal
	  \item Energy Impact: Very High
\end{itemize} &
The thermal state reaches “Serious”  after a short period of use, which is not ideal for the app. Further tests are needed to solve this problem. &
\textbf{Partially}

\textbf{completed}\\
\hline
\end{tabular}
\caption{Overall Evaluation}
\label{tab:evaluation}
\end{table}
Although much future work is desired for the completion of the app, ProteinAR succeeded in creating the first step to bringing protein visualisation to AR. Without the need to print materials, or use extra devices such as VR goggles, ProteinAR can be developed to become a useful tool for students, teachers, and scientists in the study and research of proteins.



\chapter{Conclusion}
\label{ch:conclusion}

\section{Conclusion}
\section{Future Work}
Given the limitations of the project, there are plenty of rooms for improvement with ProteinAR. Moreover, Augmented Reality technology is relatively new and Apple has been acquiring new companies specialising in AR to update the ARKit framework rapidly, thus, possibilities for the development of the app in the future are promising.

First and foremost, the \textbf{protein real-time visualisation} is not completed because it is missing a function to convert PDB file type to Collada file type. With this function completed, ProteinAR could become useful in biology class as it displays any protein's structure in real time. This should be the main focus of the future work.

Secondly, the \textbf {new protein creation} can be improved in the following directions:
\begin{itemize}
	\item Currently, the combinations of polypeptide chains are pre-loaded in a folder. In the future, if this too can be generate in real time, using server such as I-TASSER, there would be more combinations that are created, and thus, not only tertiary but quaternary structures can also be generated.
	\item The newly created protein, if not new, should also be linked to some information such as its parameter, its function, etc. This can be achieved using POST and GET REQUEST to the available source of information.
	\item The newly created protein should be exportable into a PDB file. This way, it could be used more for research purposes.
\end{itemize}

Thirdly, more \textbf{interactive elements} can be added. ProteinAR can integrate Machine Learning and CoreML to control ARKit. There are many ways to implement this. One of the popular way is to combine image classification and AR to create new experience by experimenting with hand gesture recognition. Photos of hand poses can be taken then used in training models such as TensorFloe, Keras, Custom Vision. Xcode also has its training interface. The models can be trained so that users can use hand poses to control the protein models. Moreover, the interactions between users and the protein models could become more realistic if the connection points in the models can be bended and twisted. Currently, ProteinAR only allow user to pinch, rotate and pan the models.
 
Last but not least, \textbf{more functions} can be integrated into the app. The menu function is now only link the app to RCSB home page, with no specific information. These functions can be more customised to be suitable with the purpose of usage. 
With the development of AR technology, ProteinAR can be improved much more and might become a useful tool for biologists and biology students. 






\printbibliography
\appendix
\label{ch:appendix}

\chapter{Code snippets}
\doublespacing

\begin{figure}[!htp]
	\centering
	\includegraphics[width=\textwidth]{images/displayTextfunc.png}
	\caption{Function to display 3D Text}
	\label{fig:displayTextfunc}
\end{figure}

\begin{figure}[!htp]
	\centering
	\includegraphics[width=\textwidth]{images/rotate.png}
	\caption{Rotation Gesture}
	\label{fig:rotate}
\end{figure}

\begin{figure}[!htp]
	\centering
	\includegraphics[width=\textwidth]{images/pan.png}
	\caption{Pan Gesture}
	\label{fig:pan}
\end{figure}

\begin{figure}[!htp]
	\centering
	\includegraphics[width=\textwidth]{images/testdownloadfull.png}
	\caption{Full test script for unit test}
	\label{fig:testdownloadfull}
\end{figure}
\end{document}

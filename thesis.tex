% !TEX TS–program = pdflatexmk

\documentclass[MSCIM]{mscim}
\usepackage{graphicx}
\usepackage[english]{babel}
\usepackage{csquotes}
\usepackage[style=apa,doi=false,isbn=false,url=true,eprint=false]{biblatex}
\usepackage{hyperref}
\usepackage[utf8]{inputenc}
\usepackage{float}
\usepackage{array}
\usepackage{tabularx}
\usepackage{appendix}
%\usepackage{apacite}

\addbibresource{refs.bib}

\begin{document}
\DeclareGraphicsExtensions{.pdf,.png,.gif,.jpg}

\title{\textbf{ProteinAR} \\an interactive ios application\\for protein visualisation and design in augmented reality}

\author{Thao Phuong Le}
\principaladviser{Sabin Tabirca}

\beforeabstract

\prefacesection{Abstract}
Proteins are complex molecules with various functions that are critical to any living organism which comes in all different shapes and sizes. The shape of a protein defines its function, therefore, protein structure study is of great importance, since it can help scientists to control or modify the protein to change its function and help with various disease treatments. 

In this project, ProteinAR - an iOS application was developed. The goal of the project is to integrate technology into biology to aid with study and research by visualising protein structures and enable interactions with these structures in Augmented Reality. ProteinAR aims to download protein data and display it in real-time. RCSB PDB is the main source of protein data for this project. ProteinAR downloads protein data from RCSB when a protein ID is inputted. 

ProteinAR succeeded in visualising protein structures on AR screen and also in allowing users to interact with these structures. Using ProteinAR, users have the freedom to observe even the most complex protein structures by zooming in or rotating them by interaction with the phone screen instead of looking through a microscope. Moreover, new protein structure creation is another key function of ProteinAR. Users can combine the polypeptide chains (flex coil, rig coil, helix, sheet) to create new protein models and interact with them. 

ProteinAR was written in Swift on Xcode, a native language and IDE for iOS apps developed by Apple. ProteinAR integrated the framework ARKit for Augmented Reality functions. 

Even though further developments could be done to complete the app, ProteinAR is a successful first step in bringing the AR technology into protein visualisation and design.

\afterabstract

\prefacesection{Acknowledgements}


 
 
\afterpreface

\chapter{Introduction}
\label{ch:intro}

In recent years, there have been major advances in technology and molecular biology. Technology has been of great use in the field of biology aiding in research as well as making the content more accessible. This project focuses on the display and design of protein structure.

A protein is not a single substance; there are many different proteins in an organism or cell, and they come in every shape and size, each performing a unique and specific job \parencite{noauthor_introduction_nodate}. Proteins are considered the ``ultimate players in the processes that allow an organism to function and reproduce'' \parencite{stephenson_protein_2016}.

Proteins are made up of linear chains of amino acids, called polypeptides. Each protein is formed by one or more polypeptide chains, linked together in a specific order \parencite{noauthor_introduction_nodate}. Proteins are the fundamental components of all living cells \parencite{hutchison_protein_2013}. Proteins have a countless number of functions that are extremely important in the biology of many organisms. They form enzymes to speed up reactions by break-down, link-up, or rearranging the substrates \parencite{noauthor_introduction_nodate}. They form hormones to control specific physiological processes such as ``growth, development, metabolism and reproduction'' \parencite{noauthor_introduction_nodate}. To maintain these roles, the shape of a protein is critical. If the shape changes, the protein will lose its functionality. There are four levels of protein structure: primary, secondary, tertiary, and quaternary \parencite{noauthor_introduction_nodate}.
Knowing the structure of a protein makes understanding of the function of that protein much easier. By being able to manipulate the structure of a protein, scientists can create hypotheses about how to affect, how to control or how to modify protein to design mutations and change a protein's function. 

The year 2020 substantiates the importance of studies in molecular biology. The impact of Severe Acute Respiratory Syndrome Coronavirus-2 (SARS-CoV-2) is worldwide, as it is ``a newly emerging, highly transmissible and pathogenic coronavirus in humans that has caused the ongoing global public health emergencies and economic crises'' \parencite{mittal_covid-19_2020}. At the time of writing, the number of infections worldwide has reached millions, while the death toll is in the hundreds of thousands. In the efforts to develop a vaccine, much research has been conducted. Some research developed on the protein structure of SARS-CoV-2 has provided insight into its evolution. As Wiesława has pointed out: ``The chief characteristic of proteins that allows their diverse set of functions is their ability to bind other molecules (proteins or small-molecule substrates) specifically and tightly.'' \parencite{hutchison_protein_2013}, particular to SARS-CoV-2 are the protein spikes that the virus uses to bind with and enter human cells \parencite{wrobel_sars-cov-2_2020}. The spikes of SARS-CoV-2 are highly stable, and help to bind to human cells tightly. Therefore, analysing the structure of these spikes could provide clues about the virus’s evolution. The study of the structure of spike proteins can aid with drug discovery and vaccine design. 
Understanding the new importance of implementing IT in biological research, \textbf{this project aims to aid with protein structural study as well as to raise interest in protein design.}

Due to limitations, this project only provides the first steps in bringing the visualisation of protein into AR and creating a simple protein structure in an iOS application using ARKit framework. \textbf{The main goal of this project, however, is to visualise protein structure in AR using an iOS App and allow user interaction with the structures.} 
There are various previous studies on protein visualisation in 3D and VR, however, studies pertaining to AR are limited, especially on iOS. This project proposed the implementation of protein structures display to serve as a trial for future study and research as it may provide more a visually appealing display than simple 3D, and reduces the side effects of VR. All protein models to be displayed are retrieved from \href{https://www.rcsb.org/}{RCSB Protein Data Bank}. 

The aim of this project is to visualise protein structures in two ways. The first way is directly displaying the complex protein models from RCSB PDB, and the second way is visualising the design of a simple protein structure that user created. The second function is implemented so that this project's application is not only appealing to biologists but can also be used by anyone curious about biology. The ability to construct a protein structure as a mini-game might make it easier for users to understand more about protein structure. 

In this project, a mobile application for the iOS system was developed: \textbf{ProteinAR}. This app has two main categories: \textbf{education} and \textbf{mini-game}. 

The \textbf{education} category assumes that the users have prior knowledge of proteins. They can input the name (ID) of a protein and get the 3D visualisation of the protein structure in AR. Users can study the protein by zooming in, turning, and flipping the protein structure. Due to the complexity of protein structures, it is not easily observed even under advanced microscopes. Thus, the ability to interact with the structures in this way can be largely beneficial to researchers. Moreover, since this is a mobile app, users can interact and discuss the structure with other users at the same time, which can be considered a promising tool for study and research on proteins.

The \textbf{mini-game}, is user-friendly and accessible to those who are unfamiliar with proteins or biology in general. Users can combine the polypeptide chains (i.e. Flex Coil, Rig Coil, Helix, Sheet) to create a new protein. This might make the study of proteins sound more appealing to users. and motivate those who wish to enter the field. In this game, users are also able to interact with the polypeptide chains and protein models. 

Last but not least, ProteinAR integrates other functions to make the app more interesting, such as enabling photo-capture of the proteins, video-capture of the process, and providing users with additional information about protein.

This paper will elaborate on the background and research, the problems and solutions, the design and implementation, and the final evaluation of the project.Due to time constraints as well as the ongoing pandemic, there were some limitations to the project, which would also be mentioned in the paper. 

Finally, this paper will discuss some critical points in dealing with the fairly new AR technology, especially use of the ARKit framework concerning: the feasibility of retrieving and displaying PDB contents, the usability of the app (AR) and room for future work.

In this paper, Chapter \ref{ch:litRev} presents findings from a literature review on AR and the usage of AR technology in education and research, especially in the biological field. In this chapter, some current apps on protein visualisation will be mentioned with personal insights. Chapter \ref{ch:methodology} introduces the technological background of software and language used in developing ProteinAR. Chapter \ref{ch:analysis2} proposes the requirement functions as solutions for the current problems while Chapter \ref{ch:design} demonstrates the app's structural design based on these solutions. The implementations of this project can be found in Chapter \ref{ch:implementation}, along with code snippets from the source code and explanations. After that, in Chapter \ref{ch:evaluation}, the conducted tests' results will be exemplified with an overall evaluation of the project. Finally, Chapter \ref{ch:conclusion} encapsulates the project and outlines the directions for future work. Some code snippets can be found in Appendix A.

Some important technical notes about the project:
\textbf{ProteinAR} was designed in Xcode 12, written in Swift 5, on MacOS version: Catalina 10.5.5. Because there is no support for AR on MacOS, the built-in simulator is unable to display AR functions and can cause some other errors. The project was run and tested on an iPhone. The attached demo video is recorded on iPhone X, iOS version 14. Other versions of Xcode or macOS or iOS might be unable to run ProteinAR and might also generate some unwanted errors. 


\chapter{Literature review \& Existing products research}
\label{ch:litRev}

This chapter goes through some literature on Augmented Reality (AR) and the application of AR in education, with the specific mention in the biological field. Then, a brief explanation of protein structure will be provided, follow by research on the existing solutions of protein structure display in mobile application, Virtual Reality (VR) and AR with some personal insights. Finally, the finding summary is presented.

\section{Augmented Reality and application in education}
\subsection{Augmented Reality introduction}
Augmented Reality (AR) is a technology that involves “the overlay of computer graphics on the real world" \parencite{silva_introduction_2003}. AR allows users to look "at the real world and increases it with additional information generated by a computer" \parencite{chamba-eras_augmented_2017}. AR acts as a bridge, connecting physical and virtual objects, combining two worlds into one, and enables interaction between them by adding information to the real-world in real-time \parencite{chamba-eras_augmented_2017}.In AR, the user is able to stay in touch with both contexts of the real world and the virtual world.

On the other hand, even though often being mistaken for another, Virtual Reality (VR) is used to define the technology that allows the computer to generate 3D environments that users can enter and interact  \parencite{silva_introduction_2003}. The complete scenarios are generated by computers, creating the sensation of being physically in the generated scene for the user. In VR, users lose the context of the real-world but instead, are just aware of synthetic realism.

AR is similar to VR in the way that virtual objects are generated by the computer. However, while the goal of VR is to create an immersive experience for the user by shutting down the real physical world and replace it with a completely synthetic environment, the goal of AR is to enable the user to stay in touch with the real-world, while being able to interact with virtual objects \parencite{chamba-eras_augmented_2017}. Because of this, the defining characteristic of AR is that it adds layers to the real-world vision. Using AR, users have more freedom of movements while projecting images. The main components of AR are scene generator, tracking system, and display. The scene generator is responsible for rendering the scene for binding real-world scenes and virtual objects. The tracking system is important because the objects in the real and virtual worlds must be “properly aligned with respect to each other, or the illusion that the two worlds coexist will be compromised" \parencite{silva_introduction_2003}. As for the display, currently, there two well-known types of display for AR implementation. The first one is the implementation of AR smart-glasses such as the Microsoft HoloLens, Google Glass, Apple Glass. In contrast to VR goggles, AR smart-glasses look similar to sunglasses or normal glasses, thus, causing no discomfort to the users. The second type of implementations is on AR apps such as Pokemon Go. In this type of implementation, smartphone cameras are used to track the surrounding environment and digital models are “superimposed into the real-world" \parencite{moro_effectiveness_2017}.

Using an AR-implemented-app does not only have the advantage of allowing the user to interact with both the real and virtual elements of their surrounding environment but also in this way, no extra equipment is required, AR app has become more and more common, especially in the field of education \parencite{moro_effectiveness_2017}.


\subsection{Augmented Reality in Education}
There have been multiple articles promoting learning through AR. The study from the University of Cologne discussed the benefits of AR in educational environments with a conclusion that applications applying successful use of AR have been improving learning, especially in language education, mechanical skills, and spatial abilities training \parencite{diegmann_benefits_2015}.The study from the University of Girona analysed 32 studies from journals about the AR trend in education. The finding results were that the number of published studies about AR in education has progressively increased year by year, while the fields of education which had the most AR applications are Science, Humanities, and Arts, in which AR has been effective for better learning performance, learning motivation, student engagement and positive attitudes because the advantage of AR is allowing interactions and collaboration \parencite{bacca_augmented_2015}. The study by the University of La Laguna also concluded in the higher performance of students studying using AR applications as they can do it in their own time \parencite{martin-gutierrez_augmented_2015}.There were also some specific proposals on how to integrate AR in learning. Brown and Gabbard proposed using AR to personalised learning for every student \parencite{brown_interactive_2015} while Huang and his team came up with a learning model based on AR, in which educational resources are discovered on the internet and be translated to AR for an educational environment to improve students’ emotions and experiences \parencite{huang_animating_2016}.

Molecules studying can be confusing. In 2019, a study conducted using AR for leaning atoms and molecules reaction by students was conducted. In this study, female students, who do not show much interest in science and technology were the target of research \parencite{ewais_usability_2019}. The study found that using AR technology to visualise the molecules did motivate these students to study, as it helps them understand the structure much better. Another study conducted using an augmented reality web application for high school education in biomolecular life science also discussed the fact that it is difficult to understand the spatial relationship of a protein structure. “Proteins and protein interactions are too small to be seen by far, even under advance microscopes" \parencite{nickels_proteinscanar_2012-1}. However, by using this AR web-based app, students show much more interest and understanding of the field \parencite{nickels_proteinscanar_2012-1}. Another study from Georgia Gwimmet College in 2018 about the developing of an AR app that transforms 2D molecular representations into interactive 3D structures that user can manipulate also showed that students who have used augmented reality models found it convenient and faster than the traditional mode, and they also prefer it more as they have control over molecular manipulations \parencite{behmke_augmented_2018}.

These studies showed above showed how important it is for AR to be brought into molecular biology study.

\section{Protein Structure}

In the field of biology, researchers give prioritised attention to the shape of a protein. Proteins are “the most important macromolecules in all living organisms” \parencite{rashid_protein_nodate}. Sequences of amino acids that bind into linear chains create proteins. These chains have a specific folded three-dimensional (3D) shape, which enables the protein to perform a certain task \parencite{rashid_protein_nodate}. The shape of the protein defines its tasks, thus, knowing the protein structure is very important. This task is defined by the shape of the protein, which makes the understanding of protein structure of great importance. There are four different levels of protein structures: Primary Structure, Secondary Structure, Tertiary Structure, and Quaternary Structure. A sequence of amino acids in a chain forms a \emph{Primary structure}. These chains, then, would fold into three different shapes (Helix, Coil or Sheet) where the alpha-helix, the beta-sheet, and the random-coils are positioned, which is called the \emph{secondary structure}. The combination of these chains of helix, coil and sheet (polypeptide chains) forms a 3D structure – the \emph{tertiary structure} of a protein \parencite{rashid_protein_nodate}. The \emph{quaternary structure} is a large assembly of multiple polypeptide chains (Figure \ref{fig:proteinstr}).
 \begin{figure}[!htp]
	\centering
	\includegraphics[scale=0.5]{images/proteinstr.png}
	\caption{Orders of protein structure - source \href{https://www.khanacademy.org/science/biology/macromolecules/proteins-and-amino-acids/a/orders-of-protein-structure}{Khan Academy}\parencite{noauthor_introduction_nodate}}
	\label{fig:proteinstr}
\end{figure}

Given the importance of protein structure, designing proteins is extremely useful as this can help to change its functions, or create new functions. In ProteinAR, users will get to design protein structure by combining different protein secondary structures of helices, coils, and sheets to form tertiary structures.

\section{Existing solutions to protein visualisation}
 “Proteins are three-dimensional (3D) objects” \parencite{ratamero_touching_2018}. Computer models for protein have become very popular for a long time. Many projects were developed to make 3D viewing of protein possible such as {\footnotesize PYMOL, CHIMERA, VMD, ISOLDE,} etc. 
 
 \subsection{Protein visualisation in mobile applications}

There are numerous mobile applications in which proteins are visualised in 3D. The RCSB Protein Data Bank (the single worldwide repository of protein data) also provides a \href{https://www.ncbi.nlm.nih.gov/pmc/articles/PMC4271143/}{mobile app} allowing data access and visualisation. The protein can be downloaded directly from the PDB from RCSB and displayed in 3D. This app is based on the open-source molecular viewer \href{https://play.google.com/store/apps/details?id=jp.sfjp.webglmol.NDKmol&hl=en}{NDKmol}. However, NDKmol can only be used on Android and not iOS. \href{https://www.imedicalapps.com/2013/08/jmol-molecular-visualization-app/}{Jmol}l is another Android app that connects to the RCSB PDB, and visualises proteins in 3D.
There are some molecule viewers available on iOS. Unfortunately, most of them are no longer in use or experienced technical difficulty, and have therefore been removed from the Apple App Store. \href{https://www.molsoft.com/iMolview.html}{iMolview} is still available, however, the interface is not very user friendly. 


\subsection{Protein Visualisation in VR}
\subsubsection{The advancement of implementing VR in Protein Display}
Visualisation of proteins on the computer has been a great step, however, it lacks the immersive salience of 3D presence, and leads to limitations in the analysis of protein structure. Virtual Reality (VR) provides a wide field of view on an immersive display and a better perception of the protein structure by head-tracking. Furthermore, VR enables users to have the freedom of hand controllers for simple manipulation and interaction with the protein instead of the conventional manipulation on 2D using a trackpad, mouse, and keyboard \parencite{goddard_molecular_2018}. This makes VR's entrance into the world of molecular biology and protein visualisation more than welcome. 
HMDs\footnote{Head Mounted Display} are commonly used because they are accessible, becoming increasingly more available, and are affordable. VR games have become popular, thus the tools for programming software that are compatible with HMD are effective and cheap. Projects such as {\footnotesize REALITYCONVERT}, {\footnotesize AUTODESK}, {\footnotesize MOLECULE VIEWER} are well developed, providing good resource for further development on protein display in VR \parencite{ratamero_touching_2018}. {\footnotesize UNITY} is largely used with the combination of HMDs such as {\footnotesize OCULUS RIFT} and {\footnotesize HTC VIVE} to display and manipulate proteins \parencite{ratamero_touching_2018}.

\begin{figure}[!htp]
	\centering
	\includegraphics[scale=0.6]{images/OculusRift.png}
	\caption{Oculus Rift (HMD) and Kinect v2 sensor placement used during Molecular Rift development}
	\label{fig:OculusRift}
\end{figure}


There have been many advanced VR projects in molecular biology. In particular,  {\footnotesize MOLECULAR RIFT} is an open source tool that creates a virtual reality environment steered with hand movements, and incorporates {\footnotesize OCULUS RIFT} as the display to create the virtual setting \parencite{norrby_molecular_2015}. The combination of a virtual reality experience with natural acts such as hand movement creates a much better experience for the users than merely experiencing the 3D \parencite{norrby_molecular_2015}.

Though research shows that the technology in displaying Protein in VR is advanced, the tools that need to be installed on desktop systems are often tedious \parencite{xu_vrmol_2019}. The configurations might be different for different systems and therefore cause compatibility issues. Sharing between system is also difficult. With the help of Web Graphics Library (WebGL), web-based applications such as {\footnotesize JMOL}, {\footnotesize ASTERVIEWER} are more straightforward as VR experiences can be directly accessed with common web browsers. However, there are many limitations for these web-based applications because they only support a few file types and cannot perform complex tasks for analytical purpose \parencite{xu_vrmol_2019}. A few solutions were proposed for an integrative cloud-based system that can directly access databases and uses VR technology to visualise and analyse macromolecular structures, such as {\footnotesize VRMOL}. This might be the new direction for protein visualisation in VR.

\subsubsection{The limitations of using VR in Displaying Protein}

Even though VR implementation of protein display has come far, limitations are inevitable. First, the limitations in the associated hardware/software may lead to an unsuccessful application of VR, which leads to the inaccuracy and impreciseness in the results of using the application. With the increasing development of VR techniques and the growing popularity of VR games, software and hardware are becoming more integrated with VR and therefore more compatible, however, they are still costly and need to be increased in fidelity \parencite{liu_using_2018}.
The second point concerns the unnatural feeling of using VR. Even though VR offers a realistic view, the users must wear goggles that are not transparent and thus block the vision of the real world. Furthermore, the head movements are unnatural because users will have to try to move their heads to see contents. New HMDs are better because they are much lighter but mostly VR devices are still quite bulky and are relatively difficult to use. Thirdly, most VR users claim to have motion sickness. This happens because of the disparity between what the body and the eyes of a user experiences at the same time. The actual physical actions and the actions that are carried out in VR might be different and this causes motion sickness to the users. Due to this, VR can only be used for a limited amount of time.


\subsection{Protein Visualisation in AR}

As AR gains popularity, more projects are underway, but this is limited as it is an extension of VR, and it is still very new. Some studies show that AR being used in science teaching such as displaying molecular biology in AR has yielded in good results for students, as it takes less imagination and makes things easier to understand \parencite{cai_case_2014}.However, there are not many AR apps available to support visualising molecules. 

As mentioned, there are not many projects concerning the visualisation of molecules on AR. Unlike VR, where there are various numbers of HDMs incorporated software and app for protein visualisation, on AR, apps are more commonly used. There are only a few apps that can be found. BiochemAR is one of those. One such app, BiochemAR, was released in 2019 and is available on both the App Store (for iOS) and Google Play (for android). According to the developers, the idea of the app is to create a simple, easy-to-use teaching tool for both teachers and students in the classroom\parencite{sung_biochemar_2020}. The main function is to display protein in AR by scanning a QR code, thus the design is relatively basic. When a QR code is scanned, the app will use the smart devices’ built-in camera to bring the protein structure into life through VR as shown in Figure \ref{fig:bioChemAR}.
\begin{figure}[!htbp]
	\centering
	\includegraphics[scale=0.5]{images/bioChemAR.png}
	\caption{BiochemAR app screen shot}
	\label{fig:bioChemAR}
\end{figure}

As the main purpose is to make things simple and easy to use for teachers and students, there is no other function or interaction between users and the protein. Proteins are simply visualised and users can move the phone around to look at the protein in different angles and
sizes. 

Having the same idea, another app called AR Assisted Visualisation was developed in 2020 to visualise proteins. These proteins are not written under QR code form but instead printed out on paper as in Figure \ref{fig:arvisualisation}. 

\begin{figure}[!htbp]
	\centering
	\includegraphics[scale=0.8]{images/arvisualisation.png}
	\caption{AR Assisted Visualisation App \parencite{eriksen_visualizing_2020}}
	\label{fig:arvisualisation}
\end{figure}

Similar with BioChemAR, AR Assisted Visualisation only display protein structure in 3D, without any interacting elements. 

\subsection{Personal insights}
Integrating protein visualisation on mobile apps is a good solution because of its availability. Most students have access to a smartphone and it is handy to bring around as it is not bulky nor need specific customisation. However, the AR apps on protein visualisation are relatively
new (released in 2019 and 2020). Thus, there are not many user interactions and functions to it. To use the aforementioned apps, a certain document with information of the protein, whether it is a figure of a protein or a QR code, has to be printed in order to get the AR visualisation.
Moreover, the proteins can be viewed but cannot be manipulated in any way. Furthermore, these apps are one-side oriented as users can only view proteins but cannot create new ones.

\section{Finding summary}
With the advantage of allowing the user to stay in touch with the surrounding environment and not requiring any extra equipment, the use of AR in mobile applications is certainly gaining popularity. In education, the use of AR technology motivates students and establish better performance. In molecular biology, according to many studies, students are much more interested and have a greater understanding when AR technology is used in teaching. In a recent research, it showed that when undergraduate students created their own AR-protein, they were enthusiastic when performing this function, thus, their learning was enhanced when the AR module was inserted to their upper-level biochemistry class \parencite{argu_fast_2020}. With the trend of online learning, the application of AR offers a promising curriculum for biochemistry. Not only with education, but the integration of AR in molecular biology can also benefit researchers as it makes small interactions which are invisible under microscope visible. Currently, many applications are being developed to visualise protein structures, however, there are still limitations. Nevertheless, in the near future, AR technology will become much more commonly used in molecular biology. 

ProteinAR's purpose is to not only let users directly view the shape of a protein in AR, or interact with the protein by gesture touch on the screen, but also allow users to design and create their proteins. The majority of mobile apps to visualise protein are only in 3D, and mostly on Android. Therefore, the open-source API for protein visualisation directly from the PDB files are limited. This project will have to start with little availability in pre-developed techniques.

Based on the above findings, this project aims to provide a valuable tool for the field of biology with the experimental app ProteinAR. With further development, it can be applied as an education tool, and for researchers in need of accessible and accurate protein models.






\chapter{Methodology}
\label{ch:methodology}

ProteinAR is an app designed to run on an iOS system. It was written in Swift 5, on Xcode. The dataset in which protein files are downloaded from is directly connected to RCSB PDB. There were some other sources of protein data websites such as \href{https://web.expasy.org/protparam/} {Protein Parameter} or \href{https://zhanglab.ccmb.med.umich.edu/I-TASSER/}{Protein Structure Function and Prediction I-TASSER Server} were used to test out the application during the process of making.
The app only runs fully on an iOS device, not a built-in simulator due to the requirement to use the camera to achieve the AR function.

\section{Softwares used}
	\subsection{Xcode}
Xcode is an integrated development environment (IDE) for MacOS. It was first released in 2003, enables developers to create apps for Apple platforms. Xcode supports sources codes for various programming languages including C, C++, Objective-C, Swift, etc. Xcode has a built-in Interface Builder to construct graphical interfaces. 
During the making process, Xcode has a few version upgrades. The latest update was Xcode version 12. With every version updates, there are few changes in codes and functions as the main goal is to build more compact and user-friendly interfaces.
		\subsubsection{Advantages of using Xcode}
ProteinAR is written on Swift, a native language for iOS apps, released by Apple and since Xcode is the native IDE of Apple, the compatibility is perfect, making the app and tests run faster and less errors. Xcode is a highly intuitive IDE where there is the main storyboard interface, visualising the designs elements of an app, with various built-in function to customise the design, from background colours to framing and a built-in library for easy adding and changing elements such as icons, pictures, text labels, etc (Introducing Xcode 12, n.d).
		\subsubsection{Disadvantages of using Xcode}
ProteinAR used the built-in ARKit package. As this requires camera accessibility, tests cannot be run on the built-in iPhone simulators but instead, a real iPhone device. This creates a great disadvantages as iPhone iOS version keeps on updating, and thus, being incompatible with Xcode if Xcode is not the up-to-date version, which means MacOS should always stay as the latest version. Xcode’s disk size is large, thus, downloading takes a great amount of disk space and time. Moreover, in some updates, the packages supports change, meaning there might be some errors that needed to be fixed with the newer version. 

	\subsection{UCSF Chimera}
UCSF Chimera (or Chimera) is developed by the University of California. This program allows interactive visualisation of protein data. Once a PDB file is downloaded, Chimera can open the files in a 3D form and allow users to export the files in various types such as \emph{.dae, .x3d, .obj}.
	
	\section{Language used: Swift}
Swift is a powerful programming language for Apple platform. Apple released Swift from 2014, taking ideas from various other languages (Rust, Haskell, Ruby, Python, C, etc.,) but it bares most similarities to Objective-C  \parencite{noauthor_swift_nodate}. 
	\subsubsection{Advantages of using Swift}
Swift was always considered as one of the \emph{Most Loved Programming Language} on Stack Overflow for many years as it is highly interactive, with concise and expressive syntax which runs fast. There are several improvements comparing to other languages: there is no need for semi-colons, UTF-8 based encoding is used, Strings are Unicode-correct, etc. It is also designed for safety as by default, Swift objects can never be \emph{nil}. As a successor to C and Objective-C, Swift includes low-level primitives such as types, flow control and operators as well as object-oriented features such as classes, protocols, and generics \parencite{noauthor_swift_nodate}. Overall, Swift is a simple and straight-to-the-point coding language.
	\subsubsection{Disadvantages of using Swift}
As mentioned above, there were a few version updates of Xcode during the programming process. Swift is a new language, thus, are being changed constantly to reach perfection. Therefore, the syntax and packages might change from times to times. It is relatively new, therefore, solutions to coding problems might be too new to have an answer, which was the most challenging in using Swift as the main coding language. 

\section{ARKit API}
The technology to develop Augmented Reality was ready for mobile devices, however, it is too complex as algorithms for detecting objects in real world and displaying virtual object need to be created, and these are very complex for developers. This is why Apple released ARKit in 2017 as a software framework, making developing an AR iOS app so much easier. It is an API that supplies numerous and powerful features to handle the process of building Augmented Reality apps and games for iOS devices. 

Apple has been acquiring many AR companies, thus, the ARKit is built on all of these acquisitions. One of the major ones was the German company Metaio, which IKEA initially used to let customers display IKEA furnitures in their own home. Ferrari also used Metaio’s technology to allow customer changing colours of cars in showroom, and looking at car’s internal features. In 2017, Apple acquired SensoMotoric Instrument, a company specialized in eye tracking technology to use in AR. Other companies that specialized in other parts of AR technology are being acquired by Apples throughout the year. By doing this, the features of ARKit on iOS devices are frequently newly added and updated. ARKit is continuing to grow, making the creating of AR apps easier than ever \parencite{wang_beginning_2018}.

\subsection{Basic understanding of the ARKit}
There are three layers that works simultaneously in ARKit \parencite{noauthor_introduction_nodate-1} as shown in Figure \ref{fig:3Layers}.
\begin{figure}[!htp]
	\centering
	\includegraphics[width=\textwidth]{images/3Layers.jpeg}
	\caption{Three Layers to ARKit}
	\label{fig:3Layers}
\end{figure}

\textbf{Tracking} is the key function of ARKit. Without ARKit, it would be very complex for developers to write algorithm to track a device’s position, location and orientation in the real world.
\textbf{Scene Understanding} is the layer that allows ARKit to analyse the environment presented by the camera’s view to adjust and provide information in order to put place a virtual object on it. 
\textbf{Rendering} is the process where ARKIt handles the 3D models to put them in a scene such as SceneKit, Metal, RealityKit.

\subsection{Language and System Requirement for ARKit}
As mentioned in this thesis, since augmented reality requires access to cameras and high resolution display, ARKit apps can only be run on modern iOS devices: 
\begin{itemize}
	\item iPhone SE, iPhone 6s and later
	\item iPad 2017 and later
	\item All iPad Pro models
\end{itemize}
To develop an iOS app, Xcode is the best IDE to be used as it also has the built-in simulator program to mimic different iPhone and iPad models. However, with ARKit integrated, the app cannot be tested on the simulators but has to be on a real iOS devices listed above, connecting through its USB cable.
Both Swift and Objective-C can be used to create an ARKit app. This project chose Swift as the language because it is much easier to learn and run faster. 
ARKit framework allows developers to be able to focus on the features of the app rather than on the AR required technologies such as detecting, displaying and tracking virtual object in the real world. 


\section{Database used: RCSB}
ProteinAR downloads PDB files directly from \href{https://www.rcsb.org/}{RCSB}. 

PDB (Protein Data Bank) file format provides a standard representation for macromolecular structure data. These are obtained from X-ray diffraction and NMR studies (About RCSB PDB, n.d).
RCSB was the first open access digital data resource for Protein Data Bank. It provides access to 3D structure data for all biological molecules. RCSB is a global archive where PDB data are available for free \parencite{noauthor_rcsb_nodate}. The data acquired on RCSB are data submitted by biologists and biochemists around the world. On the \href{https://www.rcsb.org/}{website}, users can search for any protein name and the 3D structure will be displayed and can be interacted with. Information about the protein will also be displayed, and PDB files can be simply downloaded.
During the process of making ProteinAR, some other sources for protein data were used including \href{ https://zhanglab.ccmb.med.umich.edu/I-TASSER/}I-TASSER and \href{https://web.expasy.org/protparam/} {ProtParam}. 
ProtParam displays all the parameters for protein once the amino acid sequence is entered. ProtParam is a simple designed encoded website, allowing the GET method to get information from the server to the app easier, however, the PDB files contains 3D structure information of the protein is not available, therefore, it was used as a test to see if the POST and GET method work well in the app for similar website. 
I-TASSER predicts protein structure and function after users enter the sequence of amino acids. Similar with RCSB, I-TASSER allow free downloading of the PDB files, where the structure of protein is already created in 3D and can be opened using UCSF Chimera. The cons of using I-TASSER is that the data cannot be downloaded in real-time because user need to enter their emails into the server and get the PDB files back a few hours later. 
As the goal of ProteinAR is to visualize protein structures and display them instantly, RCSB was chosen for the database as it fits the goal.  


\chapter{Analysis: Goals and Functional Requirements for Solutions}
\label{ch:analysis2}

This chapter identifies the goals of the project to make it predominant to the existing solutions to protein visualisation on AR application. Then, the functional requirements as well as the non-functional requirements to help achieve these goals are discussed. 

\section{Project Goals}
ProteinAR is an iOS application that visualises the three-dimensional structure of proteins. The project was set with three main goals:

(1) Provide an \textbf{educational experience}: enables download and visualisation of user-specified protein structure with data from RCSB PDB. As in Chapter \ref{ch:litRev}, there are a few existing apps that use AR to visualise protein. However, these apps need to scan a code or an image to display the protein, which creates a significant limitation as the protein needed to be pre-rendered. The apps that allow direct protein structure viewing in 3D by entering proteins names also are available but not in AR. 
Thus, the \emph{\underline{first goal}} of this project is to make it possible for the app to connect to RCBS PDB server, download the protein model, and display it on AR after the user types in the name of the protein. 

(2) Provide an \textbf{entertaining experience}: enables the creation of new proteins by the combining of polypeptide chains. As the existing apps on protein visualisation are more focused on simply displaying the protein, this project is set on bringing entertaining elements into the app by the addition of a mini-game function in which users can create new proteins. 
The \emph{\underline{second goal}} of this project is to enable users to create new proteins from the combination of coils, helices, and sheets. For this project, because of the biological complexity of the quaternary structure protein, the new protein created will be in tertiary form. 

(3) Provide an \textbf{interactive experience}: enables interactions between the user and the protein models or polypeptide chains displayed on the screen. By touching the models, users are able to scale the proteins, move the proteins around, and rotate the proteins. The findings from Chapter \ref{ch:litRev} shows that little effort has been put into interactive elements in existing products. The main function of the products is showing the protein. Therefore, the  \emph{\underline{third goal}} of this project is to enable interaction with the 3D models in AR.

\section{Functional requirements for solutions}
\subsection{Educational purpose: Visualising Protein from RCSB PDB server}
There are a few problems must first be addressed in order to visualise the proteins. 
Firstly, the app needs to be able to send requests to the RCSB PDB server. Secondly, the app needs to be able to download the files from the server. Thirdly, the app needs to be able to track the location of the downloaded files. Finally, the app should be able to open the files and display them as an AR layer on the screen. 
	\subsubsection{Send request and download the files}
As mentioned, the app needs to be able to send information (user input) to the server and retrieve the files. Based on this approach, the first attempt was to use the \emph{POST} and \emph{GET} method. This can be achieved by using \emph{HTTP Request} in Swift. 
\emph{HTTP POST Request} allows the app to post information to the destination URL where the specified embedded method is \emph{POST}. This is achieved by first accessing the website, then inspecting its element to find the action method as well as the parameters needed for in this method. 

Similarly, \emph{HTTP GET Request }allows the app to get information from the destination URL where the method is specified as \emph{GET}. The approach is the same as with \emph{POST}; usually the parameters can be found by inspecting the source code of the website, often under \emph{form action}.

To test the function, \href{https://web.expasy.org/protparam/}{ProtParam} was used as the website only consists of string type data. The URL for both \emph{POST} and \emph{GET} are the same and the methods are in the form action. However, since there is no PDB files on ProtParam, RCSB PDB has to be the data source. On RCSB PDB, the methods of \emph{POST} and \emph{GET} do not exist in the \emph{form action} function. The PDB files are directly downloaded by a separate URL in which the only variable part (parameter) is the name of the protein. Understanding this, ProteinAR uses \emph{URLSesssion} and \emph{downloadTask()}. \emph{URLSession} makes network transfers easy and \emph{downloadTask()} fetches the contents of a specified URL, saves it to a local file and calls a completion handle. The \emph{URLSession} tracks the storing place of the download task while it happens. This will be explained more in Chapter \ref{ch:implementation}.
	
	\subsubsection{Display the file}
When the 	files are downloaded, they are saved in the \emph{.pdb} format. In Swift, when a file is downloaded, it is downloaded to a temporary location, after which it can can be moved to the \emph{Document Directory}. ProteinAR specifies the format of the download by saving it as ``proteinName.pdb". 
The solution to visualise the \emph{.pdb} is to convert it to a \emph{.dae} files and then load it on the \emph{SCN}Scene as a scene. 
In order to load the file, one solution is to use move all downloaded items into the project folder using \emph{moveItem}. However, this affects app performance as the entire content of the project is loaded every time the app is run. The solution that was used in this app is to keep all downloaded files in the \emph{Document Directory}. The file path and file name will be specified so that whenever a model is needed, it will be identified using these attributes. 
For loading the file, there needs to be a converter which converts the downloaded \emph{.pdb} file to \emph{.dae} file. This converter will automatically convert any downloaded \emph{.pdb} file in the \emph{Document Directory} into \emph{.dae} so that it can be loaded as a \emph{SCNScene} in the app. 
Unfortunately, due to technical difficulties as well as time constraints, a converter could not be made and remains the largest avenue for future work on this app. Therefore, to demonstrate the app's functionalities, a sample folder of existing "protein.dae" files is imported. 

 	
\subsection{Entertainment purpose: Create new proteins from combination}
\subsubsection{Add polypeptide chains to screen}
The app needs to be able to display user-selected individual polypeptide chain. There are four types of polypeptide chains: Flex Coil, Rig Coil, Helix, and Sheet. Each polypeptide chain is input into the project as a \emph{.dae} model. In order for these models to be loaded on ARKit, they must be converted into \emph{.scn} files. Each model consists of different nodes: the model, lighting, camera, etc. By using the pre-defined function of \emph{SCNScene}, the 3D models can be loaded into the AR view. By passing  the name of each models as a parameter, only one function is needed to add each of the four different polypeptide chains using four different buttons. 

If a model is loaded on screen and the location is not specified, the model may render off-screen. 
An additional problem can occur when two of the same polypeptide types are loaded: in such an occurrence, the two models will appear in the exact same location with the exact same orientation appearing as if there is in fact only one model on the screen. To solve this problem, the app randomises the orientations of the models every time a new model is added to screen by using the pre-defined function of \emph{eulerAngles} to specify the \emph{SCNVector3} with random x, y, and z.

\subsubsection{Combining polypeptide chains}
After adding individual polypeptide chains to the screen, ProteinAR must be able to combine these chains into proteins. If such a combination exists, then the resulting protein should be displayed. For this to happen, successful combinations of these chains are pre-loaded into the apps in a “Combinations” folder. 
In the code, an empty string array for the protein name is created. Every time a user adds a polypeptide chain to the screen, the name of the protein is appended to the array. After a user clicks the “Try” button to combine the polypeptide chains, the names in the array are concatenated using the \emph{array.joined()} function. The name of the models in the “Combinations” folder have a naming convention so that when the array are joined, the name it generated matches with the name of the models in the “Combinations” folder. See Chapter \ref{ch:implementation} for further details. 

\subsection{Interactive purpose: Interacting with the models}
After the polypeptide chains or the protein models are loaded onto the screen, users should be able to interact with the models by using the touchscreen. To make this happen, \emph{UIGestureRecognizer} was used. There are three types of \emph{Gesture Recognizer} used in ProteinAR:

\begin{table}[h!]
\centering
\begin{tabularx}{\textwidth} {
  | >{\raggedright\arraybackslash}X 
  | >{\raggedright\arraybackslash}X 
  | >{\raggedright\arraybackslash}X | }
\hline
UI Gesture & Gesture Description & Function in the app \\
\hline
\hline
Pinch Gesture & “A two-fingers gesture that moves the two fingertips closer or farther apart” \parencite{wang_beginning_2018}. & Allows users to scale (zoom in, zoom out) on the model. \\
\hline
Rotation Gesture & “A two-fingers gesture that moves the two fingertips in a circular motion” \parencite{wang_beginning_2018}. & Allows users to rotate the models in any angle. \\
\hline
Pan Gesture & “Press a finger on the screen and then slide it across the screen” \parencite{wang_beginning_2018}. & Allows users to move the models on the screen. \\
\hline
\end{tabularx}
\caption {Interacting Gestures in ProteinAR}
\label{tab:gesture}
\end{table}


\section{Non-functional requirements}
\subsection{Core Data}
Core Data is a popular framework provided by Apple to manage the model layer object in an application. 
Core data can automate solutions to common tasks associated with object life cycle and object graph management, including persistence \parencite{noauthor_core_nodate}. In this app, in order to manage the downloaded PDB files, Core Data is used in which Protein is defined as an \emph{Entity}, having two attributes \emph{name} and \emph{location}, stored as \emph{String}. After the file is downloaded and stored in \emph{Document Directory}, a new \emph{Protein Entity} is saved into Core Data, with two attribute values of \emph{name} and \emph{location}. When a model is called, it will use the specified ``proteinName" name attribute and ``filePath.dae" location attribute to open the file in the app. 
Because there are only two attributes in a \emph{Protein Entity}, it can easily be replaced by using String value directly in the app to call the files. However, taking the future work into consideration, where more attributes can be added to a \emph{Protein Entity} such as molecular weights of protein, number of amino acids in a protein, etc., using Core Data can help managing and displaying all these values more effectively.

\subsection{Constraint}
Although it is not mandatory, the app should be able to run on different iOS devices without problem. As the screen size of different iOS devices are different, if the app was designed on the view of iPhone 11 but run on iPhone 6, the buttons might be off screen or other elements might move around, making it impossible to navigate through the app. This is why constraints are important in developing an iOS app. ProteinAR does not have many elements on the screen at the same time, however, the \emph{Auto Layout} was chosen as the solution for the constraints. Using \emph{Auto Layout}, every new view that is a layer on top of a view is made into a \emph{childView} attaching to the \emph{parentView} which makes it easy for the anchor to be pinched to the \emph{parentView}. \emph{NSLayoutConstraint} was used to keep the elements in place. 

\subsection{Protein combination models and Polypeptide chains models management}
The combinations of polypeptide chains are kept in a “Combination” folder and has a naming convention that makes it easy to find each model. It is the combination of the names of the polypeptides, which makes it possible for the array to be combined into the new names. 

When users tap on the individual polypeptide buttons, the models are displayed distinctively without having to tap on the “Try” button. When the “Try” button is tapped, the models on screen combine. In order to do this, the function to add polypeptide chain is created separatedly. 

\section{Summary}
 There are three main goals the project was set to achieve. The first goal is to allow download and visualisation of protein structures from RCSB. The second goal is to allow new protein structures creation and the third goal is to enable interactions with the structures. For this to happen, a number of functional requirements are needed. These include functions to download the files, display the files, loading models as individual and combination, as well as gesture recognitions. Besides, non-functional requirements are also mentioned such as the use of core data, constraints and models management. 

\chapter{Project Design}
\label{ch:design}

This chapter elaborates the design of ProteinAR by starting with the skeleton of the app, followed by the solutions design for each and every functions. The overall structure of the app is illustrated by an UML diagram. After that, the UI design is explained. A brief description on the design of core data is presented in the end of the chapter.

\section{Application Skeleton Design}
The structure of ProteinAR is fairly simple. It consists of four main screens including the landing screen as shown in Figure \ref{fig:appskeleton}. 
\begin{figure}[!htp]
	\centering
	\includegraphics[width=\textwidth]{images/appskeleton.png}
	\caption{The skeleton of the app}
	\label{fig:appskeleton}
\end{figure}

On the landing view (first view) (Figure \ref{fig:firstview}), there are three buttons, leading to the three other views of the apps. 
\begin{figure}[!htp]
	\centering
	\includegraphics[scale=0.6]{images/firstscreenview.png}
	\caption{The First Screen View – Landing view after launch screen}
	\label{fig:firstview}
\end{figure}

(1) By tapping on \emph{Introduction}, the segue will bring up the introduction view. Instead of using multiple screens connecting from the introduction, there are three sub-screens added by using \emph{page control} on the \emph{Introduction Screen View} to give information about the app as shown in Figure \ref{fig:introviewxib}.
\begin{figure}[!htp]
	\centering
	\includegraphics[width=0.5\textwidth]{images/introviewxib.png}
	\caption{Introduction View Controller and Page Controller}
	\label{fig:introviewxib}
\end{figure}
Using \emph{Page Control} maintains coherency for the same content, while at the same time keep less words per screen, making it more appealing to users. Users can click on the \emph{GET STARTED} button to go back to the first view to explore the options or simply drag the screen down and away.

(2) Tapping on the \emph{Education} will bring users to the Education View Controller as shown in Figure \ref{fig:eduview}.
\begin{figure}[!htp]
	\centering
	\includegraphics[scale=0.6]{images/eduview.png}
	\caption{Education View Controller}
	\label{fig:eduview}
\end{figure}
Since the main function on this screen is for the user to input the protein’s name and get the pdb file back,  the design is kept simple with a \emph{textfield} and a \emph{GET button}. To make the app more appealing, there are four more buttons with four more minor actions on top of the screen. These actions are \emph{Menu, Screen Record, Screen Capture} and \emph{Exit} as shown in Figure \ref{fig:topbuttons}. These functions will be explained in section 5.2.   
\begin{figure}[!htp]
	\centering
	\includegraphics[scale=0.8]{images/topbuttons.png}
	\caption{Four buttons on top of Education View Controller and Game View Controller}
	\label{fig:topbuttons}
\end{figure}

(3) Tapping on \emph{Mini-game} will bring up the \emph{Game View Controller} (Figure \ref{fig:gameview}).
In this view, the user can create new proteins by combining the coils, helix and sheet in different orders simply by adding each polypeptide onto the screen by tapping on them, and then pressing \emph{Try}. Similar to Education View Controller, the four buttons on top of the screen are kept. 
\begin{figure}[!htp]
	\centering
	\includegraphics[scale=0.6]{images/gameview.png}
	\caption{Game View Controller}
	\label{fig:gameview}
\end{figure}


\section{Solution Design}
\begin{figure}[!htp]
	\centering
	\includegraphics[width=\textwidth]{images/uml.png}
	\caption{Application Solution Design}
	\label{fig:uml}
\end{figure}

Figure \ref{fig:uml} shows the overall structure of the app. In this Figure, the four frames represents the four screens of the app. Different colours are used to indicate different components of the app.
\begin{itemize}
\item \textbf{Purple} represents the buttons displayed on the app's screen.
\item \textbf{Yellow} represents user action (swipe, press, tap, long press, type).
\item \textbf{Green} represents the options displayed on the screen after an action was taken.
\item \textbf{White} represents the actions that the app executes after a button was pressed or an option was chosen.
\item \textbf{Pink} represents the destination screen or URL the app opens after a button was pressed or an option was chosen. 
\end{itemize}
The app starts from "First View Controller" and depends on the chosen buttons, the suitable "View Controller" will be in display. From any "View Controller", the user can always go back to "First View Controller" by pressing on the "Exit" button. The details of each function design will be elaborate in the following part of this section.

\subsection{Utility buttons}
The utility buttons are the same on both Education View Controller and Game View Controller. This creates coherency throughout the app. However, the downside is that all functions and buttons have to be duplicated on the two views, causing heavier memory load for the app. 
\begin{itemize}
	\item \emph{Menu} is the function that gives users extra options. The extra options on Education View Controller and Game View Controller are slightly different. On Education View Controller, when the user press \emph{Menu}, an \emph{Alert Service} is used, where the options rise up from the bottom of the screen, giving the user four options: \emph{More About Protein}, \emph{Help}, and \emph{PDB 101} and \emph{Cancel}. While \emph{More About Protein} get users directly to the \href{http://rcsb.org}{homepage of RCSB PDB} and \emph{PDB 101} links to the \href{http://pdb101.rcsb.org}{PDB 101 page} on the RCSB website, the \emph{Help} options bring a small pop-up screen layer on top of the AR scene. This pop-up screen contains some guidelines on how to use and navigate around the \emph{Education View} and \emph{Game View} respectively. 
Since this is more akin to a demo-app, the options only directly open link to the RCSB website, however, in future development, more in-depth options can be integrated to create a more scientific experience for the user. 
	\item \emph{Record} is the function to record the AR screen and then save the recorded video to the phone’s \emph{camera roll} if the users choose to do so. In interacting with a protein or creating a new one, users might want to record the process as there might be interesting and new findings for future study. When users long press on the \emph{Record} button, the recording process will start. By doing so, there will be a pop-up on screen asking for permission to save the recorded file to the \emph{camera roll}. The recording can be stopped simply by tapping on the record button. The app will then bring up a \emph{Preview screen}, allowing users to watch the recorded video before deciding to save the video or not. The two actions of \emph{long-pressing} and \emph{tapping} are enabled using the \emph{UIGestureRecognizer} of the ARKit. 
	\item \emph{Camera} is the function to capture the AR screen and save the photo to the phone’s \emph{Camera Roll}. When the user taps the \emph{Camera} button, the screen will be captured and saved immediately. The UIButton flashes colours to indicate that the shot has been taken.  Even though iOS already has the screen-capture function, by using that, all the buttons on the screen will also be captured, which is not desirable. With this \emph{Camera} function, users can save a photo of just the protein they want. 

\end{itemize}

\subsection{Getting pdb files from RCSB Server and storing it in CoreData}
This is one of the critical functions of the project. It requires the app to be able to download the PDB files, save it and then display it on the screen. 
In order to achieve the download function, there were much trial and error as mentioned in Chapter \ref{ch:analysis2} of using the \emph{HTTP Request POST} and \emph{HTTP Request GET} method. In the process of making the task possible, Alamofire was also considered as an option. Alamofire is a Swift-based HTTP networking library for iOS which simplifies a number of common networking tasks. However, after a few tries, the conclusion was that it was not necessary since the main task of the function is just to download a PDB file. This can be achieved using \emph{URLSession} with \emph{downloadTask()}. The downloaded destination is pre-defined to the internal \emph{Document Directory}. 
To save the downloaded file’s information to CoreData, firstly, a CoreData model was created with an \emph{Entity} Protein. This \emph{Entity} has two attributes of \emph{name} and \emph{location}. Both attribute types are of \emph{String} type. If the download process is successful (the file exists, the connection was stable, etc.), at the same time of downloading, a new \emph{NSObject} is created with the two attributes. These will be saved as an \emph{Entity} in the CoreData database. 
\emph{NSFetchRequestResult} is used to fetch the data in CoreData database back to the app. This process is visualised in Figure \ref{fig:coredata}.

\begin{figure}[!htp]
	\centering
	\includegraphics[scale=0.7]{images/CoreData.png}
	\caption{Process of downloading, saving and displaying downloaded protein model}
	\label{fig:coredata}
\end{figure}


\subsection{Visualisation of protein models from pDB files}
In order to visualise protein models from downloaded pDB files, a converter to convert file type \emph{.pdb} to file type \emph{.dae} must be made. The ideal design is as shown in Figure \ref{fig:coredata}. UCSF Chimera was used in the process of converting, however, it is only compatible with MacOS, not iOS and therefore could not be implemented into the app.

\subsection{Combining polypeptide chains into a protein}
Each polypeptide chain is designed to be referred to as a value in an array. Every time a user presses on a polypeptide chain’s button, that model of protein is displayed, while at the same time, that model’s name is added as a value in the array. After these actions, if the \emph{Clear} button is pressed, not only are the models on the screen deleted but also the values in the array are emptied. On the other hand, if \emph{Try} is pressed, all the values in the array will combined into a new name. First, the screen will be cleared and then the new name protein model will be displayed. Together with this, a text of “Congratulations, you have created a new protein made of …” will also be displayed if the combination is valid. If the combination is invalid, no models will be displayed. The errors will be caught and, on the console, “This model does not exist” will be printed. On the user's end, a 3D text of “Sorry, this combination cannot be made” will appear on screen. The simplification of the design can be found in Figure \ref{fig:array}.
\begin{figure}[!htp]
	\centering
	\includegraphics[width=\textwidth]{images/array.png}
	\caption{Create a new protein name from existing ones}
	\label{fig:array}
\end{figure}


\section{User Interface (UI) Design}
\subsection{Introduction to user interface}
In order to appropriate the tools of computers and smart devices, users need to communicate with them. The way users can communicate with the product (software, app, website) is through interacting with the user interface (UI) of that product. The purpose of a UI is to enable users to control a computer or a device they are interacting with, by giving commands and receiving feedback in a chain to complete a task. 
The user interface of any computer-based product does not only create first impressions which convinces users to continue to use that product, it also plays an important role in maintaining the interest of the users. With a complicated or inefficient UI, users would not want to keep using the product because it requires too much cognitive effort. Therefore, a UI should be \emph{intuitive} – be kept simple where no training should be needed to operate, and be \emph{efficient} – functions are precise, on point, and \emph{user-friendly} \parencite{noauthor_what_nodate}.
Currently, there are three formats of user interfaces \parencite{noauthor_what_nodate}:
\begin{itemize}
\item \textbf{Graphical User Interfaces (GUIs)} – interactions happen through visual representations on digital control panels such as a computer desktop, or a website interface.
\item \textbf{Voice-controlled interfaces(VUIs)} – interactions happen through voice representation such as Siri, Google Home or Alexa.
\item \textbf{Gesture-based interfaces} – interactions happen through physical motions in 3D spaces, such as in VR games.
\end{itemize}
The UI of ProteinAR is categorized as a GUI since users interact with the device through the visual representations of functions on a phone screen. 

\subsection{ProteinAR’s user interface design}
\subsubsection{Logo design}
The logo for the app was designed simple with just a letter P. This was created in GIMP and then exported to various sizes to maintain the resolution in different views (refer to Figure \ref{fig:appicon}). Other designs with symbols or words were considered but sticking to the “simplicity is the best” approach, the logo ended up with only one simple letter and two colours, making it easy to remember for users. This is not a new approach. Simple logo design with only one letter can be found amongst popular apps such as Facebook app or Google app. 
\begin{figure}[!htp]
	\centering
	\includegraphics[scale=0.65]{images/appicon.png}
	\caption{App’s logo in different sizes.}
	\label{fig:appicon}
\end{figure}

\subsubsection{Colour scheme}
The logo, the flash screen, the buttons and popup view elements in the app all follow the same colour scheme. In ProteinAR, an analogous colour scheme was chosen as shown in Figure \ref{fig:UIFinal}. 
\begin{figure}[hbt!]
	\centering
	\includegraphics[width=\textwidth]{images/UIFinal.png}
	\caption{UI design of ProteinAR}
	\label{fig:UIFinal}
\end{figure}

This is one of the traditional colour palettes which is the combination of related colours that are placed next to each other on the colour wheel. Analogous is known to be one of the most-used colour pallets because they are harmonious and pleasing to the eyes. ProteinAR uses two colours from the Pinks and Mauves colour sections as shown in Figure \ref{fig:colorwheel}. 
\begin{figure}[hbt!]
	\centering
	\includegraphics[scale=0.6]{images/colorwheel.png}
	\caption{Analogous Colour Scheme}
	\label{fig:colorwheel}
\end{figure}


\subsubsection{Button design}
As for the utility buttons of the Education and Mini-game screen, the main colour scheme is maintained. As the first three buttons generate actions, they are in the same mauves colour and the exit button is in pink, which creates the slight distinction of the function. The two main function buttons of \emph{Try} and \emph{Clear} also follow the main colour scheme. 

In the utility buttons, button icons are used instead of button labels. These icons are familiar to mobile app users, this makes the design more concise and easier to navigate. 

(1) Menu button: there are many styles of menu buttons as shown in Figure \ref{fig:menubuttons}. Each menu buttons generates a type of menu display. For example, the \emph{hamburger icon} opens a navigation drawer to more actions; the \emph{kebab icon}, commonly seen on Android operating system, normally opens a smaller inline menu. In this project, the chosen icon for the \emph{Menu} button is the \emph{Veggie burger} style as it is common for this style to be placed on the top left of the screen, and it symbolises generating more actions but less actions than a \emph{hamburger icon}.
\begin{figure}[hbt!]
	\centering
	\includegraphics[width=\textwidth]{images/menubuttons.png}
	\caption{Different styles of menu buttons – source: \href{https://ux.stackexchange.com/questions/115468/what-the-difference-between-the-2-menu-icons-3-dots-kebab-and-3-lines-hambur}{ux.stackexchange}}
	\label{fig:menubuttons}
\end{figure}

(2) Record and Camera button: The record button accepts two types of actions: long press and tap. Long press generates the action of recording the screen and tap ends it. The long press action also changes the colour of the button to red, which is commonly associated with recording. Tap brings it back to its original colour, which symbolises the end of the recording action. As for the camera button, the colour only flashes, implying the act of picture taking has been done. 

As mentioned, the buttons in ProteinAR mainly follow the colour scheme of Pinks-Mauves. However, the four buttons to add polypeptide chains are the exceptions. These four buttons use images as the buttons and to increase usability for users, they are labelled with their names. The designs of the buttons are inherited from Tianshu Xu’s 2019 Master project \parencite{xu_interactive_2019} and the label colours were chosen to be matched with the colours of the polypeptides. Since this is a mini-game, the colourful elements are chosen for visual appeal and clarity. 
\begin{figure}[hbt!]
	\centering
	\includegraphics[scale=0.7]{images/polybuttons.png}
	\caption{Polypeptide chains button}
	\label{fig:polybuttons}
\end{figure}

\section{Core Data Design}
In ProteinAR, Core Data is simply designed with only one \emph{Entity} called Protein. This \emph{Entity} has two \emph{attributes}: \emph{name} and \emph{location} of type \emph{String}. In the current design of the app, Core Data only stores two information in a Protein \emph{Entity}: the name of downloaded protein and its file path in the \emph{Document Directory}. When the data is called, information stored in the attribute \emph{location} of Core Data simply act as a \emph{String} value to concatenate with the file's name and extension to fetch the file that are stored in the \emph{Document Directory}. 
Figure \ref{fig:datamodel} shows the \emph{Entity} and \emph{attributes} in the data model in Core Data.
 \begin{figure}[!htp]
	\centering
	\includegraphics[width=\textwidth]{images/datamodel.png}
	\caption{Core Data Entity and attributes}
	\label{fig:datamodel}
\end{figure}

The details on implementation of Core Data can be found in Figure \ref{fig:displayreal} with explanation in Chapter \ref{ch:implementation}. As explain in Chapter \ref{ch:analysis2}, with the current implementations of ProteinAR, integrating Core Data is unnecessary. Document path can be directly called into the displaying function using \emph{File Manager}. The reason Core Data was used is for future work, when more protein data may need to be stored. 





\chapter{Project Implementation}
\label{ch:implementation}

In this chapter, the implementation will be demonstrated with code snippets from the source code of the project. Firstly, the implementation to enable download and visualisation or protein models will be explained step-by-step. Secondly, the method for creating new proteins will be discussed. Last but not least, the chapter will elaborate on the implementation of interactive elements in the app.

\section{Download and Visualisation of Protein Models}
Due to its complexity, the process to download and visualise protein models will be explained in five steps.
\subsection{Step 1: Set up Core Data}
First, Core Data is set up by adding a new \emph{Data model} from the \emph{Core Data} section. In this app, the database has only one \emph{Entity} Protein which has two \emph{attributes}: \emph{name} and \emph{location}, defined in type \emph{String} as in Figure \ref{fig:datamodel}. Core Data Stack and Core Data Saving Support need to be added to the \emph{App Delegate} if not automatically generated by Xcode as shown in Figure \ref{fig:coredatacode}. After that, two subclasses must be created where Protein is defined as a public class in \emph{NSManagedObjectSubclass} and function \emph{fetchRequest} is defined as a public class in the extension of Protein. In these subclasses, the attributes of \emph{name} and \emph{location} are also declared as public variables (refer to Figure \ref{fig:subclass}).
 \begin{figure}[!htp]
	\centering
	\includegraphics[width=\textwidth]{images/coredatacode.png}
	\caption{Core Data Stack and Core Data Saving Support}
	\label{fig:coderatacode}
\end{figure}

 \begin{figure}[!htp]
	\centering
	\includegraphics[width=\textwidth]{images/subclass.png}
	\caption{NSManagedObject subclass}
	\label{fig:subclass}
\end{figure}

\subsection{Step 2: Download from RCSB PDB and save the PDB file locally}
\subsubsection{Download PDB file from RCSB PDB}
Donwload is the critical function in this process. The function is split into two functions: \emph{getDownloadURL} and \emph{download}.

In the function \emph{getDownloadURL} (Figure \ref{fig:download1}), the URL to the source file is created. After observing how files are downloaded from RCSB, a URL pattern was found. Instead of using the \emph{GET} method in \emph{action form}, RCSB allows downloading the PDB files directly from a URL. The structure of the URLs are the same for all of the different PDB files, starting with the same path. The file name is the only part that needs to be changed. The file name is set as the protein's name. With this logic, the URL to the source file was constructed using the parameter as the user-input-text to change the file name accordingly. After creating the URL to the download files, the download function is called with two arguments of \emph{URL} and \emph{parameters} where URL is the to-be constructed URL and parameter is the user's inputted protein name.
 \begin{figure}[!htp]
	\centering
	\includegraphics[width=\textwidth]{images/download1.png}
	\caption{Function getDownloadURL}
	\label{fig:download1}
\end{figure}

The code snippet in Figure \ref{fig:download2} shows the download function that was called by \emph{getDownloadURL}. In this function, the code uses \emph{URLSession} and \emph{downloadTask()} to generate the download task. \emph{URLSession} provides an API for downloading and uploading data to specified URLs. This API helps perform background downloads. In this code, \emph{default} type for \emph{URLSession} is used instead of \emph{shared} because it allows more freedom of configuration. \emph{URLSessionConfiguration} defines the behaviour policies when the app downloads data from the server. There are a few types of \emph{URL Session Tasks}. In this app, \emph{downloadTask} is used as it retrieves data in the form of a file and supports background downloads. 
It is important to take note of the status code returned by \emph{HTTPURLResponse} as it can allow us to address the different possible errors appropriately. The two most common errors are: the file does not exist (status code 404), and the connection to the server was interrupted (status code 500). 

\subsubsection{Save the file in Document Directory}
When the code performs its download task, the file is stored in a temporary location, as called in the code \emph{temporaryURL} (Figure \ref{fig:download2}, line 402). To save the file locally, the file should be moved to a permanent location in the \emph{Document Directory}, using an \emph{absolute path}. To achieve this, \emph{File Manager} was used, in which the destination folder was allocated to \emph{.documentDirectory}, under \emph{userDomainMask} (Figure \ref{fig:download2}, line 408). Stating the path to the \emph{Document Directory} is not enough to make \emph{an absolute path} as the whole URL to the file must be indicated. Therefore, the format of the file that will be downloaded is specified in Figure \ref{fig:download2}, line 409 as \emph{destinationURL} by adding the path components including: the inputted protein's name  as file name and \emph{.pdb} as file extension. For example, if the user domain name is abc123, the proteinID is "6MK1", the path to the downloaded PDB files would be "abc123/Documents/6MK1.pdb". After this absolute path was created, the downloaded file in the temporary location can be moved to the permanent location in \emph{Document Directory} using \emph{FileManager.default.moveItem}. 

 \begin{figure}[!htp]
	\centering
	\includegraphics[width=\textwidth]{images/download2.png}
	\caption{Function download}
	\label{fig:download2}
\end{figure}


The alternative way is to move the downloaded file into the main app bundle as shown in Figure \ref{fig:movetobundle}. The directory of the main app's bundle is created and the file can be moved by the same \emph{moveItem()} method. In the source code, this alternative way is disabled. The reason for this was previously explained: if all the downloaded files are saved into the main app's folder, the app will have to load them every time the files are loaded, making the app heavy and slow. 
 \begin{figure}[!htp]
	\centering
	\includegraphics[width=\textwidth]{images/movetobundle.png}
	\caption{Alternative: Move downloaded files to main app's folder}
	\label{fig:movetobundle}
\end{figure}


\subsection{Step 3: Assign downloaded files to Core Data}
Firstly, the \emph{context} is declared by \emph{persistentContainer} and the \emph{proteinManagedObject} is declared and initialised as \emph{nil}. 
Then, the function to save \emph{proteinManagedObject} \emph{context} is called inside of the \emph{do} action in the download function (Line 433 -Figure \ref{fig:download2}). The function to save context is displayed in Figure \ref{fig:savefunc}. 
When the download is successful, the attributes \emph{name} and \emph{location} are saved into the Protein \emph{Entity} as a \emph{String} type. 
Since \emph{proteinManageObject} is a global variable, it can be accessed anywhere in the code. 
 \begin{figure}[!htp]
	\centering
	\includegraphics[width=\textwidth]{images/savefunc.png}
	\caption{Save and Assign downloaded file to attributes in Core Data}
	\label{fig:savefunc}
\end{figure}

\subsection{Step 4: Convert PDB file to Collada file}
After the \emph{.pdb} file is downloaded and saved to \emph{Document Directory}, it must be converted to a \emph{.dae} file because ARScene only allows displaying Collada models to its AR Scene. 
A few solutions were used to solve this problem. One of those is to borrow the `PDB to Collada conversion' from UCSF Chimera. Since Chimera was written in Python, its scripts could be run in Swift since Python has a C interface API. However, the challenge was that Chimera is not compatible with iOS, so this solution could not be used. 

OpenBabel was another solution that was investigated . Unfortunately, OpenBabel is only compatible with Android and MacOS, not iOS and in effect was not able to be implemented. 

RCSB PDB published an article on the releasing of their mobile version in 2015 which can help visualise the PDB file on both iOS and Android, however, as of 2020, it was no longer available on the App Store.

This, therefore remains an outstanding issue of the app.

\subsection{Step 5: Fetch and Visualise PDB files}

Data saved into the Core Data can be fetched using \emph{NSFetchRequestResult} and can be displayed easily using the attributes assigned in Core Data. Once the PDB file is converted, the resulting Collada file can be displayed using the function shown in Figure \ref{fig:displayreal}. 
After some trials, all the models after being converted to Collada files are quite large which cannot fit on the screen and need to be scaled down. Each of these models consist of many nodes, however, \emph{node\_1} is always the main node for the whole model. Therefore, in the function (Figure \ref{fig:displayreal}, line 201 - 202), \emph{node\_1} is set as a variable and is scaled down. This assures that in the future, after being ownloaded and converted, the models can always be display completely on screen. 
 \begin{figure}[!htp]
	\centering
	\includegraphics[width=\textwidth]{images/displayreal.png}
	\caption{Function to display protein after being converted into Collada models}
	\label{fig:displayreal}
\end{figure}

As the app does not yet have a functioning “PDB to Collada' converter, already converted protein files in Collada format are made available to demonstrate the visualisation step.

\section{Create new Protein Models}
\subsection{Step 1: Import and name models}
ProteinAR uses ARKit and SceneKit to load the models. For this to happen, models need to be imported. Firstly, a new directory \emph{Scene Catalogue} must be made. In this project, the directory is named “Combinations”. Imported models are in \emph{.dae} format, however, to improve loading times for SceneKit, they are converted to \emph{.scn} type. The files are named according to the polypeptide chains that make them in the order that they make them. See Figure \ref{fig:combinations} for more details as well as some examples.
 \begin{figure}[!htp]
	\centering
	\includegraphics[scale=0.7]{images/combinations.png}
	\caption{Combinations of polypeptide chains stored in a folder}
	\label{fig:combinations}
\end{figure}
This naming convention makes it easy to pass arguments and load models in the upcoming functions. To make the models clearer, Phong shading is used and the colour of each model is assigned randomly. 

\subsection{Step 2: Add polypeptide function}
Figure \ref{fig:addProteinfunc} shows the function used to add polypeptides to the screen. In this function, the argument is pre-defined as a \emph{String} type and has a name matching the \emph{protein's name}. In \emph{ARKit}, \emph{SCNScene} is used to load the 3D models. Since all the models have the same format (inside the “Combinations.scnassets” directory with “.scn” extension), the models will easily be called by passing the names of the protein as arguments each time a Polypeptide button is pressed, as shown in Figure \ref{fig:polypeptidefunc}. 
This function also adds a camera node to the screen at the position of (0, 0, 0) as well as fixes model position using \emph{SCNVector3} to ensure that the models will always appear in front of the camera. One problem that users might encounter with using AR technology is that the space is infinite so models may be loaded off-screen. It is therefore important to ensure the position of the loaded models are visible. This function also uses \emph{eulerAngles} with a random \emph{SCNVector3} to ensure that every time new models are loaded they are at the same position but with different orientations so they do not perfectly overlap each other. This makes it clear to user that a new model has been added.
\begin{figure}[!htp]
	\centering
	\includegraphics[width=\textwidth]{images/addProteinfunc.png}
	\caption{Function to add protein to the screen}
	\label{fig:addProteinfunc}
\end{figure}

\begin{figure}[!htp]
	\centering
	\includegraphics[width=\textwidth]{images/polypeptidefunc.png}
	\caption{Actions of each of the polypeptide button}
	\label{fig:polypeptidefunc}
\end{figure}


\subsection{Step 3: Create new protein}
To simplify the process, the protein combinations are not generated by the code but instead loaded from the “Combinations” folder and displayed. To create a smooth transition and generate the feeling of joining the polypeptide chains, the process has three minor steps. 

\subsubsection{Clear everything off the screen}
Clearing the screen will make the transition to a new model more natural. 
\begin{figure}[!htp]
	\centering
	\includegraphics[width=\textwidth]{images/clearScreen.png}
	\caption{Function to clear models off the screen}
	\label{fig:clearScreen}
\end{figure}

As the models are added to the screen as nodes (model node, camera node, light node), simply removing all the nodes from \emph{ParentNode()} will clear the screen entirely.

\subsubsection{Load a combination model}
In Figure \ref{fig:polypeptidefunc}, it is shown that every time a button is pressed, besides loading a model onto the screen, it does something else. An empty array is declared in the beginning and every time a button is pressed, a value is added to the array. For example, when the \emph{fCoil button} is pressed, “fCoil” is appended to the array. The values in the array will then be concatenated using \emph{array.joined()} to create the combination name as shown in Figure \ref{fig:createProtein}. 
If the combination exits, the model will be displayed on screen. Similar to the \emph{addProtein} function, to ensure the models appear in front of the camera, camera and protein are added at a fixed position using \emph{cameraNode} and \emph{proteinNode}. Since the models generated from PDB file are large and cannot fit on the screen, they are scaled down when loaded. 

\begin{figure}[!htp]
	\centering
	\includegraphics[width=\textwidth]{images/createProtein.png}
	\caption{Function to create a new protein}
	\label{fig:createProtein}
\end{figure}

\subsubsection{Display 3D text}
These functions are called inside of the function \emph{createProtein}. If a user-generated combination is valid, together with the model, the 3D text “Congratulations” will be loaded, followed by the names of the polypeptides in order of input. If the combination is invalid, no model will be loaded and instead, only the text “Sorry”! The combination of (\emph{user-pressed-buttons}) cannot be made" will appear. As the function to load text is quite simple and similar to loading models, the code is not shown here. See Appendix A for the full code of this function. 

\section{Interactive elements}
\subsection{Interacting with Protein Models using three gestures}
ARKit is a very powerful framework as it enables gesture interaction. To do this, the gestures must be dragged onto the \emph{Main storyboard} from the built-in library. The three gestures used in ProteinAR are \emph{Pinch Gesture} , \emph{Rotation Gesture} and \emph{Pan Gesture}. The gestures are initiated by \emph{.state: .change}. In ProteinAR, the goal for setting interactions is that the interactions cover the whole screen. This is why the area of enabling gesture is set as \emph{SCNView} while \emph{hitTest} is used to run the gesture. 

The \emph{Pinch Gesture} function shown in figure \ref{fig:pinch} uses \emph{SCNVector3} to change the float values of x, y, and z. Using the two fingertips, users can zoom in and out on the models. 
\begin{figure}[!htp]
	\centering
	\includegraphics[width=\textwidth]{images/pinch.png}
	\caption{Pinch Gesture function}
	\label{fig:pinch}
\end{figure}
The other two functions for rotating and panning respectively are similar to pinch gesture, with the defining characteristics of \emph{.state} being initialised by \emph{.changed}. These other two functions will not be displayed here. The codes can be found in the Appendix A . 

\subsection{Other interactive elements}
Although it is not a requirement of the app, a more interactive display makes for a presentable UI, so, additional gestures and touches were added to the app. While this may be counterintuitive, the additions improve overall user experience. 

\subsubsection{Gesture Recognizer button}
For the \emph{Record} button, the two gestures of \emph{Tap} and \emph{Long press} were added. Among several alternative methods, creating two objective-C functions was the simplest solution. For this to work, the button should not be connected to the code as an action, but as an outlet. Then, in the \emph{viewDidLoad()}, the \emph{GestureRecognizer} can be added to the outlet as shown in Figure \ref{fig:taplong}. The \emph{GestureRecognizer} function is handled in objective C code (refer to Figure \ref{fig:objcfunc})
\begin{figure}[!htp]
	\centering
	\includegraphics[width=\textwidth]{images/taplong.png}
	\caption{Add Gesture Recognizer to Button outlet}
	\label{fig:taplong}
\end{figure}
\begin{figure}[!htp]
	\centering
	\includegraphics[width=\textwidth]{images/objcfunc.png}
	\caption{Objective-C functions to handle Gesture Recognizer}
	\label{fig:objcfunc}
\end{figure}

\subsubsection{Dismiss subview and keyboard}
When a user finishes reading the guidelines on \emph{Help Screen View} or finishes inputting in the \emph{textFied}, the sub-screen and the keyboard should be dismissed. For the \emph{Help Screen View}, the solution was to use \emph{UITouch}. This is set so that if a user touches any place that is not the \emph{Help Screen View}, the view will be hidden. 

For the keyboard, it can usually be set with \emph{textFieldShouldEndEditting} after specifying \emph{TextFieldDelegate} in the class. However, in ProteinAR, since the whole screen is covered by \emph{UIGesture}, this did not work. The solution was to set the \emph{Return} key as a \emph{Done} key in \emph{viewDidLoad} and then use the function of \emph{textFieldShouldReturn} to dismiss the keyboard, as shown in Figure \ref{fig:dismiss}.

\begin{figure}[!htp]
	\centering
	\includegraphics[width=\textwidth]{images/dismiss.png}
	\caption{Others interactive elements}
	\label{fig:dismiss}
\end{figure}

\section{Summary}
Based on the design, coding solutions were implemented to achieve the three main goals: download and visualise protein models, create new protein models, and add more interactive elements. To enable downloading and visualising protein models, the process was divided into five steps of implementation. Although the direction of the five steps were planned, step four (convert PDB file to Collada file) was not successfully implemented, resulting in the incompletion of the function. For future development, this step should be the main focus. The implementation to achieve creating new proteins and adding more interactive elements were successfully conducted. Although there are much rooms for further development, these functions lay a solid foundation for the app.

\chapter{Project Testing and Evaluation}
\label{ch:evaluation}

\section{Function Testing}
\subsection{Education Screen (Figure \ref{fig:eduscreen})}
When the protein's name is inputted, the model of the protein is displayed.
In Figure \ref{fig:eduscreen} on the left, the protein 6K01 is displayed on the AR screen because 6K01 is a valid protein's ID and on the right, when the protein is not valid, the screen shows only 3D text informing the protein does not exist.
 \begin{figure}[!htp]
	\centering
	\includegraphics[scale=0.6]{images/eduscreen.png}
	\caption{Education App Screen}
	\label{fig:eduscreen}
\end{figure}

\subsection{Mini-game screen (Figure \ref{fig:minigamescreen1}, \ref{fig:minigamescreen2})}
On the Mini-game screen, when polypeptide chain button is pressed, the model of that chain will be displayed. Similarly, when another polypeptide chain button is pressed, the second model appears on top of the first one as shown in Figure \ref{fig:minigamescreen1}. If the combination exits, it will be displayed with text showing the names of its elements. If the combination does not exist, only text will appear as shown in Figure \ref{fig:minigamescreen2}.
 \begin{figure}[!htp]
	\centering
	\includegraphics[width=\textwidth]{images/minigamescreen1.png}
	\caption{Mini-game App Screen (1)}
	\label{fig:minigamescreen1}
\end{figure}

 \begin{figure}[!htp]
	\centering
	\includegraphics[scale=0.6]{images/minigamescreen2.png}
	\caption{Mini-game App Screen (2)}
	\label{fig:minigamescreen2}
\end{figure}

\section{Unit Testing}
By running the app, only functions that can be displayed on the screen can be tested. Functions to download the PDB files was not tested. Therefore, in the project, a unit test was built to test the download function. 
In Xcode, the XCTest framework is used to write unit tests. XCTest assert that during code execution, certain conditions are satisfied and if not, the errors messages will be shown together with the test failure result. 
The full code snippet for the unit test can be found in Appendix A. In Figure \ref{fig:testdownload}, the main part of the code for the unit test is shown. In this test function, the URL is given with a valid protein's ID (6K03) and the code checks if the files 6K03.pdb exists in the \emph{Document directory} after the download function ran. The green tick on the function shows that the tested function (download) works. 

 \begin{figure}[!htp]
	\centering
	\includegraphics[width=\textwidth]{images/testdownload.png}
	\caption{Unit Test - Download function}
	\label{fig:testdownload}
\end{figure}

\section{Application Performance Testing}
Since ProteinAR is an iPhone app, evaluating on how the app performs on an iPhone is important. Xcode has a built-in debug navigator to show how the app performing on the device. In this navigator, there are reports to visualise how the application impact the running of the simulator device. 
Figure \ref{fig:cpu} shows the impact on iPhone's CPU while the app is running. The percentage keeps changing, however, it always run high, from 80 percent to 120 percent. The testing device is iPhone X with six cores, bring the maximum capacity of CPU to 600 percent. Since there are many tasks that the app has to do at the same time: recognising the real world's surfaces, putting layers on, downloading from the web, displaying models, etc., this is considered acceptable. In Figure \ref{fig:cpu}, the percentage used shown to be still in the green zone. 
 \begin{figure}[!htp]
	\centering
	\includegraphics[width=\textwidth]{images/CPU.png}
	\caption{CPU usage}
	\label{fig:cpu}
\end{figure}

As for the memory, the apps does not take much memory usage at the moment. In the future work, when a converter file is made and the app can actually display models from downloaded PDB files, memory usage will still not be a problem because the data will be stored in \emph{Document directory}. A visualisation of memory usage is shown in Figure \ref{fig:memory}.

 \begin{figure}[!htp]
	\centering
	\includegraphics[width=\textwidth]{images/memory.png}
	\caption{Memory usage}
	\label{fig:memory}
\end{figure}

An AR app should have an FPS (frame per second) rate of 30 FPS to allow the app to run smoothly and save CPU and GPU usage. In ProteinAR, the FPS rate is relatively high, at 60 FPS. This creates the smooth movements, but also takes up a lot of GPU usage (Figure \ref{fig:FPS}. This might also lead to the extra energy usage, affecting negatively to the energy impact (Figure \ref{fig:energy2}). 

 \begin{figure}[!htp]
	\centering
	\includegraphics[width=\textwidth]{images/FPS.png}
	\caption{GPU Usage}
	\label{fig:FPS}
\end{figure}

 \begin{figure}[!htp]
	\centering
	\includegraphics[width=\textwidth]{images/energy2.png}
	\caption{Energy impact}
	\label{fig:energy2}
\end{figure}

Through observation, when the app starts running, the energy impact is already stated in the report as ``Very High". The thermal state starts at "Fair" and in only a few minutes changes to "Serious". Sometimes, the thermal state goes up to "Critical" if the app is left running long. 
This affects negatively to the performing of the app and also drain the device's battery fairy quick. Since the device becomes extremely hot, the models could be loaded but lagging, extra screen (for example: the "Help Screen") would not even show after clicked because the UI elements generate extra heat.

In summary, the performance evaluation could be summed up in table \ref{tab:perEvaluation}.

\begin{table}[!h]
\centering
\begin{tabularx}{\textwidth} {
  | >{\raggedright\arraybackslash}X 
  | >{\raggedright\arraybackslash}X 
  | >{\raggedright\arraybackslash}X 
  | >{\raggedright\arraybackslash}X | }
\hline
Performance Category & Device impact & Pro & Con \\
\hline
\hline
CPU/GPU& High & Multi-tasking, smooth transition & Increase energy impact  \\
\hline
Memory & Normal & No extra workload on the device & N/A \\
\hline
FPS & High & Smooth display of models and AR layers & Increase CPU and GPU usage \\
\hline
Energy & Very High & N/A & Freeze the app and Drain battery \\
\hline
\end{tabularx}
\caption {Performance Evaluation}
\label{tab:perEvaluation}
\end{table}

To help lower the energy impact, tests and debug needs to be implemented in the future work. 

\section{Application Usability Testing}

Usability evaluation is an important evaluation that any system must take before releasing the product. This helps ensure the app meeting with the requirements of business and secure customer satisfaction. By doing usability testing, the app can be improved based on objective opinions. 

In this project, the usability evaluation is conducted based on the post-testing usability questions with 10 questions in SUS (System Usability Scale) type of questionnaire. This means the questions are asked after the users have experienced using the app. 
However, the project ran into two problems while conducting the usability testing. 

\begin{itemize}
\item Limitation of testing subjects: 

For users to remotely download and use the app, ProteinAR has to be available on the App store. After an app is submitted to the App store, it would go through a very strict review process and would not be available for a period of time. This would also require more work to complete the app since an app that still runs into problem would not be able to go through the review. Furthermore, uploading an app on the App store would make it become a commercial product. This might go against the rule of UCC. Thus, it was not possible for other users to remotely test the app. There are two ways for the app to be tested by other users which require in-person meeting:
	\begin{itemize}
		\item User directly used the app on developer's device or downloading from the developer's computer. With the on going complications of Covid-19, it is not advisable to meet up with many people, hence, touching the same phone. 
		\item User can install the app by downloading the project on Github. However, this requires target users to have access to MacOS system and an iPhone 6 or onwards with the compatible MacOS and iOS version.

Since the second way was difficult to be conducted, all the tests were carried out in the first way, which was also made difficult because of Covid-19. Thus, the number of users to conduct the test was limited to five people.
	\end{itemize}
\item Lack of objectivity in questionnaire result:

For all of the tests, users and developers were in the same place. This affected the sense of objectivity in the result of the questionnaire, thus, the evaluation result might not be accurate. 
\end{itemize}

However, the test was still conducted and the questionnaire result is shown in Figure \ref{fig:survey}. Before the test, users are given some protein's name to start with. Since there are not many buttons on the screen, it did not take much time for users to figure out how to navigate around and use the app.The element of using Augmented Reality impressed users as it is new and considered ``exciting" and ``cool", which created a huge impact in making users wanting to play again. However, the app is about protein, which might not be the subject of choice for most users, thus, recommendation to others did not get a good feedback. The entertaining element does not meet up with the users' satisfactions.
 \begin{figure}[!htp]
	\centering
	\includegraphics[width=\textwidth]{images/survey.png}
	\caption{System Usability Scale (SUS) Questionnaire and Feedbacks}
	\label{fig:survey}
\end{figure}

Some observations after user testing: The extra functions linking the app to RCSB website or PDB 101 weren't intentionally chosen. Users found it too much work to have to read through another website. This could be replaced by more simple but educational screens or websites. The AR triggers excitement of using the app, and thus, should be put in more focus such as collision with real world's object. The mini-game would be more interesting if the new protein created can be linked to daily things such as ``found in human skin, found in sheep wool". 

\section{Project Overall Evaluation}
Base on the testings that were conducted, the project overall evaluation is summarised in table \ref{tab:evaluation}

\begin{table}[!h]
\centering
\begin{tabular}{| m{0.2\textwidth} | m{0.5\textwidth} | m{0.3\textwidth}|}
\hline
Goals & Implementation & Evaluation\\
\hline
Download Protein from RCSB PDB & 
\begin{itemize}
	\item Download using \emph{downloadTask()} using constructed URL
	\item Save files to Document directory
	\item Assign attributes to Core Data
\end{itemize} &
Base on the result from Unit Testing, these functions work without errors.

\textbf{Completed}\\
\hline

Visualise Protein models on AR & 
\begin{itemize}
	\item Convert PDB file to Collada files
	\item Visualise by loading 3D models on AR screen using function \emph{displayProtein} (Figure: \ref{fig:displayreal})
\end{itemize} &
\begin{itemize}
	\item Convert script was not made
	\item Can load some pre-downloaded sample models
\end{itemize}	

\textbf{Partially completed} \\
\hline

Create new Protein on AR & 
\begin{itemize}
	\item Display individual polypeptide chains with function \emph{addProtein()} (Figure: \ref{fig:addProteinfunc}
	\item Display combinations of polypeptide chains according to user input using function \emph{createProtein} (Figure: \ref{fig:createProtein}) 
\end{itemize} &
Displaying individual polypeptide chains and clearing them to display the combination create the smooth transition and interactive feelings to users.

\textbf{Completed}\\
\hline

Interact with Protein models &
 Use UIGestureRecognizer for 
 \begin{itemize} 
	 \item Pinch Gesture (allow zooming in and out) (Figure:\ref{fig:pinch})
	 \item Rotation Gesture (allow rotating models) (see the Appendix A)
	 \item Pan Gesture (allow moving models) (see the Appendix A)
\end{itemize} &
Base on users testing questionnaire, the UI is appealing 

 \textbf{Completed}\\
 \hline
 
  Appealing UI &
 \begin{itemize}
 	\item Use two different screens for two purpose (education and mini-game)
	\item Easy to navigate
	\item Analogous colour theme
	\item Interactive elements (buttons, display, model-interactions, create new protein) (Chapter: \ref{ch:design})
\end{itemize} &
Base on users testing's questionnaire, the UI is easy to use and appealing.

\textbf{Completed}\\
\hline

Good performance&
 \begin{itemize} 
	 \item CPU Usage: High
	 \item GPU Usage: High
	 \item Memory Usage: Normal
	  \item Energy Impact: Very High
\end{itemize} &
The thermal state reaches “serious state” in a short time, which is not ideal for the app. Debugging sections are needed to solve this problem.

\textbf{Partially Completed}\\
\hline
\end{tabular}
\caption{Overall Evaluation}
\label{tab:evaluation}
\end{table}
Although much future work are to be done for the completion of the app, ProteinAR succeeded in creating the first step to bring displaying protein models in AR. Without the need to print out materials, or using extra devices such as VR goggles, ProteinAR can be developed to become an useful tool for biology students and teachers as well as scientists in studying and researching about proteins.



\chapter{Final Conclusion}
\label{ch:conclusion}
In this chapter, project summary with achieved results will be concluded, followed by the ideas of direction for future work.
\section{Conclusion}
This project aimed to bring Augmented Reality technology into biological research by visualising protein structures in AR and allow for user interactions. To achieve this, the iOS app ProteinAR was developed to download PDB files from the RCSB Protein Data Bank and display them in AR environment. ProteinAR also allows interactions, in which users can pinch, rotate, move, or create new proteins from  polypeptide chains. This app was designed using Xcode and written in Swift using ARKit, Apple's relatively new framework for developing app-based AR technology. Three main goals were set for the project: enabling the display of protein models from inputted protein IDs, enabling the creation of new protein models, and enabling user interactions with these models. To do this, the process was divided into minor steps, in which minor functions were written to achieve the goals. Additionally, design principles were integrated with extra functions to ensure the usability of the app. 
ProteinAR succeeded in downloading models from RCSB PDB. Any models that were downloaded and converted (using UCSF Chimera) can be displayed on the AR screen. ProteinAR also made it possible for users to combine polypeptide chains into new protein structures. Whether the models are downloaded or created, users can interact with the protein structure to learn more about them. Compared to existing products on visualising protein models in AR, ProteinAR allows more user interactions, as manipulation of protein structures is enabled, and the function to create new protein models from polypeptide chains is a predominant feature.

However, the project has an outstanding issue that needs to be addressed before the app can reach completion. In displaying the protein models downloaded from RCSB PDB, a conversion function from PDB to Collada file is necessary as the ARKit only allows display of the the Collada file type. Due to time constraint of the project, this remains undeveloped, thus ProteinAR can only display protein models from previously downloaded and converted models. 
Nevertheless, ProteinAR proved that retrieving and displaying data from the RCSB PDB is achievable. The advancement of AR technology signals the potential for ProteinAR to be further developed into an invaluable tool for academics of biology, and for anyone with a curious itch to see the unseeable.

\section{Future Work}
Given the limitations of the project, there is plenty of rooms for improvement with ProteinAR. Furthermore, Augmented Reality technology is relatively new and Apple has been acquiring new companies specialising in AR to update the ARKit framework rapidly, creating further possibilities for the development of the app in the future.

First and foremost, the \textbf{protein real-time visualisation} remains unfinished due to it missing a function to convert the PDB file type to the Collada file type. With this function completed, ProteinAR could become extremely useful in biology class as it displays any protein structure in real time. This should be the main focus of future work.

Secondly, \textbf {new protein creation} can be improved in the following directions:
\begin{itemize}
	\item Currently, the combinations of polypeptide chains are pre-loaded into a folder. In the future, if these combinations can be generated in real time, using a server such as I-TASSER, more combinations can be created not limited only to tertiary but quaternary structures.
	\item The newly created protein, if pre-existing, should link to some information such as its parameter, its function, etc. This can be achieved using POST and GET REQUEST to the available source of information.
	\item The newly created protein should be exportable as a PDB file. This way, it could be used for research purposes.
\end{itemize}

Thirdly, more \textbf{interactive elements} can be added. ProteinAR can integrate Machine Learning and CoreML to control ARKit. There are many ways to implement this. One popular way is to combine image classification and AR to create new experiences by experimenting with hand gesture recognition. Photos of hand poses can be taken then used in training models such as TensorFlow, Keras, Custom Vision. Xcode also has a training interface. The models can be trained so that users can use hand poses to control the protein models. Additionally, the protein models could be made more realistic by allowing the user to bend and twist the nodes of the structure. Currently, ProteinAR only allow users to pinch, rotate and pan the models.
 
Last but not least, \textbf{more functions} can be integrated into the app. Currently, the menu function only links the app to the RCSB home page, with no specific information. These functions can be further customised to be more appropriated and interesting. 



\appendix
\label{ch:appendix}

\chapter{Code snippets}
\doublespacing

\begin{figure}[!htp]
	\centering
	\includegraphics[width=\textwidth]{images/displayTextfunc.png}
	\caption{Function to display 3D Text}
	\label{fig:displayTextfunc}
\end{figure}

\begin{figure}[!htp]
	\centering
	\includegraphics[width=\textwidth]{images/rotate.png}
	\caption{Rotation Gesture}
	\label{fig:rotate}
\end{figure}

\begin{figure}[!htp]
	\centering
	\includegraphics[width=\textwidth]{images/pan.png}
	\caption{Pan Gesture}
	\label{fig:pan}
\end{figure}

\begin{figure}[!htp]
	\centering
	\includegraphics[width=\textwidth]{images/testdownloadfull.png}
	\caption{Full test script for unit test}
	\label{fig:testdownloadfull}
\end{figure}



\printbibliography
\end{document}
